\documentclass[11pt]{article}

\usepackage{amsmath}
\usepackage{amsfonts} 
\usepackage{amsthm}
\usepackage{enumitem} 
\usepackage{mathtools}
\usepackage[top=2cm,bottom=2cm,left=2.5cm,right=2.5cm,marginparwidth=1.75cm]{geometry}
\setlength{\parindent}{0cm}

\newtheorem{theorem}{Theorem}

\title{\vspace{-1.0cm}MATH 5615H Homework 3}
\author{Fletcher Gornick}
\date{September 23, 2022}

\begin{document}
 \maketitle
 \begin{enumerate}[leftmargin=0pt, label=\arabic*)]
 \item if \(s_1 = \sqrt{2}\), and \(s_{n+1} = \sqrt{2 + \sqrt{s_n}} \;\; (n = 1,2,3,\dots)\),
 prove that \(\{s_n\}\) converges, and that \(s_n < 2\) for \(n = 1, 2, 3, \dots\).
 \newpage

 \item Fix a positive number \(\alpha\).  Choose \(x_1 > \sqrt{\alpha}\), and define 
 \(x_2, x_3, x_4, \dots\), by the recursion formula
 \[x_{n+1} = \frac{1}{2}\left(x_n + \frac{\alpha}{x_n}\right).\]

 \begin{enumerate}[label=(\alph*)]
   \item Prove that \(\{x_n\}\) decreases monotonically and that \(\lim x_n = \sqrt{\alpha}\).
   \item Put \(\epsilon_n = x_n - \sqrt{\alpha}\), and show that
     \[\epsilon_{n+1} = \frac{\epsilon_n^2}{2x_n} < \frac{\epsilon_n^2}{2\sqrt{\alpha}}\]
     so that setting \(\beta = 2 \sqrt{\alpha}\),
     \[\epsilon_{n+1} < \beta\left(\frac{\epsilon_1}{\beta}\right)^{2^n} \quad (n = 1, 2, 3, \dots).\]

   \item This is a good algorithm for computing square roots, since the recursion formula is simple and the 
     convergence is extremely rapid.  For example, if \(\alpha = 3\) and \(x_1 = 2\), show that 
     \(\epsilon/\beta < \frac{1}{10}\) and that therefore
     \[\epsilon_5 < 4 \cdot 10^{-16}, \quad \epsilon_6 < 4 \cdot 10^{-32}.\]
 \end{enumerate}
 \newpage

 \item Fix \(\alpha > 1\). Take \(x_1 > \sqrt{\alpha}\), and define
 \[x_{n+1} = \frac{\alpha + x_n}{1 + x_n} = x_n + \frac{\alpha - x_n^2}{1 + x_n}\]

 \begin{enumerate}[label=(\alph*)]
   \item Prove that \(x_1 > x_3 > x_5 > \dots\).
   \item Prove that \(x_2 < x_4 < x_6 < \dots\).
   \item Prove that \(\lim x_n = \sqrt{\alpha}\).
   \item Compare the rapidity of convergence of this process with the one described in the previous problem.
 \end{enumerate}
 \newpage

 \item As is usual, for each real number \(t\) we write \([t]\) for the greatest integer \(\leq t\), e.g. 
 \([\pi] = 3\) and \([-\pi] = -4\).  Observe that, for all real numbers \(s\) and \(t\), 
 \(((s - [s]) = (t - [t])) \Leftrightarrow (s - t \text{ is an integer})\).  In particular, \(t - [t]\) is 
 the unique real number \(x \in [0,1)\) such that \(t-x\) is an integer. \\

 Fix an irrational number \(\alpha \in \mathbb{R}\), and define a sequence \(\{x_n\}_{n \geq 1}\) by 
 setting, for each integer \(n \geq 1\), 
 \[x_n = n\alpha - [n\alpha].\]

 \begin{enumerate}[label=(\roman*)]
   \item Check that \(\{x_n\}_{n \geq 1}\) is a sequence of distinct (that is, \(m \neq n \Rightarrow 
     x_m \neq x_n\)) irrational numbers in the interval \((0,1)\). \\

     Let \(N \geq 3\) be an integer, and partition the interval \([0,1]\) into \(N\) consecutive subintervals
     \(I_1^N \cup I_2^N \cup \dots \cup I_N^N\) of length \(1/N\) by setting
     \[I_j^N = \left[\left(\frac{j-1}{N}\right), \left(\frac{j}{N}\right) \right] \quad \text{for } 1 \leq j \leq N.\]

     Of course, the union \(I_1^N \cup I_2^N \cup \dots \cup I_N^N\) is equal to \([0,1]\), and for distinct 
     indicies \(j < k\), the intersection \(I_j^N \cap I_k^N\) is empty unless \(k = j+1\), in which case 
     theis intersection consists of the single rational number \(j/N\).  Thus, by (i), each term \(x_n\) of 
     our sequence lies in a unique subinterval \(I_j^N\).  Indeed, \(x_n\) lies in the interior 
     \(\left(\frac{j-1}{N}, \frac{j}{N}\right)\) of that interval.

     
   \item Prove that at least one of the first \(N-1\) terms \(x_1, x_2, \dots, x_{N-1}\) of our sequence 
     lies either in the first subinterval \(I_1^N = \left[0,\frac{1}{N}\right]\) or in the last subinterval 
     \(I_1^N = \left[1-\frac{1}{N},1\right]\). (If not, why must some subinterval \(I_l^N\) with
     \(2 \leq l \leq N-1\) contain two distinct terms \(x_j\) and \(x_k\) with \(1 \leq j < k < N-1\)? 
     If this is true, what then can you say about \(x_{(k-j)}\)?)

   \item Suppose that \(p \geq 1\) is an index such that \(x_p \in I_1^N\), and consider the subsequence 
     \(\{x_{l_p}\}_{l \geq 1}\) of our original sequence.  Prove the following statements about it's 
     behavior:  Suppose that the term \(x_{l_p}\) lies in the \(j\)th subinterval \(I_j^N\).  If \(j < N\),
     then the nexy term  \(x_{(l+1)_p}\) lies either again in \(I_j^N\) or the next subinterval \(I_{j+1}^N\). 
     If \(j = N\), then \(x_{(l+1)_p}\) lies either again in \(I_N^N\) or in the first subinterval \(I_1^N\). 
     Moreover, at most \(1 + [1/(Nx_p)]\) consecutive terms of the subsequence \(\{x_{l_p}\}_{l \geq 1}\), 
     can lie in any one subinterval \(I_k^N\).  (First observe and explain why \(
     x_{(l+1)_p} = x_{l_p} + x_p - [x_{l_p} + x_p]\).)

   \item Suppose that \(p \geq 1\) is an index such that \(x_p \in I_N^N\), and again consider the 
     subsequence \(\{x_{l_p}\}_{l \geq 1}\).  Prove the following statements: Suppose that the term 
     \(x_{l_p}\) lies in the \(j\)th subinterval \(I_j^N\).  If \(j > 1\), then the next term \(x_{(l+1)_p}\) 
     lies either again in \(I_j^N\) or in the previous subinterval \(I_{j-1}^N\).  If \(j = 1\), then 
     \(x_{(l+1)_p}\) lies either again in \(I_1^N\) or in the last subinterval \(I_N^N\).  At most 
     \(1 + [1/(N(1 - x_p))]\) consecutive terms of the subsequence \(\{x_{l_p}\}_{l \geq 1}\) can lie in 
     one subinterval \(I_k^N\).  (First observe and explain why \(x_{(l+1)_p} = x_{l_p} - (1 - x_p) - 
     [x_{l_p} - (1 - x_p)]\).)

   \item Let \(a < b\) be distinct real numbers in \([0,1]\). Prove that there are infinitely many indicies 
     \(n\) for which \[a < x_n < b.\]

    
   \item Prove that there are infinitely many positive integers \(n\) such that the first seven digits (
     counting from the left) of the usual (base 10) expression of \(2^n\) are \(7777777\dots\) .
     (Why is \(\log_{10}(2)\) an irrational number? Why does \(n\log_{10}(2) - [n\log_{10}(2)]\) specify 
     the digits of \(2^n\)?)
     \newpage

   \item Also deduce from (ii) the following statement: Let \(\alpha\) be an irrational number.  Then there 
     exists a sequence \(\big\{\frac{p_l}{q_l}\big\}_{l \geq 1}\) of rational numbers (\(p_l \in \mathbb{Z}, 
     q_l \in \mathbb{N^+}\)) with strictly increasing denominators 
     \[1 \leq q_1 < q_2 < \hdots < q_l < q_{l+1} < \hdots\]
     such that 
     \[0 < \left|\alpha - \frac{p_l}{q_l}\right| < \frac{1}{q_l(q_l + 1)} \quad \text{for all } l \geq 1.\]
  \end{enumerate}
  \newpage

  \item As is usual, for all strictly positive real exponents \(\lambda > 0\), we set \(0^\lambda = 0\).

    Let \(\alpha\) be a real number with \(0 < \alpha < 1\).  Let \(\{x_n\}_{n \geq 0}\) be a sequence of 
    real numbers, all \(\geq 0\), which converges to a real number \(L \geq 0\).  Prove that 
    \[\lim_{n \to \infty} (x_n)^\alpha = L^\alpha.\]
    (You may want to treat separately the cases \(L = 0\) and \(L > 0\).)


 \end{enumerate}
\end{document}
