\documentclass[11pt]{article}

\usepackage{amsmath}
\usepackage{amsfonts} 
\usepackage{amsthm}
\usepackage{enumitem} 
\usepackage{mathtools}
\usepackage[top=2cm,bottom=2cm,left=2.5cm,right=2.5cm,marginparwidth=1.75cm]{geometry}
\setlength{\parindent}{0cm}

\newtheorem{theorem}{Theorem}

\title{\vspace{-1.0cm}MATH 5615H Homework 5}
\author{Fletcher Gornick}
\date{October 5, 2022}

\begin{document}
 \maketitle
 \begin{enumerate}[leftmargin=0pt, label=\arabic*)]
    \item Prove that the convergence of \(\sum a_n\) implies the convergence of 
      \[\sum \frac{\sqrt{a_n}}{n},\]
      if \(a_n \geq 0\).
      \newpage

    \item If \(\sum a_n\) converges, and if \(\{b_n\}\) is monotonic and bounded, prove that 
      \(\sum a_n b_n\) converges.
      \newpage

    \item Let \(\{b_n\}_{n \geq 1}\) be a monotonic decreasing sequence of nonnegative real numbers.
      Prove \[\text{If} \;\; \sum_{n=1}^{\infty}b_n \;\; \text{converges, then} 
      \lim_{n \to \infty} nb_n = 0.\]
      (The conclusion \(\displaystyle\lim_{n \to \infty} \left(\frac{n}{2}\right)b_n\)) is equivalent,
      but more suggestive.  The crux of the problem is to show that the limit exists at all).  Give an 
      example to show that the converse is false.
      \newpage

    \item Let \(\beta\) be a real number, and consider the series
      \[
        \sum_{n=1}^{\infty}\frac{a_n}{n} \;\; \text{where} \;\; a_n=
        \begin{cases}
          3, &\text{ if } n = 1 \mod 3 \\
          4, &\text{ if } n = 2 \mod 3 \\
          \beta, &\text{ if } n = 0 \mod 3 \\
          
        \end{cases}
      \]
      Let \(B\) be the set of all real numbers \(\beta\) for which this series is convergent.  Find, with 
      proof, all elements of \(B\), or else prove that \(B\) is the empty set.
      \newpage

    \item Let \(\{z_n\}_{n \geq 0}\) be a complex sequence convergent to a complex number \(L\).  As usual, 
      if \(N\) and \(l\) are integers with \(0 \leq l \leq N\), we write 
      \(\begin{pmatrix} N \\ l \end{pmatrix}\) for the binomial coefficient 
      \(\begin{pmatrix} N \\ l \end{pmatrix} = \left( \displaystyle\frac{N!}{l!(N-l)!} \right)\), where 
      \(0!=1\).  Prove that
      \[\lim_{N \to \infty} 2^{-N} \left( \sum_{l=0}^{N} \begin{pmatrix} N \\ l \end{pmatrix} z_l \right) = L\]
 \end{enumerate}
\end{document}
