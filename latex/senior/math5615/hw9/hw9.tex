\documentclass[11pt]{article}

\usepackage{amsmath}
\usepackage{amsfonts} 
\usepackage{amsthm}
\usepackage{enumitem} 
\usepackage{mathtools}
\usepackage[top=2cm,bottom=2cm,left=1.5cm,right=2cm,marginparwidth=1.75cm]{geometry}
\setlength{\parindent}{0cm}

\pagenumbering{gobble}
\newtheorem{theorem}{Theorem}

\begin{document}
\begin{itemize}
  \item [\textbf{4.6}]
    Let \((E, \rho)\) and \((F, \mu)\) be metric spaces, and make \(E \times F\) a metric space by defining 
    \(\sigma \colon E \times F \to \mathbb{R}\), \(\sigma\left((x_1,y_1), (x_2,y_2)\right) = \rho(x_1,x_2) + \mu(y_1,y_2)\).  
    Suppose that \((E, \mu)\) is compact.  Prove that a function \(f \colon E \to F\) is continuous if and only if the graph 
    \(\{\left(x,f(x)\right) \mid x \in E\}\) is a compact subest of \(E \times F\).
  \newpage


  \item [\textbf{4.17}]
    Let \(f\) be a real function defined on \((a,b)\).  Prove that the set of points at which \(f\) has a simple discontinuity
    is at most countable.  \textit{Hint:} Let \(E\) be the set on which \(f(x-) < f(x+)\).  With each point \(x\) of \(E\), 
    associate a triple \((p,q,r)\) of rational numbers such that
    \begin{enumerate}[label=(\alph*)]
      \item \(f(x-) < p < f(x+)\),
      \item \(a < q < t < x\) implies \(f(t) < p\),
      \item \(x < t < r < b\) implies \(f(t) > p\).
    \end{enumerate}
    The set of all such triples is countable.  Show that each triple is associated with at most one point of \(E\).  Deal 
    similarly with the other possible types of simple discontinuities.
  \newpage


  \item [\textbf{4.21}]
    Suppoes \(K\) and \(F\) are disjoint sets in a metric space \(X\), \(K\) is compact, \(F\) is closed.  Prove that there 
    exists \(\delta > 0\) such that \(d(p,q) > \delta\) if \(p \in K\), \(q \in F\).  \textit{Hint:} \(\rho_F\) is a continuous 
    positive function on \(K\).

    Show that the conclusion may fail for two disjoint closed sets if neither is compact.
  \newpage


  \item [\textbf{4.22}]
    Let \(A\) and \(B\) be disjoint nonempty closed sets in a metric space \(X\), and define
    \[f(p) = \frac{\rho_A(p)}{\rho_A(p) + \rho_B(p)} \;\; (p \in X).\]
    Show that \(f\) is a continuous function on \(X\) whose range lies in \([0,1]\), that \(f(p) = 0\) precisely on \(A\) 
    and \(f(p) = 1\) precisely on \(B\).  This establishes a converse of Exercise 3: Every closed set \(A \subset X\) is 
    \(Z(f)\) for some continuous real \(f\) on \(X\).  Setting
    \[V = f^{-1}\left(\left[0, \tfrac12\right)\right), \quad W = f^{-1}\left(\left(\tfrac12,1\right]\right),\]
    Show that \(V\) and \(W\) are open and disjoint, and that \(A \subset V, B \subset W\).  (Thus pairs of disjoint closed 
    sets in a metric space can be covered by pairs of disjoint open sets.  This property of metric spaces is called 
    \textit{normality}.)
  \newpage


  \item [\textbf{4.25}]
    If \(A \subset \mathbb{R}^k\) and \(B \subset \mathbb{R}^k\), define \(A + B\) to be the set of all sums \(x + y\) with 
    \(x \in A\), \(y \in B\).
    \begin{enumerate}[label=(\alph*)]
      \item If \(K\) is compact and \(C\) is closed in \(\mathbb{R}^k\), prove that \(K + C\) is closed. \\
        \textit{Hint:} Take \(\textbf{z} \not\in K + C\), put \(F = \textbf{z} - C\), the set of all \(\textbf{z} - y\) 
        with \(y \in C\).  Then \(K\) and \(F\) are disjoint.  Choose \(\delta\) as in Exercise 21.  Show that the open ball 
        with center \(\textbf{z}\) and radius \(\delta\) does not intersect \(K + C\).

      \item Let \(\alpha\) be an irrational real number.  Let \(C_1\) be the set of all integers, let \(C_2\) be the set of all
        \(n\alpha\) with \(n \in C_1\).  Show that \(C_1\) and \(C_2\) are closed subsets of \(\mathbb{R}\) whose sum 
        \(C_1 + C_2\) is \textit{not} closed, by showing that \(C_1 + C_2\) is a countable dense subset of \(\mathbb{R}\).
    \end{enumerate}

  \newpage

\end{itemize}
\end{document}

