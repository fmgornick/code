\documentclass[11pt]{article}
\usepackage[utf8]{inputenc}
\usepackage{amsmath,amsthm}
\usepackage{fullpage}
\usepackage{amsfonts}
\usepackage{amssymb,multicol}
\usepackage{graphicx}
\usepackage{hyperref, enumitem}
\usepackage{outlines}
\usepackage{xcolor}

\newcommand{\Z}{\mathbb{Z}}
\newcommand{\N}{\mathbb{N}}
\newcommand{\n}{\vspace{0.5cm}}
\newcommand{\mmod}{\;\%\;}

\newtheorem*{theorem}{Theorem}

\newlist{checklist}{itemize}{2}
\setlist[checklist]{label=$\square$}

\begin{document}
  \begin{center}
    {\Large Homework 4} \n

    Fletcher Gornick
  \end{center}

  \begin{enumerate}
    \item[0.] If you would like any of these problems to be graded for proficiency with the core skills, list the skill and the corresponding problem.
      \begin{outline}
        \1 5. Using the Euclidean algorithm to find the inverse of (the class of) one number modulo another.
          \2 problem 2(a)
        \1 6. Showing that an equation cannot have integer solutions because it does not have solutions modulo some integer.
          \2 problem 2(b)
        \1 7. Using Euclid’s Lemma in a proof.
          \2 problem 3(a)
        \1 8. Using the Chinese Remainder Theorem either in a proof or to solve a modular system of equations.
          \2 problem 2(a)
      \end{outline} \n

    \item Find all solutions to the linear congruence equation \[15x \equiv 6 \pmod 9\] or explain why no solution exists.
      \begin{align*}
        15x      &= 9x + 6x \\
                 &\equiv 6x \pmod 9 \\
        15x      &\equiv 6 \pmod 9 \\
        \iff 6x  &\equiv 6 \pmod 9 \\
        \iff 2x  &\equiv 2 \pmod 3
      \end{align*}
      \(2x \equiv 2 \pmod 3\) has exactly one solution because 2 and 3 are coprime.  The solution is pretty obvious, just 1.  So we can now plug this answer into our original congruence equation to get \(x \equiv 1 \pmod 3\), so \(x \equiv 1,4,7 \pmod 9\).

    \item Solve each of the following simultaneous systems of congruences or explain why no solution exists.

          The chinese remainder theorem states that for a system of congruences \(x \equiv a_i \pmod{m_i}\) with \(m_1, \hdots, m_n\) pairwise coprime, we can find a unique solution modulo \(N\)
          \[x \equiv \left( \sum_{i=1}^{n} a_iy_iz_i \right) \pmod{N}, \text{ where } N = \prod_{i=1}^{n} m_i, \;\; y_i = \frac{N}{m_i}, \text{ and } z_i \equiv y_i^{-1} \pmod{m_i}.\]
      \begin{enumerate}
        \item \(\begin{cases}
            x \equiv 5 \pmod{13} \\
            x \equiv 2 \pmod{7} \\
            x \equiv 4 \pmod{11} \\
          \end{cases}\) \n\\
          So for this question, it's clear that 13, 7, and 11 are all coprime, so the theorem applies here.  \(N = 13 \cdot 7 \cdot 11 = 1001\), \(a_1 = 5, a_2 = 2, a_3 = 4\), \(y_1 = 1001/13 = 77, y_2 = 1001/7 = 143, y_3 = 1001/11 = 91\)

          \begin{minipage}{0.28\textwidth}
            \begin{align*}
              z_1      &\equiv 77^{-1} \pmod{13} \\
                       &\equiv (-1)^{-1} \pmod{13} \\
                       &\equiv -1 \pmod{13} \\
              z_1      &\equiv 12 \pmod{13} \\
            \end{align*}
          \end{minipage}
          \begin{minipage}{0.28\textwidth}
            \begin{align*}
              z_2      &\equiv 143^{-1} \pmod{7} \\
                       &\equiv 3^{-1} \pmod{7} \\
              1        &= 7 - 2 \cdot 3 \\
              z_2      &\equiv -2 \pmod{7} \\
                       &\equiv 5 \pmod{7} \\
            \end{align*}
          \end{minipage}
          \begin{minipage}{0.28\textwidth}
            \begin{align*}
              z_3      &\equiv 91^{-1} \pmod{11} \\
                       &\equiv 3^{-1} \pmod{11} \\
              1        &= 4 \cdot 3 - 11 \\
              z_3      &\equiv 4 \pmod{11} \\
            \end{align*}
          \end{minipage}

          With this, we can now solve for \(x\).
          \begin{align*}
            x &\equiv a_1y_1z_1 + a_2y_2z_2 + a_3y_3z_3 \pmod{N} \\ 
              &\equiv 5 \cdot 77 \cdot 12 + 2 \cdot 143 \cdot 5 + 4 \cdot 91 \cdot 4 \pmod{1001} \\
              &\equiv 7506 \pmod{1001} \\
              &\equiv 499 \pmod{1001}. \\
          \end{align*}

        \item \(\begin{cases}
            x \equiv 3 \pmod{9} \\
            x \equiv 2 \pmod{6} \\
            x \equiv 1 \pmod{5} \\
          \end{cases}\) \n

          This system of congruences actually isn't solvable.  6 and 9 aren't coprime, so there's not necessarily a solution.  We can note that if \(x \equiv 3 \pmod 9\), then \(x \equiv 0 \pmod 3\).  Also, if \(x \equiv 2 \pmod 6\), then \(x \equiv 2 \pmod 3\).  But since \(0 \not\equiv 2 \pmod 3\), there's no \(x\) that can satisfy the above system of congruences.
      \end{enumerate}

    \item This exercise will help us work towards understanding square roots.
      \begin{enumerate}
        \item Let \(p\) be a prime number and let \(b\) be an integer such that \(x^2 \equiv b \pmod p\) has a solution.  Show that, if \(a_1\) and \(a_2\) are two such solutions, then \(a_1 \equiv a_2 \pmod p\) or \(a_1 \equiv -a_2 \pmod p\)
          \begin{proof}
            If \(a_1\) and \(a_2\) are both solutions to \(x^2 \equiv b \pmod p\), then:
            \begin{align*}
              a_1^2 &\equiv a_2^2 \pmod p \\
              \iff p &\mid (a_1^2 - a_2^2) & \text{(by definition of congruence)}\\
              \iff p &\mid (a_1 + a_2)(a_1 - a_2) & \text{(difference of squares)}\\
            \end{align*}
            And by Euclid's Lemma, if \(p\) divides \((a_1 + a_2)(a_1 - a_2)\), then \(p\) must divide at least one of \((a_1 + a_2)\) or \((a_1 - a_2)\). \n

            Case 1: \(p \mid (a_1 + a_2)\).  From the definition of congruence, we see \(a_1 \equiv -a_2 \pmod p\). \\
            Case 2: \(p \mid (a_1 - a_2)\).  From the definition of congruence, we see \(a_1 \equiv a_2 \pmod p\). \n

            Therefore, we can conclude that if \(a_1\) and \(a_2\) are both solutions to \(x^2 \equiv b \pmod p\), then \(a_1 \equiv \pm a_2 \pmod p\).
         \end{proof}
        \item Provide a counterexample to show that the above statement is not necessarily true if \(p\) is not prime.  In other words, provide integers \(m, b, a_1\), and \(a_2\) such that \(a_1^2 \equiv b \pmod m\), \(a_2^2 \equiv b \pmod m\), but \(a_1 \not\equiv \pm a_2 \pmod m\). Make sure you explain why your counterexample is, in fact, a counterexample. \n\\
          Take \(m=8, b=1, a_1=3, a_2=7\),

          \begin{minipage}{0.28\textwidth}
            \begin{align*}
              3^2 &= 9 \\
                  &\equiv 1 \pmod8 \\
            \end{align*}
          \end{minipage}
          \begin{minipage}{0.28\textwidth}
            \begin{align*}
              7^2 &= 49 \\
                  &\equiv 1 \pmod8 \\
            \end{align*}
          \end{minipage}
          \begin{minipage}{0.28\textwidth}
            \begin{align*}
              3 &\not\equiv \pm 7 \pmod8. \\
            \end{align*}
          \end{minipage}
      \end{enumerate}

    \item Prove that \(x^2 + y^2 = 95\) has no integer solutions.
      \begin{proof}
        If there did exist integers \(x\) and \(y\) such that \(x^2 + y^2 = 95\), then \(x^2 + y^2 \equiv 3 \pmod4\).
        \begin{align*}
          0^2 &\equiv 0 \pmod4 \\
          1^2 &\equiv 1 \pmod4 \\
          2^2 &\equiv 0 \pmod4 \\
          3^2 &\equiv 1 \pmod4 \\
        \end{align*}
        Since, any integer \(z\) must be in one of the four equivalence classes \(\overline{0},\overline{1},\overline{2},\overline{3} \pmod4\), we can conclude that \(z^2 \equiv 0,1 \pmod4\), meaning that for any \(x,y \in \Z\), 

        \[x^2 + y^2 \equiv \begin{cases}
          0, &\text{ if } x \equiv y \equiv 0 \pmod4, \\
          2, &\text{ if } x \equiv y \equiv 1 \pmod4, \\
          1, &\text{ otherwise.} \\
        \end{cases}\]

        Therefore, since \(x^2 + y^2 \not\equiv 3 \pmod4\) for any \(x,y \in \Z\), we can conclude that \(x^2 + y^2 = 95\) has no integer solutions (via contraposition).
      \end{proof}
      
  \end{enumerate}
\end{document}
