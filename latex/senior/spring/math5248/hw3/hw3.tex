\documentclass[11pt]{article}
\usepackage[utf8]{inputenc}
\usepackage{amsmath,amsthm}
\usepackage{fullpage}
\usepackage{amsfonts}
\usepackage{amssymb,multicol}
\usepackage{graphicx}
\usepackage{hyperref, enumitem}
\usepackage{outlines}
\usepackage{xcolor}

\newcommand{\Z}{\mathbb{Z}}
\newcommand{\N}{\mathbb{N}}
\newcommand{\n}{\vspace{0.5cm}}
\newcommand{\mmod}{\;\%\;}

\newtheorem*{theorem}{Theorem}

\newlist{checklist}{itemize}{2}
\setlist[checklist]{label=$\square$}

\begin{document}
  \begin{center}
    {\Large Homework 3} \n

    Fletcher Gornick
  \end{center}

  \begin{enumerate}
    \item[0.] If you would like any of these problems to be graded for proficiency with the core skills, list the skill and the corresponding problem.
      \begin{outline}
        \1 1. Using a known plaintext attack to determine the key of a cipher of any kind.
          \2 problem 5
        \1 2. Apply the definition of divisibility in a proof.
          \2 problem 6
        \1 3. Apply the definition of a unit modulo m in a proof.
          \2 problem 3
        \1 4. Using Bezout’s identity in a proof.
          \2 problem 6
        \1 5. Using the Euclidean algorithm to find the inverse of (the class of) one number modulo another.
          \2 problem 2
      \end{outline} \n

    \item Use the Euclidean Algorithm to find \(\gcd(65330,5420)\).
      \begin{align*}
        65330 &= 12 \cdot 5420 + 290 \\
        5420  &= 18 \cdot 290 + 200 \\
        290   &= 200 + 90 \\
        200   &= 2 \cdot 90 + 20 \\
        90    &= 4 \cdot 20 + 10 \\
        20    &= 2 \cdot 10
      \end{align*}
      \(\gcd(65330,5420) = 10\).

    \item Use the Euclidean Algorithm to find a multiplicative inverse of 206 modulo 5427.
      \begin{align*}
        71 &= 5427 - 26 \cdot 206 \\
        64 &= 206 - 2 \cdot 71 \\
        7  &= 71 - 64 \\
        1  &= 64 - 9 \cdot 7 \\
           &= 64 - 9(71 - 64) = 10 \cdot 64 - 9 \cdot 71 \\
           &= 10(206 - 2 \cdot 71) - 9 \cdot 71 = 10 \cdot 206 - 29 \cdot 71 \\
           &= 10 \cdot 206 - 29(5427 - 26 \cdot 206) \\
           &= 764 \cdot 206 - 29 \cdot 5427
      \end{align*}
      We now have that \(764 \cdot 206 - 29 \cdot 5427 = 1\), meaning \(764 \cdot 206 \equiv 1 \pmod{5427}\), so by definition, 764 is the inverse of 206 modulo 5427.

    \item Prove that, if \(a_1\) and \(a_2\) are units modulo \(m\), then \(a_1 a_2\) is also a unit modulo \(m\).
      \begin{proof}
        If \(a_1\) and \(a_2\) are units modulo \(m\), then they both have multiplicative inverses modulo \(m\), so there exists \(b_1,b_2 \in \Z\) such that \(a_1b_1 \equiv 1 \pmod m\) and \(a_2b_2 \equiv 1 \pmod m\).  This means \(a_1a_2 \cdot b_1b_2 = a_1b_1 \cdot a_2b_2 \equiv 1 \pmod m\), and since \(b_1b_2 \in \Z\), \(a_1a_2\) has a multiplicative inverse modulo \(m\), and thus is a unit.
      \end{proof}

    \item Consider an affine cipher with key \((5,4)\).
      \begin{enumerate}
        \item Encrypt the word ``cryptology'' using that cipher. \n\\
          Affine cipher is defined like so: \(E_{(a,b)}(x) = ax + b \mmod{26}\).  So we first transfer our text to numbers using the following table, apply the cipher function to each element, then retrieve our encrypted text back through the table. \n\\
        \scalebox{0.73}{
            \begin{tabular}{ |c|c|c|c|c|c|c|c|c|c|c|c|c|c|c|c|c|c|c|c|c|c|c|c|c|c| }
              \hline
              A & B & C & D & E & F & G & H & I & J & K & L & M & N & O & P & Q & R & S & T & U & V & W & X & Y & Z \\
              \hline
              0 & 1 & 2 & 3 & 4 & 5 & 6 & 7 & 8 & 9 & 10 & 11 & 12 & 13 & 14 & 15 & 16 & 17 & 18 & 19 & 20 & 21 & 22 & 23 & 24 & 25 \\
              \hline
            \end{tabular}
          }

          \begin{align*}
            \text{``cryptology''} &\xrightarrow{\text{table}} (2,17,24,15,19,14,11,14,6,24) \\
                                  &\xrightarrow{E_{(5,4)}} (14,11,20,1,21,22,7,22,8,20) \\
                                  &\xrightarrow{\text{table}} \text{``OLUBVWHWIU''}
          \end{align*}

        \item You recieve the ciphertext ``DAROVSWR''.  What is the decrypted plaintext? \n\\
          Since \(E_{(5,4)}(x) = 5x + 4 \mmod{26}\), we know \(5x \equiv E_{(5,4)}(x) - 4 \pmod{26}\), now we can isolate \(x\) to get the following equation: \(x \equiv 5^{-1} \cdot 5x \equiv 5^{-1} \left( E_{(5,4)}(x) - 4 \right) \pmod{26}\).  

          So we must first use the extended Euclidean algoritm to find the multiplicative inverse of 5 modulo 26, but it's basically just one step: \(1 = 26 - 5 \cdot 5\), so \(21 \equiv -5 \equiv 5^{-1} \pmod{26}\).

          With this, we can now define a new equation for decoding our affine cipher:
          \[E^{-1}_{(5,4)}(x) = 21 \cdot (x - 4) \mmod{26},\]
          So we can use the same steps outlined in (a), only this time we plug in our decoding equation \(E^{-1}_{(5,4)}\).
          \begin{align*}
            \text{``DAROVSWR''} &\xrightarrow{\text{table}} (3,0,17,14,21,18,22,17) \\
                                &\xrightarrow{E^{-1}_{(5,4)}} (5,20,13,2,19,8,14,13) \\
                                  &\xrightarrow{\text{table}} \text{``FUNCTION''}
          \end{align*}
      \end{enumerate}
      \newpage

    \item Two enemies of yours are passing messages using an affine cipher.  You know that they always write formal notes starting with the greating ``Hello'' in the plaintext.  You intercept a ciphertext that starts with ``fkhhc.'' What key did they use?
      \begin{align*}
        \text{``HELLO''} &\xrightarrow{\text{table}} (7,4,11,11,14) \\
        \text{``FKHHC''} &\xrightarrow{\text{table}} (5,10,7,7,2) \\
      \end{align*}
      From this, we can pick out the first two digits in each to get the following system of equations (assuming their using an affine cipher with some arbitrary key \((a,b)\)):
      \begin{align*}
        7 \cdot a + b &\equiv 5 \pmod{26} \\
        4 \cdot a + b &\equiv 10 \pmod{26} \\
      \end{align*}
      So \(3a \equiv -5 \equiv 21 \pmod{26}\), from this we clearly see that \(a=7\), and since 3 is relatively prime to 26, we know this solution must be unique (from a theorem in our worksheet).

      Now we have \(49 + b \equiv 5 \pmod{26}\), so setting \(b=8\), we get \((49 + 8 ) \mmod 26 = 5\).  

      We can now conclude that The key is \((7,8)\).


    \item Let \(a\), \(b\) and \(d\) be integers. Prove that, if \(d|a\) and \(d|b\), then \(d| \gcd(a,b)\). (Hint: Use Bezout’s identity.)
      \begin{proof}
        By Bezout's identity, \(\gcd(a,b) = ax_1 + by_1\) for some \(x_1,y_1 \in \Z\).

        Suppose \(d|a\) and \(d|b\), so \(a = dx_2\) and \(b = dy_2\) for some \(x_2,y_2 \in \Z\), but this means
        \[\gcd(a,b) = ax_1 + by_1 = dx_1x_2 + dy_1y_2 = d(x_1x_2 + y_1y_2),\]
        And since \(x_1x_2 + y_1y_2 \in \Z\), it follows that \(d\) must divide \(\gcd(a,b)\).
      \end{proof}
      
  \end{enumerate}
\end{document}
