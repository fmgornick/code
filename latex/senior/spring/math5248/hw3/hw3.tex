\documentclass[11pt]{article}
\usepackage[utf8]{inputenc}
\usepackage{amsmath,amsthm}
\usepackage{fullpage}
\usepackage{amsfonts}
\usepackage{amssymb,multicol}
\usepackage{graphicx}
\usepackage{hyperref, enumitem}
\usepackage{outlines}
\usepackage{xcolor}

\newcommand{\Z}{\mathbb{Z}}
\newcommand{\N}{\mathbb{N}}
\newcommand{\n}{\vspace{0.5cm}}
\newcommand{\mmod}{\;\%\;}

\newtheorem*{theorem}{Theorem}

\newlist{checklist}{itemize}{2}
\setlist[checklist]{label=$\square$}

\begin{document}
  \begin{center}
    {\Large Homework 2} \n

    Fletcher Gornick
  \end{center}

  \begin{enumerate}
    \item[0.] If you would like any of these problems to be graded for proficiency with the core skills, list the skill and the corresponding problem.
      \begin{outline}
        \1 2. Apply the definition of divisibility in a proof.
          \2 problem 5
        \1 3. Apply the definition of a unit modulo m in a proof.
          \2 problem 3
      \end{outline} \n

    \item Use the Euclidean Algorithm to find \(\gcd(65330,5420)\).
      \begin{align*}
        65330 &= 12 \cdot 5420 + 290 \\
        5420  &= 18 \cdot 290 + 200 \\
        290   &= 200 + 90 \\
        200   &= 2 \cdot 90 + 20 \\
        90    &= 4 \cdot 20 + 10 \\
        20    &= 2 \cdot 10
      \end{align*}
      \(\gcd(65330,5420) = 10\).

    \item Use the Euclidean Algorithm to find a multiplicative inverse of 206 modulo 5427.
      \begin{align*}
        71 &= 5427 - 26 \cdot 206 \\
        64 &= 206 - 2 \cdot 71 \\
        7  &= 71 - 64 \\
        1  &= 64 - 9 \cdot 7 \\
           &= 64 - 9(71 - 64) = 10 \cdot 64 - 9 \cdot 71 \\
           &= 10(206 - 2 \cdot 71) - 9 \cdot 71 = 10 \cdot 206 - 29 \cdot 71 \\
           &= 10 \cdot 206 - 29(5427 - 26 \cdot 206) \\
           &= 764 \cdot 206 - 29 \cdot 5427
      \end{align*}
      We now have that \(764 \cdot 206 - 29 \cdot 5427 = 1\), meaning \(764 \cdot 206 \equiv 1 \pmod{5427}\), so by definition, 764 is the inverse of 206 modulo 5427.

    \item Prove that, if \(a_1\) and \(a_2\) are units modulo \(m\), then \(a_1 a_2\) is also a unit modulo \(m\).

    \item Consider an affine cipher with key \((5,4)\).
      \begin{enumerate}
        \item Encrypt the word ``cryptology'' using that cipher.
        \item You recieve the ciphertext ``DAROVSWR''.  What is the decrypted plaintext?
      \end{enumerate}

    \item Two enemies of yours are passing messages using an affine cipher.  You know that they always write formal notes starting with the greating ``Hello'' in the plaintext.  You intercept a ciphertext that starts with ``fkhhc.'' What key did they use?

    \item Let \(a\), \(b\) and \(d\) be integers. Prove that, if \(d|a\) and \(d|b\), then \(d| \gcd(a,b)\). (Hint: Use Bezout’s identity.)
  \end{enumerate}
\end{document}
