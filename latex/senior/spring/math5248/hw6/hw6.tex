\documentclass[12pt]{article}
\usepackage[utf8]{inputenc}
\usepackage{amsmath,amsthm}
\usepackage{fullpage}
\usepackage{amsfonts}
\usepackage{outlines}
\usepackage{amssymb,multicol}
\usepackage[colorlinks=true,urlcolor=blue]{hyperref}
\usepackage{enumitem}
\usepackage{xcolor,systeme}

\newcommand{\Z}{\mathbb{Z}}
\newcommand{\ds}{\displaystyle}
\newtheorem*{theorem}{Theorem}
\newcommand{\n}{\vspace{0.5cm}}

\newlist{checklist}{itemize}{2}
\setlist[checklist]{label=\(\square\)}

\begin{document}
  \begin{center}
    {\Large Homework 6} \n\\
    Fletcher Gornick
  \end{center}

\begin{enumerate}
  \item[0.] If you would like any of these problems to be graded for proficiency with the core skills, list the skill and the corresponding problem. 
      \begin{outline}
        \1 8. Using the Chinese Remainder Theorem either in a proof or to solve a modular system of equations.
          \2 problem 2(a) - I split the matrix modulo 21 into matrices modulo 3 and 7.
      \end{outline} \n

  \item Martha is sending a message talking about how great her sister is.  Martha has encrypted her message using a Hill cipher with a \(2\times 2\) key.  You know that she has encrypted the word ``best'' to the ciphertext ``JCTF''.  What are all possible keys to Martha's Hill cipher? (There may be one or more possible keys.)

    ``BEST'' \(\to \begin{pmatrix} 1 & 18 \\ 4 & 19 \end{pmatrix}\), and ``JCTF'' \(\to \begin{pmatrix} 9 & 19 \\ 2 & 5 \end{pmatrix}\)
    \[\begin{pmatrix} a & b \\ c & d \end{pmatrix}\begin{pmatrix} 1 & 18 \\ 4 & 19 \end{pmatrix} \equiv  \begin{pmatrix} 9 & 19 \\ 2 & 5 \end{pmatrix} \pmod{26}\]
    \[\det \begin{pmatrix} 1 & 18 \\ 4 & 19 \end{pmatrix} = -53 \equiv 25 \pmod{26}\]
    Since \(\gcd(25,26) = 0\), we have that this matrix is invertible modulo 26, meaning our hill cipher will only have one solution.  Clearly \(25^{-1} \equiv 25 \equiv -1 \pmod{26}\), as \(25 \equiv -1 \pmod{26}\), and \((-1)^2 = 1\), so we just multiply -1 by our adjugate matrix to get it's inverse.
    \[\begin{pmatrix} 1 & 18 \\ 4 & 19 \end{pmatrix}^{-1} \equiv -1 \cdot \begin{pmatrix} 19 & -18 \\ -4 & 1 \end{pmatrix} \equiv \begin{pmatrix} 7 & 18 \\ 4 & 25 \end{pmatrix} \pmod{26}\]
    \[\begin{pmatrix} a & b \\ c & d \end{pmatrix} \equiv \begin{pmatrix} 9 & 19 \\ 2 & 5 \end{pmatrix}\begin{pmatrix} 7 & 18 \\ 4 & 25 \end{pmatrix} \equiv \begin{pmatrix} 9 & 13 \\ 8 & 5 \end{pmatrix} \pmod{26}.\]

    This tells us that the key to our hill cipher is ``JINF''. \n

  \item Find all possible ordered pairs of integers \((x,y)\) with \(0 \le x,y<15\) satisfying the following system of congruences:
    \[\systeme*{9x+12y\equiv 3\pmod{21}, 11x+5y\equiv 12\pmod{21}}\] 
    or explain why no such integers exist.

    \[\begin{pmatrix} 9 & 12 \\ 11 & 5 \end{pmatrix} \binom xy \equiv \binom{3}{12} \pmod{21}\]
    \[\det \begin{pmatrix} 9 & 12 \\ 11 & 5 \end{pmatrix} = -87 \equiv 18 \pmod{21}\]

    Since \(\gcd(18, 21) = 3\), this matrix is not invertible modulo 21, so there are potentially none or multiple solutions.  To check, we split into mod 3 and mod 7.
    \[\begin{pmatrix} 0 & 0 \\ 2 & 2 \end{pmatrix} \binom xy \equiv \binom{0}{0} \pmod{3} \implies 2(x+y) \equiv 0 \pmod 3.\]
    \begin{align*}
      \binom xy &\equiv \begin{pmatrix} 2 & 5 \\ 4 & 5 \end{pmatrix}^{-1}\binom35 \equiv 4^{-1} \cdot \begin{pmatrix} 5 & -5 \\ -4 & 2 \end{pmatrix} \\
                &\equiv 2 \cdot \begin{pmatrix} 5 & 2 \\ 3 & 2 \end{pmatrix} \binom35 \equiv \binom13 \pmod7
    \end{align*}
    So we have potential solutions \(\displaystyle\left\{ \binom{1}{3}, \binom{1}{10}, \binom{8}{3}, \binom{8}{10} \right\}\), but since the only \((x,y)\) pair that satisfies \(2(x+y) \equiv 0 \pmod3\) is \(\displaystyle\binom{8}{10}\), we have that \(x=8, y=10\) is our only solution with \(0 \leq x,y < 15\).  Although if we include \(x=15\), we have another solution: \(x=15,y=3\). \n

  \item Vigen\`{e}re Cipher Basics
    \begin{enumerate}
      \item Encrypt `meet me in the alley after midnight' with the Vigen\`{e}re cipher with key `pandora'. \n\\
        `meet me in the alley after midnight' + \((15,0,13,3,14,17,0)\) \\
        \(\to\) `berw av ic tuh oclty nihvr biqqwxhi' \n

      \item If you know that the plaintext ``yum, cookies'' has been encrypted to ``zun, cpolifs'' using a Vigen\`{e}re cipher, can you determine the key of that Vigen\`{e}re cipher? If so, state what the key is and list any assumptions you are making about the key. If not, what can you say about the key? \n\\
        We technically can't be sure what the cipher key is based off this information.  Going off the pattern I've noticed I would guess that the key is ``ba'', but there's also the possibility that it's ``baba'', ``bababa'', ``babababa'', ``bababababa'', or any grouping of letters that starts with the string ``bababababa''. \n
    \end{enumerate}

    \newpage
  \item Consider an alphabet with four characters: \(\spadesuit\), \(\heartsuit\), \(\diamondsuit\), and \(\clubsuit\).  Suppose that you have a text \(\underline{s}\) written in this alphabet which consists of \(70\) \(\spadesuit\)s,  \(18\) \(\heartsuit\)s, \(10\) \(\diamondsuit\)s, and \(2\) \(\clubsuit\)s.  What is \(\textrm{IndCo}(\underline{s})\)? \n\\
    Let \(S := \{\spadesuit, \heartsuit, \diamondsuit, \clubsuit\}\), \(n = 70 + 18 + 10 + 2 = 100\), and for any \(s \in S\), let \(F_{s}\) be number of \(s\)'s in our text \(\underline{s}\):
    \[\textrm{IndCo}(\underline{s}) = \frac{1}{n(n-1)} \sum_{s \in S} F_s (F_s-1) = \frac{1}{100 \cdot 99}(70 \cdot 69 + 18 \cdot 17 + 10 \cdot 9 + 2 \cdot 1) \approx 0.528\]


\end{enumerate}

\end{document}
