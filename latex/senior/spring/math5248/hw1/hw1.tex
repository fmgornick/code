 \documentclass[12pt]{article}
\usepackage[utf8]{inputenc}
\usepackage{amsmath,amsthm}
\usepackage{fullpage}
\usepackage{amsfonts}
\usepackage{amssymb,multicol}
\usepackage{graphicx}
\usepackage{hyperref, enumitem}
\usepackage{outlines}
\usepackage{xcolor}

\newcommand{\Z}{\mathbb{Z}}
\newcommand{\N}{\mathbb{N}}
\newcommand{\n}{\vspace{0.5cm}}
\newcommand{\mmod}{\;\%\;}

\newtheorem*{theorem}{Theorem}

\newlist{checklist}{itemize}{2}
\setlist[checklist]{label=$\square$}

\begin{document}
\begin{center}
  {\Large Homework 1} \n

  Fletcher Gornick
\end{center}

\begin{enumerate}
\item[0.] If you would like any of these problems to be graded for proficiency with the core skills, list the skill and the corresponding problem.
  \begin{outline}
    \1 1. Using a known plaintext attack to determine the key of a cipher of any kind.
      \2 problem 2
    \1 2. Apply the definition of divisibility in a proof.
      \2 problem 5
  \end{outline} \n

\item Compute the following reductions.  Explain your steps.  For parts (a) and (c), show your work without using any integers larger than 144. 
\begin{enumerate}
  \item $(131\cdot 142) \mmod 3$ 
    \begin{align*}
      131 \cdot 142 &= (3 \cdot 43 + 2) \cdot (3 \cdot 47 + 1) \\
                    &= 3 \cdot (3 \cdot 43 \cdot 47 + 43 \cdot 1 + 47 \cdot 2) + 2 \cdot 1 \\
                    &= 2 \pmod3 &(3x + 2 \equiv 2 \pmod {3}) \\
  \end{align*}

  \item $(-2000) \mmod 93$
  \begin{align*}
      -2000 &= -20 \cdot 100 \\
            &= -20 \cdot (97 + 3) \\
            &= -20 \cdot 97 - 20 \cdot 3 &(97x = 0 \pmod {97}) \\
            &\equiv -20 \cdot 3 \pmod{97} \\
            &= -60 \pmod{97} \\
            &= 37 \pmod{97} &(x + 97 \equiv x \pmod {97}) \\
  \end{align*}

  \item $\left(3^{74}\right) \mmod 13$ 
    \begin{align*}
      3^{74} &= 9^{37} \pmod{13} & \left(x^{2y} = (x^2)^y\right) \\
             &= 9 \cdot 9^{36} \pmod{13} & (x^{y+1} = x \cdot x^y)\\
             &= 9 \cdot 81^{18} \pmod{13} \\
             &= 9 \cdot 3^{18} \pmod{13} & \text{(*)} \\
             &= 9 \cdot 9^9 \pmod{13} \\
             &= 81 \cdot 9^8 \pmod{13} \\
             &= 3 \cdot 9^8 \pmod{13} \\
             &= 3 \cdot 81^4 \pmod{13} \\
             &= 3 \cdot 3^4 \pmod{13} \\
             &= 3 \cdot 9^2 \pmod{13} \\
             &= 3 \cdot 81 \pmod{13} \\
             &= 3 \cdot 3 \pmod{13} \\
             &= 9 \pmod{13} \\
  \end{align*}
  % *: the reason we can rewrite \(9 \cdot 81^{18}\) as \(9 \cdot 3^{18}\) 
  *: Suppose \(x,y,m,n,r \in \N\) and \(y = mx + r\) with \(0 \leq r < m\), then...
  \begin{align*}
    y^n &= (mx + r)^n \\
        &= \sum_{k=0}^n \binom n k (mx)^k \cdot r^{n-k} &\text{(binomial theorem)} \\
        &= r^n + m \left(\sum_{k=1}^n \binom n k m^{k-1} \cdot x^k \cdot r^{n-k}\right) \\
        &\equiv r^n \pmod m
  \end{align*}
\end{enumerate} \n

\item Suppose you know that the plaintext ``snow" has been encoded to the ciphertext ``IDEM" using a shift cipher.  You intercept the ciphertext ``MYDJUH," which you know has been encoded with the same shift cipher.  Decrypt the second message ``MYDJUH." Make sure to show your work.   What type of attack did you perform? \n

  \scalebox{0.74}{
      \begin{tabular}{ |c|c|c|c|c|c|c|c|c|c|c|c|c|c|c|c|c|c|c|c|c|c|c|c|c|c| }
        \hline
        A & B & C & D & E & F & G & H & I & J & K & L & M & N & O & P & Q & R & S & T & U & V & W & X & Y & Z \\
        \hline
        0 & 1 & 2 & 3 & 4 & 5 & 6 & 7 & 8 & 9 & 10 & 11 & 12 & 13 & 14 & 15 & 16 & 17 & 18 & 19 & 20 & 21 & 22 & 23 & 24 & 25 \\
        \hline
      \end{tabular}
    } \n

    `I' - `S' = \(8 - 18 = -10 \equiv 16 \pmod{26}\), so the shift key is 16 for encrypting.  If we want to decrypt ``MYDJUH'', we shift backwards 16, or forwards 10 (\(-16 \equiv 10 \pmod{26}\)).
    \begin{align*}
      \text{``MYDJUH''} \gg 10 &\equiv (E_{10}(12),E_{10}(24),E_{10}(3),E_{10}(9),E_{10}(20),E_{10}(7)) \\
                             &= (22,34,13,19,30,17) \\
                             &\equiv (22,8,13,19,4,17) \pmod{26} \\
                             &\equiv \text{``WINTER''}
    \end{align*}

    This is a ``known-plaintext attack''. \n
  
\item Let $N$ be an integer so that $N\mmod20 = 2$.  What is $N\mmod5$?  Prove that your answer is true for any integer $N$ such that $N\mmod20=2$.
  \begin{align*}
    N \mmod 20 = 2 &\iff N = 2 \pmod{20} \\
                &\iff N = 20k + 2 \text{ for some } k \in \Z \\
                &\iff N = 5 \cdot (4k) + 2 \\
                &\iff N = 2 \pmod 5 \\
                &\iff N \mmod 5 = 2
  \end{align*} 
  The penultimate step can be deduced because \(4k\) is an integer, so 2 is the reduction of \(N\) modulo 5 by the division algorithm. \n

\item Fix an integer $N$ such that $N\mmod5=2$. What are the possible values of $N\mmod20$? Explain your answer. 
  \begin{theorem}
    \(N\mmod5 = 2 \iff N\mmod20 \in \{2, 7, 12, 17\}\)
  \end{theorem}
  \begin{proof}
    \(N\mmod5 = 2\) can be rewritten as \(N = 5k + 2\) for some \(k \in \Z\).  Let's consider 4 separate cases for \(k\).

    \(k\mmod4 = 0\): So \(k = 4\ell\) for some \(\ell \in \Z\), telling us...
    \begin{align*}
      N \;&=\; 5k + 2 \;=\; 5 \cdot 4\ell + 2 \;=\; 20\ell + 2 \\
          &\iff N\mmod20 = 2
    \end{align*}

    \(k\mmod4 = 1\): So \(k = 4\ell + 1\) for some \(\ell \in \Z\), telling us...
    \begin{align*}
      N \;&=\; 5k + 2 \;=\; 5 \cdot (4\ell + 1) + 2 \;=\; 20\ell + 7 \\
          &\iff N\mmod20 = 7
    \end{align*}

    \(k\mmod4 = 2\): So \(k = 4\ell + 2\) for some \(\ell \in \Z\), telling us...
    \begin{align*}
      N \;&=\; 5k + 2 \;=\; 5 \cdot (4\ell + 2) + 2 \;=\; 20\ell + 12 \\
          &\iff N\mmod20 = 12
    \end{align*}

    \(k\mmod4 = 3\): So \(k = 4\ell + 3\) for some \(\ell \in \Z\), telling us...
    \begin{align*}
      N \;&=\; 5k + 2 \;=\; 5 \cdot (4\ell + 3) + 2 \;=\; 20\ell + 17 \\
          &\iff N\mmod20 = 17
    \end{align*}

    Since these cases cover all possible values for \(k\), we know these 4 outcomes are the only possibilities when \(N\mmod5=2\).
  \end{proof}
  
  


\item Let $a$, $b$, and $c$ be integers such that $a|b$ and $a|c$. Prove that,  for any integers $u$ and $v$, $a|ub+vc$. 
  \begin{proof}
    Since \(a\) divides both \(b\) and \(c\), we can find integers \(s\) and \(t\) such that 
    \[b = as, \;\; c = at\]
    Plugging these new values in for \(ub+vc\), we get that
    \[ub+vc \;=\; aus + avt \;=\; a (us + vt)\]
    Since \((us + vt) \in \Z\), we have that \(a|ub+vc\) by definition of divisibility.
  \end{proof}
  

\item Let $m$ be any positive integer. Prove that  if $r$ is the reduction of $N$ modulo $m$ with $r\ne 0$, then $m-r$ is the reduction of $-N$ modulo $m$.  (Note: you likely have a lot of this problem completed from the in-class activity!)

  \begin{proof}
    Given integer \(r\) such that \(0 < r < m\), we call \(r\) the reduction of \(N\) modulo \(m\) if we can write \(N = qm + r\) for some integer \(q\).
    
    Manipulating the terms, we get...
    \begin{align*}
      N = qm + r &\iff -N = -qm - r \\
                 &\iff -N = -qm - r + m - m \\
                 &\iff -N = -(q+1)m + (m-r)
    \end{align*}
    We now have \(-(q+1) \in \Z\) and \(0 < (m-r) < m\) (\(r\) assumed to be \(\neq 0\)).  Since these values satisfy the constraints of the division algorithm, we can conclude that \(m-r\) is the reduction of \(-N\) modulo \(m\).
  \end{proof}
  
\end{enumerate}

\end{document}
