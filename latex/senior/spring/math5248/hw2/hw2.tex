\documentclass[11pt]{article}
\usepackage[utf8]{inputenc}
\usepackage{amsmath,amsthm}
\usepackage{fullpage}
\usepackage{amsfonts}
\usepackage{amssymb,multicol}
\usepackage{graphicx}
\usepackage{hyperref, enumitem}
\usepackage{outlines}
\usepackage{xcolor}

\newcommand{\Z}{\mathbb{Z}}
\newcommand{\N}{\mathbb{N}}
\newcommand{\n}{\vspace{0.5cm}}
\newcommand{\mmod}{\;\%\;}

\newtheorem*{theorem}{Theorem}

\newlist{checklist}{itemize}{2}
\setlist[checklist]{label=$\square$}

\begin{document}
  \begin{center}
    {\Large Homework 1} \n

    Fletcher Gornick
  \end{center}

  \begin{enumerate}
    \item[0.] If you would like any of these problems to be graded for proficiency with the core skills, list the skill and the corresponding problem.
      \begin{outline}
        \1 2. Apply the definition of divisibility in a proof.
          \2 problem 5
        \1 3. Apply the definition of a unit modulo m in a proof.
          \2 problem 3
      \end{outline} \n

    \item Write out all elements of \(\Z_8\) using two different collections of representatives for the equivalence classes. (Note: Your answer will have two different sets; each equivalence class in the first set should be equal to an equivalence class in the second set.)
      \begin{align*}
        \Z_8 &= \{\bar0,\bar1,\bar2,\bar3,\bar4,\bar5,\bar6,\bar7\} \\
             &= \{\bar8,\bar9,\bar{10},\bar{11},\bar{12},\bar{13},\bar{14},\bar{15}\}
      \end{align*}

    \item Make a multiplication table for \(\Z_{10}\) Use it to identify the units (invertible elements) of \(\Z_{10}\).  What is \(\varphi(10)\)?
      \[
        \begin{tabular}[c]{l|l l l l l l l l l l}
          \(\times\) & 0 & 1 & 2 & 3 & 4 & 5 & 6 & 7 & 8 & 9 \\
          \hline
          0 & 0 & 0 & 0 & 0 & 0 & 0 & 0 & 0 & 0 & 0 \\
          1 & 0 & 1 & 2 & 3 & 4 & 5 & 6 & 7 & 8 & 9 \\
          2 & 0 & 2 & 4 & 6 & 8 & 0 & 2 & 4 & 6 & 8 \\
          3 & 0 & 3 & 6 & 9 & 2 & 5 & 8 & 1 & 4 & 7 \\
          4 & 0 & 4 & 8 & 2 & 6 & 0 & 4 & 8 & 2 & 6 \\
          5 & 0 & 5 & 0 & 5 & 0 & 5 & 0 & 5 & 0 & 5 \\
          6 & 0 & 6 & 2 & 8 & 4 & 0 & 6 & 2 & 8 & 4 \\
          7 & 0 & 7 & 4 & 1 & 8 & 5 & 2 & 9 & 6 & 3 \\
          8 & 0 & 8 & 6 & 4 & 2 & 0 & 8 & 6 & 4 & 2 \\
          9 & 0 & 9 & 8 & 7 & 6 & 5 & 4 & 3 & 2 & 1 \\
        \end{tabular}
      \] 
      This table shows us that \(1 \cdot 1 \equiv 1 \pmod{10}\), \(3 \cdot 7 \equiv 1 \pmod{10}\), and \(9 \cdot 9 \equiv 1 \pmod{10}\), so the units modulo 10 are 1,3,7, and 9.  This tells us \(\varphi(10) = 4\). \n
      

    \item Let \(n\) be a positive integer. Prove that \(n\) is a unit modulo \(5n-1\).

      We can simply multiply \(n\) by 5 to give us \(5n = (5n-1) + 1\), this tells us
      \[5n = (5n-1) + 1 \equiv 1 \pmod{5n-1}.\]
      From above we see that 5 is the mulitplicative inverse of \(n\) modulo \(5n-1\) (by definition of multiplicative inverse).  So we can conclude that \(n\) is a unit modulo \(5n-1\) via the definition.
      \newpage

    \item Prove or disprove each of the following statements.
      \begin{enumerate}[label=(\alph*)]
        \item For positive integers \(n\) and \(N\) and any integer \(x\), \[(x \mmod N) \mmod n = x \mmod n.\]
          This is false.  Take, for example, \(x = 10, n = 4, N = 5\), we have that \[(x \mmod N) \mmod n = (10 \mmod 5) \mmod 4 = 0 \mmod 4 = 0 \neq 2 = 10 \mmod 4 = x \mmod n.\]

        \item For integers \(a,b,c\) and \(m\) with \(m,c > 0\), if \(a \equiv b \pmod{mc}\), then \(a \equiv b \pmod m\).
          \begin{proof}
            If \(a \equiv b \pmod{mc}\), then \(mc \mid (a-b)\) (by definition), that is, \[a-b = mck\] for some \(k \in \Z\).  Adding parentheses, we get that \[a-b = m (ck),\] and since \(ck \in \Z\), we have that \(a \equiv b \pmod m\) by definition of modular congruence.
          \end{proof} \n
      \end{enumerate}
    
    \item Suppose that \(a = bq + r\) for some integers \(a,b,q\) and \(r\).  Without using any properties of the gcd besides the definition as the largest common divisor of \(x\) and \(y\), prove that \(\gcd(a,b) = \gcd(b,r)\).  (Hint: \(x=y\) if and only if \(x \leq y\) and \(x \geq y\).)
      \begin{proof}
        Let \(c\) be the greatest common divisor of \(a\) and \(b\), or, in other words, let \(c\) be the largest integer such that \(c|a\) and \(c|b\).  Similarly, let \(d\) be the greatest common divisor of \(b\) and \(r\).  We show that \(c=d\).

        First, note that if \(d|b\) and \(d|r\), then \(b = \ell d\), \(r = kd\) for \(k,\ell \in \Z\).  This then tells us that 
        \[a = bq + r = (\ell d)q + (kd) = (\ell q + k)d,\]
        and since \((\ell q+k) \in \Z\), \(d\) must divide \(a\) as well by definition of divisibility.  Now since \(d\) divides both \(a\) and \(b\), we know that \(d\) is a common divisor, we also know that \(c\) is the greatest common divisor of \(a\) and \(b\), so we can conclude that \(c \geq d\).

        Just as we did above, we can write \(a = sc\), \(b = tc\) for some \(s,t \in \Z\).  Again, plugging in these values, we get
        \begin{align*}
          a = bq+r \iff r &= a-bq \\
                          &= (sc) - (tc)q \\
                          &= (s - tq)c
        \end{align*}
        Again, since \((s - tq) \in \Z\), we have that \(c\) divides \(r\) by definition of divisibility.  Now, since \(c|b\) and \(c|r\), we know that \(c\) is a common divisor of \(b\) and \(r\), but \(d\) is the greatest common divisor of \(b\) and \(r\), so \(c \leq d\).

        In conclusion, since \(\gcd(a,b) \leq \gcd(b,r)\) and \(\gcd(a,b) \geq \gcd(b,r)\), it must be the case that \(\gcd(a,b) = \gcd(b,r)\).
      \end{proof}
      
  \end{enumerate}
\end{document}
