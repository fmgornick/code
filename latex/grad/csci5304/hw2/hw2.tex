\documentclass[11pt]{article}

\usepackage{amsmath}
\usepackage{amsfonts}
\usepackage{amsthm}
\usepackage{blkarray}
\usepackage{caption}
\usepackage{enumitem}
\usepackage{hyperref}
\usepackage{mathtools}
\usepackage{tikz}
\usepackage[top=1.5cm,bottom=2cm,left=1.25cm,right=1.75cm,marginparwidth=1.75cm]{geometry}
\setlength{\parindent}{0cm}

\newcommand{\R}{\mathbb{R}}
\newcommand{\n}{\vspace{0.3cm}}
\newtheorem{theorem}{Theorem}

\def\lc{\left\lceil}
\def\rc{\right\rceil}
\def\lf{\left\lfloor}
\def\rf{\right\rfloor}

\title{\vspace{-1.0cm}CSCI 5304 Homework 2 }
\author{Fletcher Gornick}
\date{September 30}

\begin{document}
\maketitle
\begin{enumerate}
	\item \begin{enumerate}
		      \item Solve the linear system \(Ax = b\) by Gaussian elimination, where:
		            \[A =
			            \begin{pmatrix}
				            1  & -2 & -1 & 0  \\
				            -2 & 3  & 2  & 1  \\
				            -1 & 2  & 3  & -2 \\
				            0  & -1 & -4 & 6
			            \end{pmatrix}
			            \quad b = \begin{pmatrix} -3 \\ 5 \\ 1 \\ 4 \end{pmatrix}
		            \]

		      \item What is the \(LU\) factorization of \(A\), what is it's determinant?

		      \item Using the \(LU\) factors obtained in (b), find the last column of the inverse of \(A\), without computing the whole inverse.
	      \end{enumerate}

	\item Let \(A = LU\) be the \(LU\) factorization of \(A \in \R^{n \times n}\), with \(|\ell_{i,j}| \leq 1\).  Verify the equation:
	      \[u_{i,:} = a_{i,:} - \sum_{j=1}^{i-1} \ell_{i,j}u_{j,:},\]
	      Then use this relation to show that \(\lVert U \rVert_\infty \leq 2^{n-1} \lVert A \rVert_\infty\).

	\item For the following matrix:
	      \[A =
		      \begin{pmatrix}
			      1 & -1 & 1 \\
			      0 & 4  & 2 \\
			      6 & 2  & 0
		      \end{pmatrix}
	      \]
	      \begin{enumerate}
		      \item Determine the standard \(LU\) factorization of the matrix on the right.
		      \item Compute the determinant of \(A\).
		      \item Compute the first column of the inverse of \(A\)
		      \item Repeat the above questions when partial pivoting is used, i.e. find the permutation matrix \(P\) and the matrices \(L,U\) such that \(PA = LU\).  Compute the determinant of \(A\) based on this factorization, and compute the first column of the inverse of \(A\), based on this factorization.
	      \end{enumerate}

	\item We saw that Gaussian elimination is equivalent to multiplying the initial matrix \(A\) by a swquence of Gaussian transformations from the left.  This exercise explores what happens in the case of Gauss-Jordan (\(GJ\)) elimination.
	      \begin{enumerate}
		      \item Show that for Gauss-Jordan, we have the same result:
		            \[A_k = M_kA_{k-1}, \; k = 1, 2, \dots, n,\]
		            With \(A_0 = A\)
	      \end{enumerate}
\end{enumerate}
\end{document}
