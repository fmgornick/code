\documentclass[11pt]{article}

\usepackage{amsmath}
\usepackage{amsfonts}
\usepackage{amsthm}
\usepackage{blkarray}
\usepackage{caption}
\usepackage{enumitem}
\usepackage{hyperref}
\usepackage{mathtools}
\usepackage{tikz}
\usepackage[top=1.5cm,bottom=2cm,left=1.25cm,right=1.75cm,marginparwidth=1.75cm]{geometry}
\setlength{\parindent}{0cm}

\newcommand{\N}{\mathbb{N}}
\newcommand{\R}{\mathbb{R}}
\newcommand{\C}{\mathbb{C}}
\newcommand{\n}{\vspace{0.3cm}}
\newtheorem{theorem}{Theorem}

\def\lc{\left\lceil}
\def\rc{\right\rceil}
\def\lf{\left\lfloor}
\def\rf{\right\rfloor}

\title{\vspace{-1.0cm}CSCI 5304 Homework 4}
\author{Fletcher Gornick}
\date{November 9, 2023}

\begin{document}
\maketitle
\begin{enumerate}
	\item \begin{enumerate}
		      \item Show that a real \(n \times n\) matrix \(A\) (not necessarily symmetric) is positive definite iff \(\frac12(A + A^T)\) is SPD.
		            \begin{proof}
			            Let \(x\) be any nonzero vector,
			            \begin{align*}
				            \left(\frac12(A + A^T)x, x \right) > 0
				             & \iff \frac12 x^T(A + A^T)x > 0              & \text{(constants can be pulled out)} \\
				             & \iff \frac12 x^T Ax + \frac12 x^T A^T x > 0 & (\text{distribution of vectors})     \\
				             & \iff \frac12(Ax, x) + \frac12(x, Ax) > 0    & (\text{definition of inner product}) \\
				             & \iff \frac12(Ax, x) + \frac12(Ax, x) > 0    & (\text{symmetry of inner product})   \\
				             & \iff (Ax, x) > 0.
			            \end{align*}
			            Since \(\left(\frac12 (A + A^T),x \right) > 0 \text{ iff } (Ax,x) > 0\) for all nonzero \(x\), we can conclude the above statement holds.
		            \end{proof}

		      \item Find a \(2 \times 2\) real matrix which satisfies \((Ax,x) > 0\) for all \textit{real} nonzero vectors \(x\), but which is not positive definite when regarded as a member of \(\C^{2 \times 2}\).
		            \[(Ax,x) = x^TAx = (x_1 \; x_2) \begin{pmatrix} a_{11} & a_{12} \\ a_{21} & a_{22} \end{pmatrix} \binom{x_1}{x_2} = a_{11}x_1^2 + a_{12}xy + a_{21}xy + a_{22}y^2,\]
		            So if we choose \(A = \begin{pmatrix} 1 & 1 \\ -1 & 1 \end{pmatrix}\), we see that \((Ax,x) = x_1^2 + x_2^2 > 0\) for all \(x \in \R^2\), but if we treat \(A\) as complex, we can take \(x = \binom1i\), and we get
		            \[(Ax,x) = a_{11} \bar x_1 x_1 + a_{21} x_1 \bar x_2 + a_{12} \bar x_1 x_2 + a_{22} \bar x_2 x_2 = 1 \cdot 1 \cdot 1 -1 \cdot 1 \cdot (-i) + 1 \cdot 1 \cdot i + 1 \cdot (-i) \cdot i = 2 + 2i \not \in \R\]
		            Since there's an \(x \in \C\), \(x \neq 0\) such that \((Ax,x) \not \in \R\), we know \(A\) cannot be positive definite in the complex sense.
	      \end{enumerate} \n

	\item Let \(A \in \R^{n \times n}\) be a symmetric positive definite matrix.
	      \begin{enumerate}
		      \item Show that \(|a_{ij}| < \sqrt{a_{ii}a_{jj}}\)
            \begin{proof}
              Since \(A\) SPD, we know there exists \(m \times n\) matrix \(G\) such that \(A = G^T G\), so we can calculate \(a_{ij}\) like so:
              \[a_{ij}^2 = \left( \sum_{k=1}^m (G^T)_{ik}(G)_{kj} \right)^2 = \left( \sum_{k=1}^m g_{ki}g_{kj} \right)^2 \leq \left( \sum_{k=1}^m g_{ki}^2 \right) \left( \sum_{k=1}^m g_{kj}^2 \right) = a_{ii}a_{jj} \]
              
              The inequality from above was an application of the Cauchy-Schwarz inequality.  This isn't exacly what we want though, we still must show that \(|a_{ij}| \neq \sqrt{a_{ii}a_{jj}}\), so let's take vector \(x \in \R^n\) consisting of all zeros, except \(x_i = \sqrt{a_{jj}}, \; x_j = -\sqrt{a_{ii}}\), and assume \(a_{ij} = \sqrt{a_{ii}a_{jj}}\).  This yields:
              \[(Ax,x) = x^T A x = \textstyle\sum\limits_{k=1}^n \textstyle\sum\limits_{\ell=1}^n x_k a_{k \ell} x_\ell = x_{ii}^2a_{ii} + x_{jj}^2a_{jj} - 2x_ix_ja_{ij} = a_{jj}a_{ii} + a_{ii}a_{jj} - 2\sqrt{a_{ii}}\sqrt{a_{jj}}\sqrt{a_{ii}a_{jj}} = 0.\]
              Therefore it also cannot be the case that \(|a_{ij}| = \sqrt{a_{ii}a_{jj}}\), and we can conclude \(|a_{ij}| < \sqrt{a_{ii}a_{jj}}\) if \(A\) is symmetric positive definite.
            \end{proof}

		      \item Show that \(|a_{ij}| < (a_{ii}+a_{jj})/2\)
            \begin{proof}
              If we take \(x = e_i - e_j\), we get the following value for \((Ax,x)\):
                \[(Ax,x) = x_i^2a_{ii} + x_j^2a_{jj} - 2x_ix_ja_{ij} = a_{ii} + a_{jj} - 2a_{ij}.\]
              Since \(A\) is SPD, \(a_{ii} + a_{jj} - 2a_{ij} > 0 \implies a_{ij} < (a_{ii}+a_{jj})/2\).  Similarly, taking \(x = e_i + e_j\) yields \(-a_{ij} < (a_{ii}+a_{jj})/2\).  Therefore, we can conclude \(|a_{ij}| < (a_{ii}+a_{jj})/2\).
            \end{proof}
	      \end{enumerate}

	\item Show that the following symmetric matrices are not positive definite.
	      \[
		      A = \begin{pmatrix} 0 & 3 & 1 \\ 3 & 2 & 0 \\ 1 & 0 & 2 \end{pmatrix}, \quad
		      B = \begin{pmatrix} 2 & 3 & 1 \\ 3 & 2 & 0 \\ 1 & 0 & 2 \end{pmatrix}, \quad
		      C = \begin{pmatrix} 1 & 0 & 2 \\ 0 & 2 & 0 \\ 2 & 0 & 2 \end{pmatrix}.
	      \]
        \begin{proof}
        For matrix \(A\), we see \(a_{11} = 0\), so if we simply take \(e_1\), we see \(e_1^T A e_1 = 0\). \n\\
        For matrix \(B\), we see \(b_{11} = b_{22} = 2\) and \(b_{12} = 3\), but \(b_{12} = 3 \geq 2 = \sqrt{b_{11}b_{22}}\), this violates the statement proven in 2(a), so we can conclude \(B\) cannot be positive definite. \n\\
        Similarly, in matrix \(C\), we see \(c_{11} = 1,  c_{33} = 2,\) and \(c_{13} = 2\).  These values also violate the statement proven in 2(a), so we can again confirm \(C\) is not SPD (via contrapositive).
        \end{proof}

	\item The Hald cement data is used in several books and papers as an example of regression and least-squares.  The right-hand side is the heat evolved in cement during hardening.  The variables are coefficients \(\xi_1, \dots, \xi_4\) of four different ingredients of the mix.  The right-hand side \(b\) and the matrix \(A \in \R^{13 \times 4}\) can be found in the matlab section on the class web-site.  To \(\xi_1, \dots, \xi_4\) we add a constant (\(\xi_0\)) to the model.  In effect, we want to find \(\xi_0, \xi_1, \dots, \xi_4\) so that \(\xi_0 + \xi_1 a_{i,1} + \cdots + \xi_4 a_{i,4} \approx b_i\) for \(i=1:13\).
	      \begin{enumerate}
		      \item Solve the least squares problem to get the \(\xi_i\)'s by the method of normal equations.  What is \(\kappa_2(A)\)?

		      \item We now show how to get rid of the constant unknown from the system.  Write \(x = \binom{\xi_0}{y}\) where \(\xi_0\) is a scalar, and show how to eliminate \(\xi_0\) from the system.  The resulting problem is now a least-squares problem of the form \(\min \lVert By-c \rVert_2\) involving only the \(y\) vector.  Show the matrix \(B\) and new right-hand side \(c\).  What is the condition number of \(B\)?

		      \item Continued from (b).  How can you interpret \(c\) relative to \(b\) and \(B\) relative to \(A\)?
	      \end{enumerate}

	\item The purpose of this exercise is to test 3 ways of computing the QR factorization of the matrix \(A\);
	      \begin{enumerate}
		      \item The classical Gram-Schmidt algorithm
		      \item The modified Gram-Schmidt algorithm
		      \item The Cholesky factorization of \(A^T A\)
	      \end{enumerate}

	      Explain how the Cholesy factorization of \(A^T A\) can be used.  In the following, you should us the script \verb!cholR! that's posted.  You can use \verb!inv! to invert triangular matrices.

	      A data set is posted on the class web-site.  Write a script which loads the matrix and then for each of the three methods above, compute the \(Q\) and \(R\) factors and the error measures
	      \[\lVert A - Q * R \rVert_2, \quad \lVert I - Q^T * Q \rVert_2.\]
	      Present your results in the form of a table and comment on them.

	\item We saw in class that if \(w \in \R^n\) and \(\lVert w \rVert_2 = 1\), then the matrix \(P = I - 2ww^T\), called a Householder reflector, is at the same time symmetric and unitary.  Let \(x \in \R^n\), with \(x \neq 0\), and let \(y = Px\).
	      \begin{enumerate}
		      \item When do we have \(y = x\)?
		      \item Assume now that \(w^T x > 0\).  Show that \(y \neq x\) and that \(w = (x-y)/\lVert x-y \rVert_2\).
		      \item Show a geometric illustration of the result in (b).
	      \end{enumerate}

	\item Implement a julia function which computes the Householder QR factorization of a full rank matrix \(A\).  The function should look like:
	      \[\verb!V, bet = houseQR(A)!\]
	      where \verb!V! is the matrix of vectors \(v_1, \dots, v_m\) that are related to the successive Householder reflectors \(P_j = I - \beta_j v_j v_j^T\) used to transform \(A\) into upper triangular form, and \verb!bet! is the vector of coefficients \(\beta_j\).  Show the julia function.

	\item Test the program developed above for the same data as the one used for Question 5 above.  Obtain the \(Q_1, R_1\) matrices of the factorization \(A = Q_1R_1\) - where \(Q_1\) is \(m \times n\) and \(R_1\) is \(n \times n\), from the Householder factorization.  Compare the \(R_1\) matrix obtained from the Householder method with the \(R\) matrix obtained from the Modified Gram-Schmidt method seen in Question 5 (\(\lVert R_1 - R \rVert_1\)).  Similarly, compute the norms \(\lVert I - Q^T Q \rVert_1\) and \(\lVert I - Q_1^T Q_1 \rVert_1\).
\end{enumerate}
\end{document}
