\documentclass[10pt]{article}

\title{CSCI 2011 HW 3}
\author{Fletcher Gornick}

\usepackage{amsmath,amssymb}
\usepackage{enumitem}
\usepackage{colortbl}
\usepackage{soul}
\usepackage{xcolor}
\usepackage[margin=1.0in]{geometry}

\def \n {\par \vspace{\baselineskip}}

\begin{document}
\maketitle

\section{Problem 1 (2 Parts)}
\subsection{3.1 Problem 14}
\textbf{Consider the following quantified statement: For every even integer $a$ and every odd integer
$b$, $a + b$ is odd.}

\begin{enumerate}[label=(\alph*)]
    
    \item \textbf{Express this quantified statement in symbols.}

        Let $S$ be the set of all even integers and let $T$ be the set of all odd integers.

        $\forall a \in S, \forall b \in T: a+b \in T$


    \item \textbf{Express the negation of this quantified statement in symbols.}

        $\exists a \in S, \exists b \in T: a+b \not\in T$
        

    \item \textbf{Express the negation of this quantified statement in words.}

        There exists an even integer $a$, and an odd integer $b$, such that $a+b$ is not odd.


\end{enumerate}



\subsection{3.1 Problem 18}
\textbf{State the negation of the quantified statement below.}

\textbf{For every integer $a$, there exists an integer $b$ such that $|\frac{a+1}{2} - b| \leq 1$.}

\n
There exists an integer $a$ such that for every integer $b$, $|\frac{a+1}{2} - b| > 1$.


\section{3.2 Problem 16}
\textbf{Prove that if $a$ and $b$ are positive integers, then $\frac{a}{b} + \frac{b}{a} \geq 2$.}

Assume that $\frac{a}{b} + \frac{b}{a} \geq 2$, it follows that...

\begin{align*}
    && \frac{a}{b} + \frac{b}{a} \geq 2 &\Rightarrow \frac{a^2 + b^2}{ab} \geq 2 &\\
    && &\Rightarrow a^2 + b^2 \geq 2ab &\\
    && &\Rightarrow a^2 - 2ab + b^2 \geq 0 &\\
    && &\Rightarrow (a - b)^2 \geq 0 &\\
\end{align*}

We know that $(a-b)^2 \geq 0$, because any number squared is greater than or equal to zero (even negative).

Therefore it must be the case that $\frac{a}{b} + \frac{b}{a} \geq 2$.

\section{3.2 Problem 18}
\textbf{Prove the following:}

\begin{enumerate}[label=(\alph*)]

    \item \textbf{If $a$ and $b$ are even integers, then $a+b$ is even.}

        Let $a = 2k$, and $b = 2l$, for $k,l \in \mathbb{Z}$ (definition of an even number),
        it follows that $a + b = 2k + 2l = 2(k + l)$.  Since $k + l$ is an integer, $2(k + l)$
        must be an even integer (by definition again).  Therefore $a + b$ is even.


    \item \textbf{If $c$ and $d$ are even integers, do we know that $c+d$ is even?}

        Yes, because the previous proof applies to all even integers, not just $a$ and $b$.


    \item \textbf{For integers $x$ and $y$, if we know $x+y$ is even, do we know that $x$ and $y$
        are even?}

        No.  For example, say $x = 1$, and $y = 3$.  $1 + 3 = 4$, which is even, but both x and y are odd.


    \item \textbf{If $a$ and $b$ are integers that are not both even, do we know that $a+b$ is
        not even?}

        No.  Just like in the previous example, say $a = 1$ and $b = 3$.  $a$ and $b$ are integers that are not
        both even in this case, but their sum is 4, which is even.

\end{enumerate}



\section{3.3 Problem 8}
\textbf{Give a proof of
    \begin{align*}
        \text{Let } n \in \mathbb{Z} \text{.  Then } n - 3 \text{ Is even if and only if } n + 4 \text{ is odd.}
    \end{align*}
using}

\begin{enumerate}[label=(\alph*)]

    \item \textbf{two direct proofs.}

        \begin{itemize}
            \item Assume $n + 4$ is odd.  This would mean that for some integer $k$, $n + 4 = 2k + 1$.  It follows that
                $n - 3 = 2k - 6 = 2(k - 3)$.  Since $k - 3$ is an integer, and $n - 3 = 2(k - 3)$, $n - 3$ must be even.
        \end{itemize}

        \begin{itemize}
            \item Now let's assume $n - 3$ is even.  This means that for some integer $k$, $n - 3 = 2k$.  It follows that
                $n + 4 = 2k + 7 = 2(k + 3) + 1$.  Since $k + 3$ is an integer, and $n + 4 = 2(k + 3) + 1$, $n + 4$ must be odd.
        \end{itemize}


    \item \textbf{one direct proof and one proof by contrapositive}

        \begin{itemize}
            \item DIRECT PROOF: Assume $n + 4$ is odd.  This would mean that for some integer $k$, $n + 4 = 2k + 1$.  It follows 
                that $n - 3 = 2k - 6 = 2(k - 3)$.  Since $k - 3$ is an integer, and $n - 3 = 2(k - 3)$, $n - 3$ must be even.
        \end{itemize}

        \begin{itemize}
            \item PROOF BY CONTRAPOSITIVE: Now let's assume that $n + 4$ is not odd, or simply, $n + 4$ is even.  This means that 
                there exists an integer $k$ such that $n + 4 = 2k$.  It follows that $n - 3 = 2k - 7 = 2(k - 4) + 1$.  Since $k - 4$
                is an integer, and $n - 3 = 2(k - 4) + 1$, $n - 3$ must be odd.
        \end{itemize}


    \item \textbf{two proofs by contrapositive.}

        \begin{itemize}
            \item Assume that $n + 4$ is not odd, or simply, $n + 4$ is even.  This means that there exists an integer $k$ 
                such that $n + 4 = 2k$.  It follows that $n - 3 = 2k - 7 = 2(k - 4) + 1$.  Since $k - 4$ is an integer, and 
                $n - 3 = 2(k - 4) + 1$, $n - 3$ must be odd.
        \end{itemize}

        \begin{itemize}
            \item Now let's assume that $n - 3$ is odd.  This means that there exists an integer $k$ such that $n - 3 = 2k + 1$.
                It follows that $n + 4 = 2k + 8 = 2(k + 4)$.  Since $k + 4$ is an integer, and $n + 4 = 2(k + 4)$, $n + 4$ must be even.
        \end{itemize}
    

\end{enumerate}





\section{3.3 Problem 12}
\textbf{Let $a,b$ and $m$ be integers.  Prove that if $2a + 3b \geq 12m +1$, then $a \geq 3m + 1$ or $b \geq 2m + 1$.}

We will proceed by examining the contrapositive...

Let $a,b$ and $m$ be integers.  Prove that if $a < 3m + 1$ and $b < 2m + 1$, then $2a + 3b < 12m + 1$.

\n
Since $a$, $b$, and $m$ are integers, the statement can be rewritten as..

If $a \leq 3m$ and $b \leq 2m$, then $2a + 3b < 12m + 1$.

\n
Therefore $2a + 3b \leq 2(3m) + 3(2m) \leq 12m < 12m + 1$.

\n
Thus, either $a \geq 3m + 1$ or $b \geq 2m + 1$, if $2a + 3b \geq 12m + 1$.

\end{document}
