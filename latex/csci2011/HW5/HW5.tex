\documentclass[10pt]{article}

\title{CSCI 2011 HW 5}
\author{Fletcher Gornick}

\usepackage{amsmath,amssymb}
\usepackage{enumitem}
\usepackage{colortbl}
\usepackage{soul}
\usepackage{xcolor}
\usepackage[margin=1.0in]{geometry}

\def \n {\par \vspace{\baselineskip}}
\setlength{\parindent}{0cm}

\begin{document}
\maketitle

\section{4.1 Problem 10}
\textbf{Let $r \geq 2$ be an integer.  Prove that $1 + r + r^2 + ... + r^n = \frac{r^{n+1} - 1}{r - 1}$ for every positive integer $n$.}

\n
Base Case: Since this claim must hold true for every positive integer $n$, we can use $n = 1$ as our base case.  
Therefore $1 + \cdots r^n = 1 + r^1 = 1 + r$.  Since $\frac{r^{n+1} - 1}{r - 1} = \frac{r^2 - 1}{r - 1} = \frac{(r+1)(r-1)}{r-1} = r + 1$,
our base case holds.

\n
Inductive Step: Now let's assume for some $k \geq 1$, that $1 + r + r^2 + ... + r^k = \frac{r^{k+1} - 1}{r - 1}$.  We show that
$1 + r + r^2 + ... + r^k + r^{k+1} = \frac{r^{k+2} - 1}{r - 1}$.

\begin{align*}
    && 1 + r + r^2 + ... + r^k + r^{k+1} &= \frac{r^{k+1} - 1}{r - 1} + r^{k+1} && \text{(by the inductive hypothesis)} \\
    && &= \frac{r^{k+1} - 1 + r^{k+1}(r-1)}{r - 1} \\
    && &= \frac{r^{k+1}(1 + r - 1)}{r - 1} - \frac{1}{r - 1} && \\
    && &= \frac{r \cdot r^{k+1} - 1}{r - 1} && \\
    && &= \frac{r^{k+2} - 1}{r - 1} && \\
\end{align*}

Therefore the claim holds for the inductive step as well.  Hence, by the principle of mathematical induction, the claim 
$1 + r + r^2 + ... + r^n = \frac{r^{n+1} - 1}{r - 1}$ is true for all integers $n \geq 1$.


\section{4.2 Problem 14}
\textbf{Prove that $\frac{1}{\sqrt{1}} + \frac{1}{\sqrt{2}} + \cdots + \frac{1}{\sqrt{n}} > \sqrt{n + 1}$ for ever integer $n \geq 3$.}



\section{4.2 Problem 16}
\textbf{Prove for every positive integer $n$ that $2! \cdot 4! \cdot 6! \cdots (2n)! \geq ((n + 1)!)^n$.}



\section{4.3 Problem 14}
\textbf{A sequence $a_1,a_2,a_3...$ is defined recursively by $a_1 = 3$ and $a_n = 2a_{n-1} + 1$ for $n \geq 2$.}

\begin{enumerate}[label=(\alph*)]

    \item \textbf{Determine $a_2, a_3, a_4$ and $a_5$.}

    \item \textbf{Based on the variables obtained in (a), make a guess for a formula for $a_n$ for every positive integer $n$ and use induction to verify that your guess is correct.}

\end{enumerate}



\section{4.3 Problem 22}
\textbf{Use induction to show the following for Fibonacci numbers: $F_2 + F_4 + \cdots + F_{2n} = F_{2n+1} - 1$ for every positive integer $n$.}





\end{document}
