\documentclass[10pt]{article}

\title{CSCI 2011 HW 5}
\author{Fletcher Gornick}

\usepackage{amsmath,amssymb}
\usepackage{enumitem}
\usepackage{colortbl}
\usepackage{soul}
\usepackage{xcolor}
\usepackage[margin=1.0in]{geometry}

\def \n {\par \vspace{\baselineskip}}
\setlength{\parindent}{0cm}

\begin{document}
\maketitle

\section{4.1 Problem 10}
\textbf{Let $r \geq 2$ be an integer.  Prove that $1 + r + r^2 + ... + r^n = \frac{r^{n+1} - 1}{r - 1}$ for every positive integer $n$.}

\n
Base Case: Since this claim must hold true for every positive integer $n$, we can use $n = 1$ as our base case.  
Therefore $1 + \cdots r^n = 1 + r^1 = 1 + r$.  Since $\frac{r^{n+1} - 1}{r - 1} = \frac{r^2 - 1}{r - 1} = \frac{(r+1)(r-1)}{r-1} = r + 1$,
our base case holds.

\n
Inductive Step: Now let's assume for some $k \geq 1$, that $1 + r + r^2 + ... + r^k = \frac{r^{k+1} - 1}{r - 1}$.  We show that
$1 + r + r^2 + ... + r^k + r^{k+1} = \frac{r^{k+2} - 1}{r - 1}$.

\begin{align*}
    && 1 + r + r^2 + ... + r^k + r^{k+1} &= \frac{r^{k+1} - 1}{r - 1} + r^{k+1} && \text{(by the inductive hypothesis)} \\
    && &= \frac{r^{k+1} - 1 + r^{k+1}(r-1)}{r - 1} \\
    && &= \frac{r^{k+1}(1 + r - 1)}{r - 1} - \frac{1}{r - 1} && \\
    && &= \frac{r \cdot r^{k+1} - 1}{r - 1} && \\
    && &= \frac{r^{k+2} - 1}{r - 1} && \\
\end{align*}

Therefore the claim holds for the inductive step as well.  Hence, by the principle of mathematical induction, the claim 
$1 + r + r^2 + ... + r^n = \frac{r^{n+1} - 1}{r - 1}$ is true for all integers $n \geq 1$.


\section{4.2 Problem 14}
\textbf{Prove that $\frac{1}{\sqrt{1}} + \frac{1}{\sqrt{2}} + \cdots + \frac{1}{\sqrt{n}} > \sqrt{n + 1}$ for ever integer $n \geq 3$.}

\n
Base Case: Let $n = 3$, $\frac1{\sqrt1} + \frac1{\sqrt2} + \frac1{\sqrt3} \approx 2.284 > 2 = \sqrt{3 + 1}$.  Therefore, our base case holds.

\n
Inductive Step: Assume for some integer $k \geq 3$, that $\frac{1}{\sqrt{1}} + \frac{1}{\sqrt{2}} + \cdots + \frac{1}{\sqrt{k}} > \sqrt{k + 1}$.  
We show that $\frac{1}{\sqrt{1}} + \frac{1}{\sqrt{2}} + \cdots + \frac{1}{\sqrt{k}} + \frac{1}{\sqrt{k + 1}} > \sqrt{k + 2}$.
By the inductive hypothesis, we know that...

$\frac{1}{\sqrt{1}} + \frac{1}{\sqrt{2}} + \cdots + \frac{1}{\sqrt{k}} + \frac{1}{\sqrt{k + 1}} > \sqrt{k + 1} + \frac{1}{\sqrt{k + 1}}
= \frac{k + 1 + 1}{\sqrt{k + 1}}
= \frac{k + 2}{\sqrt{k + 1}} 
> \frac{k + 2}{\sqrt{k + 2}} 
= \sqrt{k + 2}$.

\n
Therefore, by the principle of mathematical induction, the claim $\frac{1}{\sqrt{1}} + \frac{1}{\sqrt{2}} + \cdots + \frac{1}{\sqrt{n}} > \sqrt{n + 1}$
holds for all integers $n \geq 3$.



\section{4.2 Problem 16}
\textbf{Prove for every positive integer $n$ that $2! \cdot 4! \cdot 6! \cdots (2n)! \geq ((n + 1)!)^n$.}

\n
Base Case: Since $n$ can be any positive integer, we can look at when $n = 1$, so $(2 \cdot 1)! = 2$, and
$((n + 1)!)^n = ((1 + 1)!)^1 = 2$.  So the base case holds.

\n
Inductive Step: Now we can assume for some integer $k \geq 1$, that $2! \cdot 4! \cdot 6! \cdots (2k)! \geq ((k + 1)!)^k$.  We show that
$2! \cdot 4! \cdot 6! \cdots (2k)! \cdot (2k + 2)! \geq ((k + 2)!)^{k + 1}$

\begin{align*}
    && 2! \cdot 4! \cdot 6! \cdots (2k)! \cdot (2k + 2)! &\geq ((k + 1)!)^k(2k + 2)! && \text{(by the inductive hypothesis)} \\
    && &\geq ((k + 1)!)^k(k + 1)!(k + 2)^{k + 1} && \Bigg(\frac{(2k + 2)!}{(k + 1)!} > (k + 2)^{k + 1}\Bigg) \\
    && &= ((k + 1)!)^{k + 1}(k + 2)^{k + 1} && \\
    && &= ((k + 2)(k + 1)!)^{k + 1} && \\
    && &= ((k + 2)!)^{k + 1} && \\
\end{align*}

Therefore, the claim holds for the inductive step.  Hence, by the principle of mathematical induction, the claim
$2! \cdot 4! \cdot 6! \cdots (2n)! \geq ((n + 1)!)^n$ holds for all integers $n \geq 1$.



\section{4.3 Problem 14}
\textbf{A sequence $a_1,a_2,a_3...$ is defined recursively by $a_1 = 3$ and $a_n = 2a_{n-1} + 1$ for $n \geq 2$.}

\begin{enumerate}[label=(\alph*)]

    \item \textbf{Determine $a_2, a_3, a_4$ and $a_5$.}

        $a_2 = 2a_1 + 1 = 2 \cdot 3 + 1 = 7$. \\
        $a_3 = 2a_2 + 1 = 2 \cdot 7 + 1 = 15$. \\
        $a_4 = 2a_3 + 1 = 2 \cdot 15 + 1 = 31$. \\
        $a_5 = 2a_4 + 1 = 2 \cdot 31 + 1 = 63$. 

    \item \textbf{Based on the variables obtained in (a), make a guess for a formula for $a_n$ for every positive integer $n$ and use induction to verify that your guess is correct.}

    My best guess for the formula for this equation is $a_n = 2^{n + 1} - 1}, \forall n \in \mathbb{N}$.

    \n
    Base Case: First, we must prove for the $n = 1$ case.  We know that $a_1 = 3$ from above, and $2^{n + 1} - 1 = 2^{1 + 1} - 1 = 3$, 
    so the base case holds.

    \n
    Inductive Step: Let's assume for some integer $k \geq 2$, that $a_k = 2a_{k-1} + 1 = 2^{k + 1} - 1$.  We show that 
    $a_{k + 1} = 2a_k + 1 = 2^{k + 2} - 1$.

    \n
    By the inductive hypothesis, we know that $a_{k + 1} = 2a_k + 1 = 2(2^{k + 1} - 1) + 1$, so $2(2^{k + 1} - 1) + 1 = 
    2 \cdot 2^{k + 1} - 2 + 1 = 2^{k + 2} - 1$, therefore the claim holds for the inductive step as well.

    \n
    By the principle of mathematical induction, the claim $a_n = 2a_{n-1} + 1 = 2^{n + 1} - 1$ holds for all integers $n \geq 2$.
\end{enumerate}



\section{4.3 Problem 22}
\textbf{Use induction to show the following for Fibonacci numbers: $F_2 + F_4 + \cdots + F_{2n} = F_{2n+1} - 1$ for every positive integer $n$.}

\n Base Case: Assume $n = 1$, we know $F_{2 \cdot 1} = 1$.  Also $F_{2 \cdot 1 + 1} - 1 = F_3 - 1 = 1$, therefore the claim holds for the base case.

\n Inductive Step: Now let's assume for some integer $k \geq 1$, that $F_2 + F_4 + \cdots + F_{2k} = F_{2k+1} - 1$.  We show that
$F_2 + F_4 + \cdots + F_{2k} + F_{2k + 2} = F_{2k+3} - 1$

\begin{align*}
    && F_2 + F_4 + \cdots + F_{2k} + F_{2k + 2} &= F_{2k+1} - 1 + F_{2k + 2} && \text{(by the inductive hypothesis)} \\
    && &= F_{2k+1} + F_{2k + 2} - 1 && \\
    && &= F_{2k+3} - 1 && \text{(by the definintion of the Fibonacci Sequence)} \\
\end{align*}

Therefore the claim holds for the inductive step.  Hence, by the principle of mathematical induction, the claim
$F_2 + F_4 + \cdots + F_{2n} = F_{2n+1} - 1$ holds for all intergers $n \geq 1$.




\end{document}
