\documentclass[10pt]{article}

\title{CSCI 2011 HW 8}
\author{Fletcher Gornick}

\usepackage{amsmath,amssymb}
\usepackage{enumitem}
\usepackage{colortbl}
\usepackage{soul}
\usepackage{xcolor}
\usepackage[margin=1.0in]{geometry}

\setlength{\parindent}{0cm}

\begin{document}
\maketitle

\section{6.2 Problem 8}
\textbf{For the function $f$ defined by $f(n) = \frac{n^2+1}{n+1}$ for each $n \in \mathbb{N}$, show that $f(n) = O(n)$.} \\

for $n \geq 1$,
$f(n) = \frac{n^2+1}{n+1} \leq \frac{n^2+n}{n+1} < \frac{n^2+n}{n} = n + 1 \leq n + n = 2n$.  Therefore $f(n) < 2n$ for $n \geq 1$, and so $f(n) = O(n)$.


\section{Chapter 6 Problem 12}
\textbf{Let $f: \mathbb{N} \to \mathbb{R}^+$ and $g: \mathbb{N} \to \mathbb{R}^+$ be defined by $f(n) = 2n^3 + n + 10$ and $g(n) = n^3 + 4n^2 + 1$
for $n \in \mathbb{N}$.  Show that $f = \Theta(g)$}. \\

First we show that $f = O(g)$. \\
for $n \geq 1$, $f(n) = 2n^3 + n + 10 \leq 2n^3 + n^2 + 10 < 10n^3 + 40n^2 + 10 = 10(n^3 + 4n^2 + 1)$.  Therefore $f(n) < 10 \cdot g(n)$ for $n \geq 1$, and so
$f = O(g)$. \\

Now we show that $f = \Omega(g)$ or, put more simply, $g = O(f)$. \\
for $n \geq 1$, $g(n) = n^3 + 4n^2 + 1 \leq n^3 + 4n^3 + 1 = 5n^3 + 1 < 10n^3 + 50 < 10n^3 + 5n + 50 = 5(2n^3 + n + 10)$.  Therefore $g(n) < 5 \cdot f(n)$ for
$n \geq 1$, and so $g = O(f)$. \\

Since $f = O(g)$ and $g = O(f)$, it must be the case that $f = \Theta(g)$


\section{6.2 Problem 14}
\textbf{Let $f: \mathbb{N} \to \mathbb{R}^+$, $g: \mathbb{N} \to \mathbb{R}^+$ and $h: \mathbb{N} \to \mathbb{R}^+$ be three functions.  Prove that
if $f = \Theta(g)$ and $g = \Theta(h)$, then $f = \Theta(h)$.} \\

For $n,k \in \mathbb{Z}$, $n \geq k$, $f = \Theta(g)$, so there must exist $c_1, c_2 \in \mathbb{N}$, such that $c_1g(n) \leq f(n) \leq c_2g(n)$. \\
For $n,k \in \mathbb{Z}$, $n \geq k$, $g = \Theta(h)$, so there must exist $d_1, d_2 \in \mathbb{N}$, such that $d_1h(n) \leq g(n) \leq d_2h(n)$.


\section{7.1 Problem 16}
\textbf{Prove that $4 \mid (3^{2n-1} + 1)$ for every positive integer $n$.}


\section{7.2 Problem 8}
\textbf{Prove that every prime except one has the form $a^2 - b^2$ for some positive integers $a$ and $b$.}

\end{document}
