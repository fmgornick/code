\documentclass[10pt]{article}

\title{CSCI 2011 HW 7}
\author{Fletcher Gornick}

\usepackage{amsmath,amssymb}
\usepackage{enumitem}
\usepackage{colortbl}
\usepackage{soul}
\usepackage{xcolor}
\usepackage[margin=1.0in]{geometry}

\setlength{\parindent}{0cm}

\begin{document}
\maketitle

\section{Chapter 5.3 Problem 28}
\textbf{Let $A = \{1,2,3\}$, $B = \{1,2,3,4,5\}$ and $C = \{1,2,3,4\}$.  Also let $f: A \to B$ and $g: B \to C$,
where $f = \{(1,4), (2,5), (3,1)\}$ and $g = \{(1,3), (2,3), (3,2), (4,4), (5,1)\}$,}

\begin{enumerate}[label=(\alph*)]

    \item \textbf{Determine $(g \circ f)(1)$, $(g \circ f)(2)$ and $(g \circ f)(3)$.} \\
        $(g \circ f)(1) = 4$ because $f(1) = 4$ and $g(4) =$ \hl{$4$}. \\
        $(g \circ f)(2) = 1$ because $f(2) = 5$ and $g(5) =$ \hl{$1$}. \\
        $(g \circ f)(3) = 3$ because $f(3) = 1$ and $g(1) =$ \hl{$3$}. 

    \item \textbf{Determine $g \circ f$.} \\
        Since in the previous example, we found all possible values of $g \circ f$, we know \hl{$g \circ f = \{(1,4), (2,1), (3,3)\}$.}

\end{enumerate}



\section{Chapter 5.4 Problem 24}
\textbf{Prove or disprove each of the following.}

\begin{enumerate}[label=(\alph*)]

    \item \textbf{There exists functions $f: A \to B$ and $g: B \to C$ such that $f$ is not one-to-one and $g \circ f: A \to C$
        is one-to-one.}

        Suppose $g \circ f$ is injective, and we want to show that $f$ is not.  There must exist elemtents $a$ and $b$ such that $a \not= b$ but 
        $f(a) = f(b)$ in order for $f$ to not be injective.  Therefore $g(f(a)) = g(f(b))$ because $f(a) = f(b)$.  Since $g(f(x)) = (g \circ f)(x)$, 
        this means that $(g \circ f)(a) = (g \circ f)(b)$, meaning that $g \circ f$ is not injective, which contradicts our supposition.  Therefore, 
        by contradictive proof, if $f$ is not one-to-one, it cannot be the case that $g \circ f$ is.

    \item \textbf{There exists functions $f: A \to B$ and $g: B \to C$ such that $f$ is not onto and $g \circ f: A \to C$ is onto.}

        Suppose $A = \{a\}$, $B = \{b,c\}$ and $C = \{d\}$.  We can also assume $f(a) = b$ and $g(b) = d$.  Therefore $f$ is not onto, because you cannot
        link element $c$ in set $B$ to any element in set $A$ through $f$.  But we do know that $g \circ f$ is onto, because $(g \circ f)(a) = g(f(a)) = g(b) = d$,
        and there's only one element in $C$ that links to $a \in A$.  Therefore, by proof of existence, there exists functions $f$ and $g$, such that 
        $g \circ f$ is onto but $f$ is not.

\end{enumerate}



\section{Chapter 5.5 Problem 12}
\textbf{Prove or disprove: The set $S = \{(a,b): a,b \in \mathbb{R}\}$ of all points in the plane is uncountable.} \\

we can take a subset of $S$ by making $b$ constant and leaving $a$ as an element in $\mathbb{R}$.  So we have a set $A$ such that  $A \subseteq S$, and 
$A = \{(a,0): a \in \mathbb{R}\}$.  We can now create a bijective function $f: \mathbb{R} \to A$, where $f(x) = (x,0), \forall x \in \mathbb{R}$.  We know this 
function is bijective because for every distinct value of $x$, we have a distinct $f(x)$ (therefore it's onto), and we know that every value in the co-domain of 
$f$ can be mapped to it's domain ($(x,0) \to x$, so it's onto as well).  Since A has the same cardinality of $\mathbb{R}$, and the set of real numbers is uncountable,
we know that $S$ is uncountable because $|A| = |\mathbb{R}|$ and $A \subseteq S$.


\section{Chapter 5 Problem 32}
\textbf{Prove that the function $f: \mathbb{R} - \{3\} \to \mathbb{R} - \{1\}$ defined by $f(x) = \frac{x}{x-3}$ is bijective.} \\

First, we must show the function is one-to-one. \\
Suppose there exists two numbers $a,b \in \mathbb{R}$ and $a,b \not= 3$, such that $f(a) = f(b)$, therefore $\frac{a}{a-3} = \frac{b}{b-3}$, which means
$ab - 3a = ab - 3b \Rightarrow 3a = 3b \Rightarrow a = b$.  This means that $f$ is one-to-one. \\

Next, we show the function is onto. \\
Suppose $y = f(x)$ and $y \not= 1$ as stated in the definition.  therefore y = $\frac{x}{x-3}$, so we can manipulate this equation to find a function mapping the
co-domain to the domain. \\
$y = \frac{x}{x-3} \Rightarrow yx - 3y = x \Rightarrow yx - x = 3y \Rightarrow x(y - 1) = 3y \Rightarrow x = \frac{3y}{y-1}$, and since $y \not= 1$, $f$ is onto. \\

Since the function is both one-to-one and onto, $f$ is bijective.


\section{Chapter 5 Problem 40}
\textbf{Determine, with explanation, whether the following is true or false.  If $A$ and $B$ are disjoint sets such that $A$ is countable
and $B$ is uncountable, then $A \cup B$ is uncountable.} \\

Since $B$ is uncountable, and $B \subseteq A \cup B$, $A \cup B$ must also be uncountable, because by theorem 5.81 from the textbook, every set that contains an
uncountable set is itself uncountable.

\end{document}
