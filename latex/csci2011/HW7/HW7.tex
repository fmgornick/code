\documentclass[10pt]{article}

\title{CSCI 2011 HW 5}
\author{Fletcher Gornick}

\usepackage{amsmath,amssymb}
\usepackage{enumitem}
\usepackage{colortbl}
\usepackage{soul}
\usepackage{xcolor}
\usepackage[margin=1.0in]{geometry}

\setlength{\parindent}{0cm}

\begin{document}
\maketitle

\section{Chapter 5.3 Problem 28}
\textbf{Let $A = \{1,2,3\}$, $B = \{1,2,3,4,5\}$ and $C = \{1,2,3,4\}$.  Also let $f: A \to B$ and $g: B \to C$,
where $f = \{(1,4), (2,5), (3,1)\}$ and $g = \{(1,3), (2,3), (3,2), (4,4), (5,1)\}$,}

\begin{enumerate}[label=(\alph*)]

    \item \textbf{Determine $(g \circ f)(1)$, $(g \circ f)(2)$ and $(g \circ f)(3)$.} \\
        $(g \circ f)(1) = 4$ because $f(1) = 4$ and $g(4) =$ \hl{$4$}. \\
        $(g \circ f)(2) = 1$ because $f(2) = 5$ and $g(5) =$ \hl{$1$}. \\
        $(g \circ f)(3) = 3$ because $f(3) = 1$ and $g(1) =$ \hl{$3$}. 

    \item \textbf{Determine $g \circ f$.} \\
        Since in the previous example, we found all possible values of $g \circ f$, we know \hl{$g \circ f = \{(1,4), (2,1), (3,3)\}$.}

\end{enumerate}



\section{Chapter 5.4 Problem 24}
\textbf{Prove or disprove each of the following.}

\begin{enumerate}[label=(\alph*)]

    \item \textbf{There exists functions $f: A \to B$ and $g: B \to C$ such that $f$ is not one-to-one and $g \circ f: A \to C$
        is one-to-one.}

        Suppose $g \circ f$ is injective, and we want to show that $f$ is not.  There must exist elemtents $a$ and $b$ such that $a \not= b$ but 
        $f(a) = f(b)$ in order for $f$ to not be injective.  Therefore $g(f(a)) = g(f(b))$ because $f(a) = f(b)$.  Since $g(f(x)) = (g \circ f)(x)$, 
        this means that $(g \circ f)(a) = (g \circ f)(b)$, meaning that $g \circ f$ is not injective, which contradicts our supposition.  Therefore, 
        by contradictive proof, if $f$ is not one-to-one, it cannot be the case that $g \circ f$ is.

    \item \textbf{There exists functions $f: A \to B$ and $g: B \to C$ such that $f$ is not onto and $g \circ f: A \to C$ is onto.}

        Suppose $A = \{a\}$, $B = \{b,c\}$ and $C = \{d\}$.  We can also assume $f(a) = b$ and $g(b) = d$.  Therefore $f$ is not onto, because you cannot
        link element $c$ in set $B$ to any element in set $A$ through $f$.  But we do know that $g \circ f$ is onto, because $(g \circ f)(a) = g(f(a)) = g(b) = d$,
        and there's only one element in $C$ that links to $a \in A$.  Therefore, by proof of existence, there exists functions $f$ and $g$, such that 
        $g \circ f$ is onto but $f$ is not.

\end{enumerate}



\section{Chapter 5.5 Problem 12}
\textbf{Prove or disprove: The set $S = \{(a,b): a,b \in \mathbb{R}\}$ of all points in the plane is uncountable.} \\

In this case, $S$ is the set of $\mathbb{R} \times \mathbb{R}$.  Say we have two sets, $A$ and $B$, which are both the sets of real numbers, so 
$\mathbb{R} \times \mathbb{R} = A \times B$.  Now let's suppose the cartesian product of the two sets is countable, therefore
$S: A \times B \to \mathbb{N}$.  This means that for some $a \in A$, we have a set $C = \{(a,b)\}: b \in B\}$, so we have that $C \subseteq S = A \times B$



\section{Chapter 5 Problem 32}
\textbf{Prove that the function $f: \mathbb{R} - \{3\} \to \mathbb{R} - \{1\}$ defined by $f(x) = \frac{x}{x-3}$ is bijective.}



\section{Chapter 5 Problem 40}
\textbf{Determine, with explanation, whether the following is true or false.  If $A$ and $B$ are disjoint sets such that $A$ is countable
and $B$ is uncountable, then $A \cup B$ is uncountable.}

\end{document}
