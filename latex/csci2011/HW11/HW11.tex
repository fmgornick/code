\documentclass[10pt]{article}

\title{CSCI 2011 HW 11}
\author{Fletcher Gornick}

\usepackage{amsmath,amssymb}
\usepackage{enumitem}
\usepackage{colortbl}
\usepackage{soul}
\usepackage{xcolor}
\usepackage[margin=1.0in]{geometry}

\setlength{\parindent}{0cm}

\begin{document}
\maketitle

\section{Chapter 9.1 Problem 16}
\textbf{What is the exact term in $(\sqrt{x} - \frac{1}{\sqrt{x}})^{10}$ containing $x$?} \\
We want the term where we get $x^1$, so we have $(\sqrt{x})^{10-r}(-\frac{1}{\sqrt{x}})^r = cx$ for some $r,c \in \mathbb{R}$.  In this
case, $(\sqrt{x})^6 = x^3$ and $(-\frac{1}{\sqrt{x}})^4 = \frac{1}{x^2}$, so $(\sqrt{x})^6 \cdot (-\frac{1}{\sqrt{x}})^4 = x$, therefore $r = 4$.
We know that the coefficient $c$ is equal to $\binom{n}{r}$, which is $\binom{10}{4} = 210$.  This means that our term containing $x$ is $210x$.

\section{Chapter 9.2 Problem 20}
\textbf{A coin collector has 10 identical silver coins and 6 identical gold coins.  In how many ways can he place these 16 coins in a
    row if no two gold coins are placed next to each other?} \\
Since no gold coins can touch, we can create a template for available spaces for the extra silver coins...
$$\_\_\ GS\ \_\_\ GS\ \_\_\ GS\ \_\_\ GS\ \_\_\ GS\ \_\_\ G\ \_\_$$
we can represent each gold silver pair (and the one single gold at the end) as delimeters for our extra 5 silver pieces.  Since there are 6
delimeters and 5 silver coins, there are 11 total, giving us $\binom{11}{5} = 462$ possible orientations.

\section{Chapter 10.1 Problem 18}
\textbf{During the last class period of the semester, each student in a graduate computer science class with 10 students is required 
    to give a brief report on his or her class project. The professor randomly selects the order in which the reports are to be given. 
    Two students have been working on similar projects and would like to give their reports consecutively. What is the probability that 
    this will happen?} \\\\
the probability of them presenting consecutively is equal to the number of outcomes where they are constecutive divided by the total number of
outcomes.  Since there are 10 presenting times, there are $10! = 3628800$ possible outcomes.  If person 1 and person 2 are paired together, then
we can treat them as one single grouping giving us $9! = 362880$ different outcomes, but this only covers half.  If it must be person 1 then 
person 2, then there are 362880 probabilities, but there's also the case where it's person 2 followed by person 1, which gives us another 362880
probabilities.  Therefore, the probability of them going consecutively is $\frac{2!9!}{10!} = \frac{725760}{3628800} = \frac15$.
\\\\\\\\

\section{Chapter 10.2 Problem 24}
\textbf{A student has an exam in a discrete mathematics class tomorrow. The exam is divided into two parts. The first part consists 
    only of true-false questions and the second part consists only of multiple choice questions. She estimates that the probability of 
    getting B or better on Part 1 is 0.8 and the probability of getting B or better on Part 2 is 0.6. She also estimates that the 
    probability of getting B or better on both parts is 0.5.} \\\\
    probability of getting a B or better on part 1 = $P(B1) = 0.8$ \\
    probability of getting a B or better on part 2 = $P(B2) = 0.6$ \\
    probability of getting a B or better on part 1 and part 2 = $P(B1 \cap B2) = 0.5$ 
\begin{enumerate}[label=(\alph*)]
    \item \textbf{What is the probability of getting B or better on Part 1 if she got a B or better on Part 2?} \\
        $P(B1 \vert B2) = \frac{P(B1 \cap B2)}{P(B2)} = \frac{0.5}{0.6} = \frac56$.

    \item \textbf{Are the events ``getting B or better on Part 1'' and ``getting B or better on Part 2'' independent?} \\
        If $P(B1)$ and $P(B2)$ were independent, then $P(B1) = P(B1 \vert B2)$, but $P(B1) = 0.8$, and $P(B1 \vert B2) = 0.83$.  Since
        $P(B1) \not= P(B1 \vert B2)$, the two events are NOT independent.
\end{enumerate}

\section{Chapter 10 Problem 20}
\textbf{A bowl contains 3 red balls, 2 white balls and one blue ball.}

\begin{enumerate}[label=(\alph*)]
    \item \textbf{What is the expected number of white balls obtained if three balls are selected at random from the bowl?} \\
        we can describe a random variable $X$ as the number of white balls obtained.  Since we are drawing three balls, and there are two
        white balls, the possible values of $X$ are 0, 1, or 2.  We must first solve for the probabilities of each of these before finding the
        expected value. \\

        $P(X=0) = \frac{\binom33}{\binom63} + \frac{\binom32\binom11}{\binom63} = \frac15$ \\
        $P(X=2) = \frac{\binom22\binom31}{\binom63} + \frac{\binom22\binom11}{\binom63} = \frac15$ \\
        $P(X=1) = 1 - P(X=0) - P(X=2) = 1 - \frac15 - \frac15 = \frac35$ \\

        Therefore, our expected value is $0 \cdot \frac15 + 1 \cdot \frac35 + 2 \cdot \frac15 = 1$.

    \item \textbf{What is the expected number of white balls obtained if three balls are selected at random from the bowl, one at a 
        time, where a ball is returned to the bowl after it is selected?} \\
        In this case, we can create a new random variable Y, with possible values 0, 1, 2, and 3.  Again, we must first calculate the probabilities
        of each possible outcome. 
        \begin{align*}
            P(X=0) = \Big(\frac46\Big)^3 = \frac{8}{27} && P(X=1) = 3 \cdot \Big(\frac26\Big) \cdot \Big(\frac46\Big)^2 = \frac49 \\
            P(X=2) = 3 \cdot \Big(\frac26\Big)^2 \cdot \Big(\frac46\Big) = \frac29 && P(X=3) = \Big(\frac26\Big)^3 = \frac{1}{27}
        \end{align*}
        Therefore, our expected value is $0 \cdot \frac{8}{27} + 1 \cdot \frac49 + 2 \cdot \frac29 + 3 \cdot \frac{1}{27} = 1$.
\end{enumerate}

\end{document}
