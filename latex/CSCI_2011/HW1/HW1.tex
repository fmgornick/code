\documentclass[10pt]{article}

\title{CSCI 2011 HW 1}
\author{Fletcher Gornick}

\usepackage{amsmath,amssymb}
\usepackage{enumitem}
\usepackage[margin=1.0in]{geometry}

\begin{document}
\maketitle

\section{1.2 Problem 18}
\textbf{Let \textit{P}, \textit{Q} and \textit{R} be statements. Determine whether
the following is true.} $$P \oplus (Q \oplus R) \equiv (P \oplus Q) \oplus R$$



\section{1.3 Problem 18}
\textbf{The \underline{inverse} of the implication of $P \Rightarrow Q$ is the
implication $(\sim P) \Rightarrow (\sim Q)$.}

\begin{enumerate}[label=(\alph*)]

    \item \textbf{Use a truth table to verify that $P \Rightarrow Q \not\equiv 
        (\sim P) \Rightarrow (\sim Q)$.}

    \item \textbf{Find another implication that is logically equivalent to
        $(\sim P) \Rightarrow (\sim Q)$ and verify your answer.}

\end{enumerate}



\section{1.3 Problem 25}
\textbf{For two statements \textit{P} and \textit{Q}, use truth tables to verify
the following.}

\begin{enumerate}[label=(\alph*)]

    \item \textbf{$P \vee Q \equiv (\sim P) \Rightarrow Q$.}

    \item \textbf{$P \wedge Q \equiv \sim (P \Rightarrow (\sim Q))$.}
        
    \item \textbf{$\sim (P \Rightarrow Q) \equiv P \wedge (\sim Q)$.}

\end{enumerate}



\section{1.4 Problem 12}
\textbf{For every two statements \textit{P} and \textit{Q}, use truth tables to
verify the following.}

\begin{enumerate}[label=(\alph*)]

    \item \textbf{$P \Leftrightarrow Q \equiv (\sim P) \Leftrightarrow (\sim Q)$.}

    \item \textbf{$P \Leftrightarrow Q \equiv (P \wedge Q) \vee ((\sim P) \wedge
        (\sim Q))$.}

    \item \textbf{$\sim (P \Leftrightarrow Q) \equiv P \Leftrightarrow (\sim Q)$.}

\end{enumerate}



\section{1.5 Problem 10}
\textbf{Let \textit{S} and \textit{R} be two compound statements with the same
component statements. If \textit{S} is a tautology and \textit{R} is a contradiction,
then what is the truth value of the following?}

\begin{enumerate}[label=(\alph*)]
    
    \item \textbf{$S \vee R$}

    \item \textbf{$S \wedge R$}

    \item \textbf{$S \Rightarrow R$}

    \item \textbf{$R \Rightarrow S$}

    \item \textbf{$S \Leftrightarrow R$}

\end{enumerate}


\end{document}
