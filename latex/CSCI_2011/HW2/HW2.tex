\documentclass[10pt]{article}

\title{CSCI 2011 HW 2}
\author{Fletcher Gornick}

\usepackage{amsmath,amssymb}
\usepackage{enumitem}
\usepackage{colortbl}
\usepackage{soul}
\usepackage{xcolor}
\usepackage[margin=1.0in]{geometry}

\def \n {\par \vspace{\baselineskip}}

\begin{document}
\maketitle

\section{2.1 Problem 26}
\textbf{For $n \in \mathbb{N}$, two sets $A$ and $C$ have the property that $A \subset C$, 
$|A| = n$, and $|C| = n+3$. How many sets $B$ are there such that $A \subset B \subset C$?}

\n

$|C|-|A| = 3$, so $C$ has three more elements than $A$. $B \supset A$, so $B$ must have
more elements than $A$, $B$ can not have the same number of elements as $A$ because 
$B \not\supseteq A$. $B \subset C$, so $B$ must have less elements than $C$. Again, they
cannot be equal because $B \not\subseteq C$.  This means that B must have one or two more
elements than A that are also in C.

\n

ex: $A = \{a\}, C = \{a,b,c,d\}$

possible sets for B: $\{a,b\}, \{a,c\}, \{a,d\}, \{a,b,c\}, \{a,b,d\}, \{a,c,d\}$

\hl{Number of possible sets for B: 6}

\n

$C \cap \bar A = \{b,c,d\}$. So B includes any combination of those three values that 
sums up to one or two total values, giving 6 possible combinations...
$\big\{\{b\},\{c\},\{d\},\{b,c\},\{b,d\},\{c,d\}\big\}$



\section{2.2 Problem 22}
\textbf{For sets $A$ and $B$ of integers, define $A+B = \{a+b : a \in A, b \in B\}$.
If $|A| = |B| = 5$, how small and how large can $|A + B|$ be?}

\par \vspace{\baselineskip}

The largest possible value for $|A+B|$ comes when each element of $A$ and $B$ add up 
to a different sum.

\par \vspace{\baselineskip}

ex: $A = \{a,b,c,d,e\}, B = \{v,w,x,y,z\}$

$A + B = \{a+v, a+w, a+x, a+y, a+z, b+v, b+w, b+x, b+y, b+z, c+v, c+w, c+x, c+y, c+z,
d+v, d+w, d+x, d+y, d+z, e+v, e+w, e+x, e+y, e+z\}$

\hl{$|A + B| = 25$}

\par \vspace{\baselineskip}

The smallest possible value for $|A + B|$ comes when almost each element of $A$ and $B$ add
up to the same sum.

\par \vspace{\baselineskip}

ex: $A = \{0,1,2,3,4\}, B = \{0,1,2,3,4\}$

$A + B = \{0,1,2,3,4,5,6,7,8\}$

\hl{$|A + B| = 9$}

\n
\n
\n
\n
\n



\section{2.3 Problem 12}
\textbf{Verify, for sets $A$, $B$, and $C$, that $(A \times B) \cap (A \times C) = A \times (B \cap C)$.}

\begin{align*}
    && A \times (B \cap C) &= \{(x,y) : x \in A, y \in (B \cap C)\} && 
    \text{(definition of the product between sets)} \\
    && &= \{(x,y) : (x \in A) \wedge (y \in (B \wedge C))\} && \text{(comma and intercept = ``and'')} \\
    && &= \{(x,y) : (x \in A) \wedge (y \in B) \wedge (y \in C)\} && \text{(distributive property)} \\
    && &= \{(x,y) : ((x \in A) \wedge (y \in B)) \wedge ((x \in A) \wedge (y \in C))\} &&
    (X \wedge X \equiv X) \\
    && &= \{(x,y) : ((x,y) \in A \times B) , ((x,y) \in A \times C)\} && 
    \text{(``and'' = comma)} \\
    && &= (A \times B) \cap (A \times C) && \text{(definition of the product between sets)} \\
\end{align*}



\section{2.3 Problem 14}
\textbf{For two sets $A$ and $B$ of real numbers, the set $A \cdot B$ is defined by
$$A \cdot B = \{ab : a \in A, b \in B\}.$$ Determine each of the following sets.}

\begin{enumerate}[label=(\alph*)]

    \item \textbf{$A \cdot B$ for $A = \{\frac{1}{2}, 1, \sqrt{2}\}$ and $B = \{\sqrt{2},2,4\}$.}

        $\{\frac{\sqrt{2}}{2},1,2,\sqrt{2},2,4,2,2\sqrt{2},4\sqrt{2}\} \Rightarrow$
        \hl{$\{\frac{\sqrt{2}}{2}, 1, \sqrt{2}, 2, 2\sqrt{2}, 4\sqrt{2}\}$}

    \item \textbf{$\mathbb{R} \cdot \mathbb{R}$.}

        any real number multiplyed by another real number is just a real number, so
        \hl{$\mathbb{R} \cdot \mathbb{R} = \mathbb{R}$}

    \item \textbf{$\mathbb{R} \cdot C$ where $C \subseteq \mathbb{R}$ with $|C| = 2$.}

        $C$ contains two elements that are both real numbers. Assuming that one of it's
        elements is zero, the other can be any non-zero number. For any non-zero real
        number, you can multiply it by another real number in the $\mathbb{R}$ set to 
        produce any possible number in the $\mathbb{R}$ set. This means that...

        \hl{$\mathbb{R} \cdot C = \mathbb{R}$}

\end{enumerate}



\section{2.4 Problem 10}
\textbf{Let $A = \{1,2,3,4\}$. Partition the power set $\mathcal{P}(A)$ of $A$ into
as many subsets as possible such that no two subsets have the same number of elements.}

$\mathcal{P}(A) = \big\{\emptyset, \{1\}, \{2\}, \{3\}, \{4\}, \{1,2\}, \{1,3\}, \{1,4\}, 
\{2,3\}, \{2,4\}, \{3,4\}, \{1,2,3\}, \{1,2,4\}, \{1,3,4\}, \{2,3,4\}, \{1,2,3,4\}\big\}$

\par \vspace{\baselineskip}

$P =$ Partition of $\mathcal{P}(A)$

{\small
    \hl{
        $ P = 
        \Big\{\big\{\emptyset\big\}, 
        \big\{\{1\},\{2\}\big\}, 
        \big\{\{3\},\{4\},\{1,2\}\big\}, 
        \big\{\{1,3\},\{1,4\},\{2,3\},\{2,4\}\big\}, 
        \big\{\{3,4\},\{1,2,3\},\{1,2,4\},\{1,3,4\},
        \{2,3,4\},\{1,2,3,4\}\big\}\Big\} $
    }
}%

5 possible subsets.


\end{document}
