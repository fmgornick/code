\documentclass[11pt]{article}

\usepackage{setspace}
\usepackage{amsmath}
\usepackage{enumitem}
\usepackage{amsfonts} 
\usepackage{relsize}
\usepackage[top=2cm,bottom=2cm,left=2.5cm,right=2.5cm,marginparwidth=1.75cm]{geometry}
\setlength{\parindent}{0cm}

\renewcommand{\l}{\lambda}

\title{4512 Homework 2}
\author{Fletcher Gornick}
\date{October 11, 2021}
\spacing{1.5}

\begin{document}
 \maketitle 

 \section*{Section 2.2 Problem 6}
 \textbf{Solve the initial-value problem.}
 $$2\frac{d^2y}{dt^2} + \frac{dy}{dt} - 10y = 0 \;,\; y(1) = 5, y'(1) = 2$$

 We can rewrite $y$ as $e^{\l t}$ to give us the equation 
 $2\l^2 e^{\l t} + \l e^{\l t} - 10e^{\l t} = 0$, setting $t = 0$, we get 
 $2\l^2 + \l - 10 = (2\l + 5)(\l - 2) = 0$, therefore $\l = \frac{-5}{2}, 2$, giving us the 
 general equation for t, $y(t) = c_1e^{\frac{-5}{2}t} + c_2e^{2t}$, where $y(1) = 5, y'(1) = 2$.
 We can rewrite this in matrix notation to solve for $c_1$ and $c_2$ plugging 1 in for $t$.
\begin{align*}
&& \begin{bmatrix} e^{\frac{-5}{2}} & e^2 \\ \frac{-5}{2}e^{\frac{-5}{2}} & 2e^2 \end{bmatrix} 
\begin{bmatrix} c_1 \\ c_2 \end{bmatrix} &= \begin{bmatrix} y(1) \\ y'(1) \end{bmatrix} && \\
&& \Rightarrow \begin{bmatrix} c_1 \\ c_2 \end{bmatrix} &= 
\begin{bmatrix} e^{\frac{-5}{2}} & e^2 \\ \frac{-5}{2}e^{\frac{-5}{2}} & 2e^2 \end{bmatrix}^{-1}
\begin{bmatrix} 5 \\ 2 \end{bmatrix} && \\
&& \Rightarrow \begin{bmatrix} c_1 \\ c_2 \end{bmatrix} &= 
\frac{1}{2e^{-1/2} + \frac{5}{2}e^{-1/2}} 
\begin{bmatrix}
2e^2 & -e^2 \\ \frac{5}{2}e^{\frac{-5}{2}} & e^{\frac{-5}{2}} \end{bmatrix}
\begin{bmatrix} 5 \\ 2 \end{bmatrix} && \\
  && \Rightarrow c_1 &= \frac{2}{9}e^{\frac{1}{2}}\Big(10e^2 - 2e^2\Big) && \\
  && c_2 &= \frac{2}{9}e^{\frac{1}{2}}\Big(\frac{25}{2}e^{\frac{-5}{2}} + 2e^{\frac{-5}{2}}\Big)
\end{align*}
After simplifying, we get $c_1 = \frac{16}{9}e^{\frac{5}{2}}$, and $c_2 = \frac{29}{9}e^{-2}$.  
Finally, we can plug these values in to find the solution to the initial value problem. 
$$y(t) = \frac{1}{9}\Big(16e^{\frac{-5}{2}(t-1)}+29e^{2t-2}\Big)$$
\newpage

 \section*{Section 2.2.1 Problem 6}
 \textbf{Solve the initial-value problem.}
 $$\frac{d^2y}{dt^2} + 2\frac{dy}{dt} + 5y = 0 \;,\; y(0) = 0, y'(0) = 2$$
 Again, We can rewrite $y$ as $e^{\l t}$ to give us a new equation 
 $\l^2 e^{\l t} + 2\l e^{\l t} + 5e^{\l t} = 0$.  Setting $t=0$ gives us a new equation 
 $\l^2 + 2\l + 5 = 0$.  We can plug this equation into the quadratic formula to get 
 $\l = \frac{-2 \pm \sqrt{(-2)^2 - 4 \cdot 5}}{2} = -1 \pm 2i$, giving us the general 
 equation $y(t) = b_1e^{(-1+2i)t} + b_2e^{(-1-2i)t}$.  For the first complex root $\l_1 = -1+2i$, 
 we can apply Euler's formula to get $e^{\l_1 t} = e^{(-1+2i)t} = e^{-t}\cos(2t) + ie^{-t}\sin(2t)$ 
 which is a complex-valued solution to $y'' + 2y' + 5y = 0$. \\
 
 Since, by \textit{Lemma 1} from section 2.2.1 of 
 the textbook, both Re$\{e^{\l_1 t}\} = e^{-t}\cos(2t)$ and Im$\{e^{\l_1 t}\} = e^{-t}\sin(2t)$ 
 are two linearly independent, real-valued solutions to the above equation, we can rewrite the general
 solution to the initial-value problem as 
 $$y(t) = e^{-t}[c_1\cos(2t) + c_2\sin(2t)]$$
 Plugging $0$ in for $t$ we get that $y(0) = e^0 [c_1\cos(0) + c_2\sin(0)] = c_1 = 0$, which gives us 
 a new equation $y(t) = c_2e^{-t}\sin(2t)$.  We can now differentiate and plug in 0 to solve for $c_2$. 
 $$y'(t) = -c_2e^{-t}sin(2t) + 2c_2e^{-t}cos(2t) \;\Rightarrow\; y'(0) = 2c_2 = 2 \;\Rightarrow\; 
 c_2 = 1$$
 Now we can plug the values we found for our two constants to give us the solution to the initial 
 value problem. 
 $$\mathlarger{\mathlarger{y(t) = e^{-t}sin(2t)}}$$
 
 \newpage

 \section*{Section 2.2.2 Problem 6}
 \textbf{Solve the initial-value problem.}
 $$\frac{d^2y}{dt^2} + 2\frac{dy}{dt} + y = 0 \;,\; y(2) = 1, y'(2) = -1$$
 Again, applying $y = e^{\l t}$, we get $\l^2 e^{\l t} + 2\l e^{\l t} + e^{\l t} = 0$, and plugging 
 0 in for $t$, we get $\l^2 + 2\l + 1 = 0$, which can be simplified to $(\l + 1)^2 = 0$, giving us 
 $\l = -1$ as our only solution with a multiplicity of 2.  As stated in Section 2.2.2 of the textbook, 
 we can yield two linearly independent solutions to the equation by multiplying one of them by $t$.  
 This gives us the general solution $y(t) = c_1e^{-t} + c_2te^{-t}$.  We can now solve for the Wronskian 
 of the matrix at time $t=2$, and plug in the corresponding values $y(2)=1,y'(2)=-1$ like so...
\begin{align*}
&& \begin{bmatrix} e^{-2} & 2e^{-2} \\ -e^{-2} & e^{-2} - 2e^{-2} \end{bmatrix}
\begin{bmatrix} c_1 \\ c_2 \end{bmatrix} &= \begin{bmatrix} y(2) \\ y'(2) \end{bmatrix} && \\
&& \Rightarrow \begin{bmatrix} c_1 \\ c_2 \end{bmatrix} &=
\begin{bmatrix} e^{-2} & 2e^{-2} \\ -e^{-2} & -e^{-2} \end{bmatrix} ^{-1}
  \begin{bmatrix} 1 \\ -1 \end{bmatrix} && \\
    && \Rightarrow \begin{bmatrix} c_1 \\ c_2 \end{bmatrix} &= \frac{1}{-e^{-4} + 2e^{-4}}
\begin{bmatrix} -e^{-2} & -2e^{-2} \\ e^{-2} & e^{-2} \end{bmatrix}
\begin{bmatrix} 1 \\ -1 \end{bmatrix} && \\
  && \Rightarrow c_1 &= e^4(-e^{-2}+2e^{-2}) = e^2 && \\
  && \Rightarrow c_2 &= e^4(e^{-2}-e^{-2}) = 0 &&
\end{align*}
Now we can plug the values we found for $c_1$ and $c_2$ to give us our final solution.
$$\mathlarger{\mathlarger{y(t) = e^{2-t}}}$$
 \newpage

 \section*{Section 2.4 Problem 6}
 \textbf{Solve the initial-value problem.}
 $$y'' + 4y' + 4y = t^{5/2}e^{-2t} \;,\; y(0) = 0, y'(0) = 0$$
 We can write the general solution to this differential equation as $y(t) = c_1y_1(t) + c_2y_2(t) + y_p(t)$, 
 where $c_1y_1(t) + c_2y_2(t)$ is the general solution to $y'' + 4y' + 4y = 0$, otherwise known as the 
 complementary equation.  So we can solve for the complementary equation just like we would any normal 
 homogenous differential equation.
 $$y'' + 4y' + 4y = 0 \;\;\Rightarrow\;\; \l^2e^{\l t} + 4\l e^{\l t} + 4e^{\l t} = 0 
 \;\;\Rightarrow\;\; \l^2 + 4\l + 4 = 0 \;\;\Rightarrow\;\; \l = -2$$
 Since $\l = -2$ with a multiplicity of 2, we can now write our general solution to the initail value problem
 as $y(t) = c_1e^{-2t} + c_2te^{-2t} + y_p(t)$.  Now, to find a particular solution $y_p$ that satisfies the 
 differential equation.  Since the right side of the equation isn't an easy polynomial multiplied by an 
 exponential, we must use variation of parameters to solve for the particular solution.  Let's call the equation 
 on the right-hand side $r(x)$ for simplicity sake.  The solution to the particular equation $y_p$ can be written 
 as $y_p(t) = u_1(t)y_1(t) + u_2(t)y_2(t)$ where 
 $u_1'(t) = \frac{-r(t)y_2(t)}{W[y_1,y_2]}$, and $u_2'(t) = \frac{r(t)y_1(t)}{W[y_1,y_2]}$ as stated in section 
 2.4.  Thus, the formula for the particular solution can be written as 
 $$y_p(t) = -y_1(t) \int \frac{y_2(t)r(t)}{W[y_1,y_2]}dt + y_2(t) \int \frac{y_1(t)r(t)}{W[y_1,y_2]}dt$$
 $W[y_1,y_2]$ represents the Wronskian of the functions $y_1$ and $y_2$, so we can solve for $W$ like so...
 \begin{align*}
 W[y_1,y_2] \;=\; \begin{vmatrix} y_1(t) & y_2(t) \\ y_1'(t) & y_2'(t) \end{vmatrix}
   \;=\; \begin{vmatrix} e^{-2t} & te^{-2t} \\ -2e^{-2t} & (1-2t)e^{-2t} \end{vmatrix}
  \;=\; (1-2t)e^{-4t} + 2te^{-4t} \;=\; e^{-4t}
 \end{align*}

 Now we've attained every value needed to solve for our particular solution $y_p$.
 $$y_p(t) = -e^{-2t} \int \frac{te^{-2t} \cdot t^{5/2}e^{-2t}}{e^{-4t}}dt
 + te^{-2t} \int \frac{e^{-2t} \cdot t^{5/2}e^{-2t}}{e^{-4t}}dt$$

 On the next page, we will solve for the particular solution.
 \newpage
 \begin{align*}
   && y_p(t) &= -e^{-2t} \int \frac{te^{-2t} \cdot t^{5/2}e^{-2t}}{e^{-4t}}dt
   + te^{-2t} \int \frac{e^{-2t} \cdot t^{5/2}e^{-2t}}{e^{-4t}}dt && \\
   && &= -e^{-2t} \int \frac{t^{7/2}e^{-4t}}{e^{-4t}} dt 
   + te^{-2t} \int \frac{t^{5/2}e^{-4t}}{e^{-4t}} dt && \\
   && &= -e^{-2t} \int t^{7/2} dt + te^{-2t} \int t^{5/2} dt && \\
   && &= \frac{-2}{9}t^{9/2}e^{-2t} + \frac{2}{7}t^{9/2}e^{-2t} && \\
   && &= \frac{4}{63}t^{9/2}e^{-2t} &&
 \end{align*}
 So now we have our particular solution $y_p(t) = \frac{4}{63}t^{9/2}e^{-2t}$, so we can plug this into our 
 general equation to get $y(t) = c_1e^{-2t} + c_2te^{-2t} + \frac{4}{63}t^{9/2}e^{-2t}$.  The last thing we 
 need to do is solve for our constants $c_1$ and $c_2$.  We can use the same method from previous pages to 
 do this.
 \begin{align*}
&& \begin{bmatrix} e^{-2t} & te^{-2t} \\ -2e^{-2t} & (1-2t)e^{-2t} \end{bmatrix}
\begin{bmatrix} c_1 \\ c_2 \end{bmatrix} &= \begin{bmatrix} y(0) \\ y'(0) \end{bmatrix} && \\
&& \Rightarrow \begin{bmatrix} c_1 \\ c_2 \end{bmatrix} &= 
\begin{bmatrix} e^{-2t} & te^{-2t} \\ -2e^{-2t} & (1-2t)e^{-2t} \end{bmatrix} ^{-1}
\begin{bmatrix} 0 \\ 0 \end{bmatrix} && \\
&& \Rightarrow \begin{bmatrix} c_1 \\ c_2 \end{bmatrix} &= \frac{1}{e^{-4t}}
\begin{bmatrix} (1-2t)e^{-2t} & -te^{-2t} \\ 2e^{-2t} & e^{-2t} \end{bmatrix}
\begin{bmatrix} 0 \\ 0 \end{bmatrix} && \\
&& \Rightarrow c_1 &= 0 && \\ && c_2 &= 0 && 
 \end{align*}
 So finally, with our complementary equation and particular equation, we have that the solution to the 
 non-homogenous initial-value problem is
 $$\mathlarger{\mathlarger{y(t) = \frac{4}{63}t^{9/2}e^{-2t}}}$$
 \newpage

 \section*{Section 2.6 Problem 1}
 \textbf{It is found experimentally that a 1 kg mass stretches a spring 49/320 m. 
 If the mass is pulled down an additional 1/4 m and released, find the amplitude, 
 period, and frequency of the resulting motion, neglecting air resistance (use 
 g = 9.8 $m/s^2$)} \\

 We know by Newton's second law, $F = mg = 9.8 \cdot 1 = 9.8$.  Also from equation $(1)$ in section 2.6 
 of the textbook, $F(t) = m\frac{d^2y}{dt^2} + c\frac{dy}{dt} + ky$.  Since the spring is stretched 
 $49/320 = 0.153$ m, $y = 0.153$, and $\frac{d^2y}{dt^2} = \frac{dy}{dt} = 0$.  So now we have this 
 equation $F(t) = ky \;\Rightarrow\; 9.8 = 0.153k \;\Rightarrow\; k = 64$. \\

 Since there's no air resistance, our damping coefficient $c = 0$.  Also after the initial displacement 
 of the spring, there is no external force, so we have the equation $m\frac{d^2y}{dt^2} + ky = 0$.
 This equation can be rerwitten as $\frac{d^2y}{dt^2} + \omega_0^2 y = 0$, where $\omega_0^2 = k/m$.
 From section 2.6 equation $(3)$, we know this equation has the general form 
 $y(t) = c_1\cos(\omega_0 t) + c_2\sin(\omega_0 t)$. We can also solve for $\omega_0$, so we get 
 $\omega_0 = \sqrt{k/m} = \sqrt{64/1} = 8$, thus our general equation is now 
 $y(t) = c_1\cos(8t) + c_2\sin(8t)$ \\ 

 Before finding our amplitude, period, and frequency, we must first solve for constants $c_1$ and $c_2$. 
 This spring at time $t=0$ is initially displaced $0.25$m, so $0.25 = c_1\cos(0) + c_2\sin(0) \Rightarrow 
 c_1 = 0.25$.  Also the initial velocity of the spring at time $t=0$ is $0$, so $0 = -8c_1\sin(0) + 
 8c_2\cos(0) \Rightarrow c_2 = 0$.  This gives us our final equation $y(t) = 0.25\cos(8t)$. \\ 

 By Lemma 1 in section 2.6 of the textbook, we know our previous function $y(t)$ can be rewritten as 
 $y(t) = R\cos(\omega_0 + \delta)$, where $R = \sqrt{c_1^2 + c_2^2} = \frac{1}{4}$, and 
 $\delta = \arctan(\frac{c_2}{c_1}) = \arctan(0) = 0$.  In this case, $R$ represents the amplitude and 
 $\omega_0$ represents the freqency, and thus the period $T = 2\pi / \omega_0$. \\

 Therefore, we have an amplitude $A = R = \frac{1}{4}$, a frequency $f = \omega_0 = 8$, and a period 
 $T = \frac{2\pi}{\omega_0} = \frac{\pi}{4}$.

\end{document} 
