\documentclass[11pt]{article}

\usepackage{setspace}
\usepackage{amsmath}
\usepackage{enumitem}
\usepackage{amsfonts} 
\usepackage{relsize}
\usepackage[top=2cm,bottom=2cm,left=2.5cm,right=2.5cm,marginparwidth=1.75cm]{geometry}
\setlength{\parindent}{0cm}

\title{4512 Homework 1}
\author{Fletcher Gornick}
\date{September 16, 2021}

\spacing{1.5}
\begin{document}
 \maketitle 

 \section*{Section 1.4 Question 3}
 Find the general solution: $\frac{dy}{dt} = 1 - t + y^2 - ty^2$.
 $$\frac{dy}{dt} = 1 - t + y^2 - ty^2 \Rightarrow
 \frac{dy}{dt} = (y^2+1)(1-t) \Rightarrow
 \int \frac{dy}{1+y^2} = \int (1-t) \; dt \Rightarrow
 \arctan(y) = -\frac{1}{2}t^2+t+C$$

 so $y = tan(-\frac{1}{2}t^2+t+C)$ in the interval 
 $-k \frac{\pi}{2} < -\frac{1}{2}t^2+t+C < k \frac{\pi}{2}$
 for some $k \in \mathbb{N}$.

 \section*{Section 1.5 Question 12}
 given $\frac{dp}{dt} = bp^2 - ap, \quad a,b > 0$, show that 
 $p(t)$ approaches $0$ as $t \rightarrow \infty$ if $p_0 < a/b$.

 First thing to do is separate this equation...
$$\frac{dp}{dt}=bp^2-ap \; \Rightarrow \; \int_{p_0}^{p} \frac{dp}{bp^2-ap} = 
\int_{t_0}^{t} dt$$

Now to solve the first integral, we can first note that $bp^2-ap$ can be rewritten as
$p(bp-a)$, then we can separate these by finding values $A$ and $B$ such that
$\frac{1}{p(bp-a)} = \frac{A}{p} + \frac{B}{bp-a}$...
$$A(bp-a)+Bp = 1 \quad \Rightarrow \quad A = \frac{-1}{a}, \; B = \frac{b}{a}$$

Now we can plug these values in and solve our integral...
$$\frac{1}{a} \int_{p_0}^{p} \frac{-1}{p} + \frac{b}{bp-a} \; dp \;\; = \;\;
\frac{1}{a}\ln{\left|\frac{p_0(bp-a)}{p(bp_0-a)}\right|} \;=\; t-t_0 \;\; \Rightarrow
\;\; \frac{p_0(bp-a)}{p(bp_0-a)} \; = \; e^{a(t-t_0)} $$

After simplifying, we get 
$\mathlarger{p(t) \;=\; \frac{ap_0}{bp_0 + (a-bp_0)e^{a(t-t_0)}}}$.  As you can see,
as $t \rightarrow \infty$, the denominator gets larger, and $p(t)$ approaches 0.

 \section*{Section 1.8 Question 13}
 Find the orthogonal trajectory: $y^2 - x^2 = c$.
 $$F(x,y,c) = y^2-x^2-c=0 \;\; \Rightarrow \;\;
 \frac{\partial}{\partial x}F(x,y,c) = 2y \frac{dy}{dx} - 2x = 0 \;\; \Rightarrow \;\;
 \frac{dy}{dx} = \frac{2x}{2y} = \frac{x}{y}$$

 The orthogonal slope is the oppposite reciprocal of the original slope, so 
 $\mathlarger{\frac{dy}{dx} = \frac{-y}{x}}$.  Now we can separate and integrate to solve for $y$.
 $$\frac{dy}{dx} = \frac{-y}{x} \;\; \Rightarrow \;\;
 - \int \frac{dy}{y} \; = \; \int \frac{dx}{x} \;\; \Rightarrow \;\;
 - \ln{|y|} \; = \; \ln{|x|} + k_1 \;\; \Rightarrow \;\;
 \frac{1}{y} \; = \; k_2x \;\; \Rightarrow \;\;
 y(x) \; = \; \frac{k}{x}$$

 \section*{Section 1.8 Question 18}
 $y(t)$ = number of living bacteria at time $t$. \\

 $T(t)$ = number of toxins at time $t$. \\
 Also, production of toxins begin at time $t=0$, so $T(0) = 0$. \\

 Finally, $\frac{dT}{dt} = c$.

 \begin{enumerate}[label=(\alph*)]

   \item \textbf{Find a first-order differential equation satisfied by $y(t)$}

     $\frac{dy}{dt} = $ (bacteria birth) - (bacteria death). \\
     Bacteria grows proportionally to the amount present $y$ with a proportionality 
     constant $b$. \\
     Bacteria dies proportionally to both the amount present $y$ and the number of 
     toxins present $T$.  The proportionality constant for bacteria death is $a$. \\
     Therefore, $\frac{dy}{dt} \propto y - yT = by - ayT = y(b - aT)$. \\

   \item \textbf{Solve the differential equation to obtain $y(t)$. What happens to
     $y(t)$ as $t \rightarrow \infty$?}

     First let's find $T(t)$...
     $$\frac{dT}{dt} = c \Rightarrow T(t) = \int c \; dt = ct + C$$
     Since $T(0) = 0$, we know $C = 0$, so $T(t) = ct$.  Now we can plug this into
     our equation for $\frac{dy}{dt}$.

     $$\frac{dy}{dt} = y(b-aT) = y(b-act) \Rightarrow \int \frac{dy}{y}
     = \int (b-act) \; dt \Rightarrow \ln{|y(t)|} = bt - \frac{1}{2}act^2 + C_1$$

     Ultimately giving us... 
     $$\mathlarger{\mathlarger{\mathlarger{y(t) = Ce^{bt - \frac{1}{2}act^2}}}}$$

     As $t \rightarrow \infty$, the exponent in the equation is becoming more and 
     more negative.  This is because $-\frac{1}{2}act^2$ is growing much faster than
     $bt$, so the exponent approaches $-\infty$, making the whole equation approach 
     zero.  This basically means that the bacteria is slowly getting wiped out by the
     toxins.

 \end{enumerate}

 \section*{Section 1.10 Question 5}
 Show that the solution $y(t)$ of the initial value problem exists on the given
 interval... 
 $$\mathlarger{y' = 1 + y + y^2\cos t, \quad y(0) = 0, \quad 0 \leq t \leq \frac{1}{3}}$$
 Existence Theorem: Consider the initial value problem $\{*\}$: 
 $\frac{dy}{dt} = f(t,y), \; y(t_0) = y_0$,
 Suppose $f$ is continuous on $R \;=\; [t_0,t_0+a] \times [y_0-b,y_0+b]$
 ($\frac{\partial f}{\partial y}$ doesn't need to be continuous because we're not 
 proving for uniqueness).  Then there exists a solution of $\{*\}$ defined on 
 $[t_0,t_0+\alpha]$, where $\alpha = \min \{a,\frac{b}{M}\}$, and 
 $M = \max |f(t,y)|$. \\

 First, we can note that $f(t,y) = 1+y+y^2\cos t$ is continuous for all $y,t \in \mathbb{R}$.
 Since $f$ is continuous on $R = [0,a] \times [-b,b]$, we can show that there exists a solution
 on $[0,\frac{1}{3}]$.
 $$M = \max |f(t,y)| = 1 + b + b^2$$
 $$\alpha = \min \Big\{a,\frac{b}{M}\Big\} = \min \Big\{a,\frac{b}{1+b+b^2}\Big\}$$
 Since $f$ is continuous for all values of $t$, $a$ can be any number, so 
 $\alpha = \min \Big\{a,\frac{b}{M}\Big\}$ is dependent on $b$.
 The largest value of $\alpha$ is when $b = 1$, so $\alpha = \frac{1}{1+1+1^2} = \frac{1}{3}$,
 Therefore, by the Existence Theorem, there must exist a solution to the initial value problem
 on the interval $0 \leq t \leq \frac{1}{3}$.


\end{document}
