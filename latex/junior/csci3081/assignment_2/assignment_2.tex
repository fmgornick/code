\documentclass[11pt]{article}

\usepackage{setspace}
\usepackage[top=2cm,bottom=2cm,left=2.5cm,right=2.5cm,marginparwidth=1.75cm]{geometry}
\usepackage[utf8]{inputenc}
\usepackage[hidelinks,urlcolor=cyan]{hyperref}
\urlstyle{same}

\title{Project Coding Style Research}
\author{Fletcher Gornick}
\date{September 27, 2021}

\spacing{1.5}
\begin{document}
\maketitle

Before giving my novice opinion on the Google C++ Style Guide, I thought I'd take a look at the 
meta surrounding it.  From what I've found, there are very mixed thoughts on Google's approach 
to how C++ code should be written.  It seems as though the public consensus is that this style 
guide is focused most on making code more compatible with Google's existing code base.  That 
matter aside, I do believe that the guide is actually very insightful with only a few minor 
exceptions.  My short answer as to if I'll use Google's C++ style directly, is no, I'll definitely 
use the guide as a sort of template to guide my process, but different projects should be approached 
in different ways.  Although following one format for projects can lead to a very flexible and 
extensible program, it's much better to approach a problem your own way, this is what ultimately 
leads to strides in the development of new technology.  But for the most part, I'll probably loosely 
follow the guide supplied by Google.  With that in mind, I'd like to mention a couple topics that 
I, along with many others, don't necessarily agree with. 

The first topic that I think should be mentioned is Google's take on exceptions.  I'm sure most 
other kids on this assignment are bringing up the same point, but it's definitely something that 
can't be ignored.  It was on a Quora post by Brian Bi where I found information about Google's 
take on exceptions \cite{brian}.  Essentially, their take is that you shouldn't use them because 
of their own code base, they say it would be a hassle to introduce exceptions to their existing 
exception-free code.  Now while they may be right that introducing exceptions now on their humungous 
project would be quite difficult, for anyone starting their own project, exceptions are a great way 
to debug errors in your program, and should definitely be adopted.

\bibliographystyle{apalike}
\bibliography{sources}
\end{document}
