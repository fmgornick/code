The main goal for Iteration 1 was to build a solid foundation. It’s important that the foundation 
of this software project is flexible to new functionality and has a fundamental logic running 
throughout its architecture. \\

We wanted our classes to be easy to understand when looking at the code.  To do this, we adopted 
some of the SOLID design principles \cite{solid}.  We wanted every class to have one job, and we 
wanted each to do it well.  We also hoped to make our filters good enough so they wouldn’t need 
to be altered, while also making it easy to extend if a new filter was needed.  One of the 
biggest things we were hoping to achieve was to keep our interfaces separate, this way it would 
be easy to add something in as well as take something away, without needing to refactor our code. \\

Most of our design inspiration came from the OpenCV \cite{geek}.  They have a very extensive library 
with tons of computer vision functionality.  They are already considered to be the gold standard in 
image processing, and they even have tons of documentation done in doxygen.  They also have an 
edge detection algorithm that utilizes the laplace transform of a kernel, which we may implement 
later to squeeze out some faster performance from our canny edge detection system for when we need 
to use it in real time \cite{laplace}.