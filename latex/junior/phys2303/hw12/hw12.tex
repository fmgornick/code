\documentclass[11pt]{article}

\usepackage{setspace}
\usepackage{amsmath}
\usepackage{enumitem}
\usepackage{amsfonts} 
\usepackage{mathtools}
\usepackage{relsize}
\usepackage{graphicx}
\usepackage[top=2cm,bottom=2cm,left=2.5cm,right=2.5cm,marginparwidth=1.75cm]{geometry}
\setlength{\parindent}{0cm}
\usepackage{listings}
\usepackage{clrscode3e}
\usepackage{graphicx}

\title{\vspace{-1.0cm}PHYS 2303 Homework 12}
\author{Fletcher Gornick}
\date{April 18, 2022}

\spacing{1.5}
\begin{document}
 \maketitle
 \section*{Chapter 6 Problem 93}
 What is the frequency of the photon absorbed when the hydrogen atom makes the transition 
 from the ground state to the \(n = 4\) state?
 \newpage

 \section*{Chapter 6 Problem 136}
 (a) Calculate the number of photoelectrons per second that are ejected from a 1.0-mm\(^2\) 
 area of sodium metal by a 500-nm radiation with intensity 1.30kW/m\(^2\) (the intensity of 
 sunlight above Earth’s atmosphere). \\

 (b) Given the work function of the metal as 2.28 eV, what power is carried away by these 
 photoelectrons?
 \newpage

 \section*{Chapter 6 Problem 141}
 The momentum of light, as it is for particles, is exactly reversed when a photon is reflected 
 straight back from a mirror, assuming negligible recoil of the mirror. The change in momentum 
 is twice the photon’s incident momentum, as it is for the particles. Suppose that a beam of
 light has an intensity 1.0 kW/m\(^2\) and falls on a -2.0 m\(^2\) area of a mirror and reflects 
 from it. \\

 (a) Calculate the energy reflected in 1.00 s. \\
 (b) What is the momentum imparted to the mirror? \\
 (c) Use Newton’s second law to find the force on the mirror. \\
 (d) Does the assumption of no- recoil for the mirror seem reasonable?
 \newpage

 \section*{Chapter 6 Problem 142}
 A photon of energy 5.0 keV collides with a stationary electron and is scattered at an angle 
 of \(60^\circ\). What is the energy acquired by the electron in the collision?

\end{document}
