\documentclass[11pt]{article}

\usepackage{setspace}
\usepackage{amsmath}
\usepackage{enumitem}
\usepackage{amsfonts} 
\usepackage{mathtools}
\usepackage{relsize}
\usepackage{graphicx}
\usepackage[top=2cm,bottom=2cm,left=2.5cm,right=2.5cm,marginparwidth=1.75cm]{geometry}
\setlength{\parindent}{0cm}
\usepackage{listings}
\usepackage{clrscode3e}
\def \n {\par \vspace{\baselineskip}}

\def\lc{\left\lceil}   
\def\rc{\right\rceil}
\def\lf{\left\lfloor}   
\def\rf{\right\rfloor}

\title{\vspace{-1.0cm}PHYS 2303 Homework 1}
\author{Fletcher Gornick}
\date{January 20, 2022}

\spacing{1.5}
\begin{document}
 \maketitle 

 \subsection*{Volume 2 Section 1.2 Problem 47}
 (a) At what temperature do the Fahrenheit and Celsius scales have the same 
 numerical value?

 (b) At what temperature do the Fahrenheit and Kelvin scales have 
 the same numerical value? \newpage

 \subsection*{Volume 2 Section 2.1 Problem 25}
 A company advertises that it delivers helium at a gauge pressure of 
 \(1.72 \times 10^7\) Pa in a cylinder of volume 43.8 L. How many balloons can be 
 inflated to a volume of 4.00 L with that amount of helium? Assume the pressure
 inside the balloons is \(1.01 \times 10^5\) Pa and the temperature in the cylinder 
 and the balloons is 25.0 °C . \newpage

 \subsection*{Volume 2 Section 2.2 Problem 43}
 The product of the pressure and volume of a sample of hydrogen gas at 0.00 °C is 
 80.0 J. 

 (a) How many moles of hydrogen are present? 

 (b) What is the average translational kinetic energy of the hydrogen molecules? 

 (c) What is the value of the product of pressure and volume at 200 °C?
 \newpage

 \subsection*{Volume 2 Section 2.2 Problem 46}
 The escape velocity of any object from Earth is 11.1 km/s. At what temperature 
 would oxygen molecules (molar mass is equal to 32.0 g/mol) have root-mean-square 
 velocity \(v_{rms}\) equal to Earth’s escape velocity of 11.1 km/s?
\end{document}
