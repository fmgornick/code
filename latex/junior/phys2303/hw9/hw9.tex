\documentclass[11pt]{article}

\usepackage{setspace}
\usepackage{amsmath}
\usepackage{enumitem}
\usepackage{amsfonts} 
\usepackage{mathtools}
\usepackage{relsize}
\usepackage{graphicx}
\usepackage[top=2cm,bottom=2cm,left=2.5cm,right=2.5cm,marginparwidth=1.75cm]{geometry}
\setlength{\parindent}{0cm}
\usepackage{listings}
\usepackage{clrscode3e}
\usepackage{graphicx}

\title{\vspace{-1.0cm}PHYS 2303 Homework 9}
\author{Fletcher Gornick}
\date{March 26, 2022}

\spacing{1.5}
\begin{document}
 \maketitle
 \section*{Chapter 4 Problem 34 (a, b, c)}
Two slits of width 2 \(\mu\)m, each in an opaque material, are separated by a center-to-center distance 
of 6 \(\mu\)m. 

A monochromatic light of wavelength 450 nm is incident on the double-slit. One finds a combined interference 
and diffraction pattern on the screen. \\

(a) How many peaks of the interference will be observed in the central maximum of the diffraction pattern? \\

(b) How many peaks of the interference will be observed if the slit width is doubled while keeping the 
distance between the slits same? \\

(c) How many peaks of interference will be observed if the slits are separated by twice the distance, 
that is, 12 \(\mu\)m.
 \newpage

 \section*{Chapter 4 Problem 81}
Quasars, or quasi-stellar radio sources, are astronomical objects discovered in 1960. They are distant but 
strong emitters of radio waves with angular size so small, they were originally unresolved, the same as stars. 
The quasar 3C405 is actually two discrete radio sources that subtend an angle of 82 arcsec. If this object is 
studied using radio emissions at a frequency of 410 MHz, what is the minimum diameter of a radio telescope 
that can resolve the two sources?
 \newpage
  
 \section*{Chapter 4 Problem 86}
How far apart must two objects be on the moon to be distinguishable by eye if only the diffraction effects 
of the eye’s pupil limit the resolution? Assume 550 nm for the wavelength of light, the pupil diameter 
5.0 mm, and 400,000 km for the distance to the moon.
 \newpage

 \section*{Chapter 4 Problem 97}
A diffraction grating with 2000 lines per centimeter is used to measure the wavelengths emitted by a hydrogen 
gas discharge tube. \\

(a) At what angles will you find the maxima of the two first-order blue lines of wavelengths 410 and 434 nm? \\

(b) The maxima of two other first-order lines are found at \(\theta_1 = 0.097\) rad and \(\theta_2 = 0.132\) 
rad . What are the wavelengths of these lines?
 \newpage

 \section*{Chapter 5 Problem 34}
(a) How long would the muon in \textbf{Example 5.1} have lived as observed on Earth if its velocity was 
0.0500c? \\

(b) How far would it have traveled as observed on Earth? \\

(c) What distance is this in the muon’s frame?
 \newpage
\end{document}
