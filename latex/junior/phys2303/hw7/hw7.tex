\documentclass[11pt]{article}

\usepackage{setspace}
\usepackage{amsmath}
\usepackage{enumitem}
\usepackage{amsfonts} 
\usepackage{mathtools}
\usepackage{relsize}
\usepackage{graphicx}
\usepackage[top=2cm,bottom=2cm,left=2.5cm,right=2.5cm,marginparwidth=1.75cm]{geometry}
\setlength{\parindent}{0cm}
\usepackage{listings}
\usepackage{clrscode3e}
\usepackage{graphicx}

\title{\vspace{-1.0cm}PHYS 2303 Homework 7}
\author{Fletcher Gornick}
\date{March 10, 2022}

\spacing{1.5}
\begin{document}
 \maketitle

 \section*{Chapter 2 Problem 48}
  An object is located in air 5 cm from the vertex of a concave surface made of glass 
  with a radius of curvature 20 cm. Where does the image form by refraction and what 
  is its magnification? Use \(n_{\text{air}} = 1\) and \(n_{\text{glass}} = 1.5\).
 \newpage

 \section*{Chapter 2 Problem 61}
  An object of height 3.0 cm is placed 5.0 cm in front of a converging lens of focal 
  length 20 cm and observed from the other side. Where and how large is the image?
 \newpage

 \section*{Chapter 2 Problem 64}
  Two convex lenses of focal lengths 20 cm and 10 cm are placed 30 cm apart, with the 
  lens with the longer focal length on the right. An object of height 2.0 cm is placed 
  midway between them and observed through each lens from the left and from the right. 
  Describe what you will see, such as where the image(s) will appear, whether they will 
  be upright or inverted and their magnifications.
 \newpage

 \section*{Chapter 2 Problem 114}
  A lamp of height 5 cm is placed 40 cm in front of a converging lens of focal length 
  20 cm. There is a plane mirror 15 cm behind the lens. Where would you find the image 
  when you look in the mirror?
\end{document}
