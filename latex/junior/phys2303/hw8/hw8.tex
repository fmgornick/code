\documentclass[11pt]{article}

\usepackage{setspace}
\usepackage{amsmath}
\usepackage{enumitem}
\usepackage{amsfonts} 
\usepackage{mathtools}
\usepackage{relsize}
\usepackage{graphicx}
\usepackage[top=2cm,bottom=2cm,left=2.5cm,right=2.5cm,marginparwidth=1.75cm]{geometry}
\setlength{\parindent}{0cm}
\usepackage{listings}
\usepackage{clrscode3e}
\usepackage{graphicx}

\title{\vspace{-1.0cm}PHYS 2303 Homework 8}
\author{Fletcher Gornick}
\date{March 20, 2022}

\spacing{1.5}
\begin{document}
 \maketitle
 \section*{Chapter 3 Problem 33}
 If 500-nm and 650-nm light illuminates two slits that are separated by 0.50 mm, how far apart 
 are the second-order maxima for these two wavelengths on a screen 2.0 m away?
 \newpage

 \section*{Chapter 3 Problem 38}
 What is the angular width of the central fringe of the interference pattern of \\
 (a) 20 slits separated by \(d = 2.0 \times 10^{-3}\) mm? \\
 (b) 50 slits with the same separation?  Assume \(\lambda = 600\) nm.
 \newpage

 \section*{Chapter 3 Problem 71}
 After a minor oil spill, a think film of oil (\(n = 1.40\)) of thickness 450 nm floats on the 
 water surface in a bay. \\ 
 (a) What predominant color is seen by a bird flying overhead? \\ 
 (b) What predominant color is seen by a seal swimming underwater?
 \newpage

 \section*{Problem 4}
 Using Euler's formula... \\
 (a) show \((\cos(\theta) + i\sin(\theta))^n = \cos(n\theta) + i\sin(n\theta)\) \\
 (b) express \(\sqrt i\) in terms of real part + imaginary part.
\end{document}
