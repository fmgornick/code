\documentclass[11pt]{article}

\usepackage{setspace}
\usepackage{amsmath}
\usepackage{enumitem}
\usepackage{amsfonts} 
\usepackage{mathtools}
\usepackage{relsize}
\usepackage{graphicx}
\usepackage[top=2cm,bottom=2cm,left=2.5cm,right=2.5cm,marginparwidth=1.75cm]{geometry}
\setlength{\parindent}{0cm}
\usepackage{listings}
\usepackage{clrscode3e}
\usepackage{graphicx}

\title{\vspace{-1.0cm}PHYS 2303 Homework 11}
\author{Fletcher Gornick}
\date{April 10, 2022}

\spacing{1.5}
\begin{document}
 \maketitle

 \section*{Chapter 5 Problem 48}
 When a missile is shot from one spaceship toward another, it leaves the first at 0.950c 
 and approaches the other at 0.750c. What is the relative velocity of the two ships?
 \newpage

 \section*{Chapter 5 Problem 103}
 (a) Show that \((pc)^2 / (mc^2)^2 = \gamma^2 - 1\). This means that at large velocities
 \(pc \;>>\; mc^2\). \\
 (b) Is \(E \approx pc\) when \(\gamma = 30.0\), as for the astronaut discussed in the twin
paradox?
 \newpage

 \section*{Chapter 5 Problem 114}
 Show that \(E^2 - p^2c^2\) for a particle is invariant under Lorentz transformations.
\end{document}
