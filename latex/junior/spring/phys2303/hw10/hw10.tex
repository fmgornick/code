\documentclass[11pt]{article}

\usepackage{setspace}
\usepackage{amsmath}
\usepackage{enumitem}
\usepackage{amsfonts} 
\usepackage{mathtools}
\usepackage{relsize}
\usepackage{graphicx}
\usepackage[top=2cm,bottom=2cm,left=2.5cm,right=2.5cm,marginparwidth=1.75cm]{geometry}
\setlength{\parindent}{0cm}
\usepackage{listings}
\usepackage{clrscode3e}
\usepackage{graphicx}

\title{\vspace{-1.0cm}PHYS 2303 Homework 10}
\author{Fletcher Gornick}
\date{April 3, 2022}

\spacing{1.5}
\begin{document}
 \maketitle
 \section*{Chapter 5 Problem 45}
In a frame S, two events are observed: event 1: a pion is created at rest at the origin and event 2: 
the pion disintegrates after time \(\tau\). Another observer in a frame S' is moving in the 
positive direction along the positive \(x\)-axis with a constant speed \(v\) and observes the same 
two events in his frame. The origins of the two frames coincide at \(t = t' = 0\). \\

Find the positions and timings of these two events in the frame S' \\
(a) according to the Galilean transformation. \\
(b) according to the Lorentz transformation.
 \newpage

 \section*{Chapter 5 Problem 76}
(a) How fast would an athlete need to be running for a 100-m race to look 100 yd long? \\

(b) Is the answer consistent with the fact that relativistic effects are difficult to observe in 
ordinary circumstances? Explain.
 \newpage

 \section*{Chapter 5 Problem 96}
Near the center of our galaxy, hydrogen gas is moving directly away from us in its orbit about a 
black hole. We receive 1900 nm electromagnetic radiation and know that it was 1875 nm when emitted 
by the hydrogen gas. What is the speed of the gas?
 \newpage

 \section*{Problem 4}
 A highway patrol officer uses a device that measures the speed of vehicles by bouncing radar off 
 them and measuring the Doppler shift. The outgoing radar has a frequency of 100 GHz and the 
 returning echo has a frequency 15.0 kHz higher. What is the velocity of the vehicle?
\end{document}
