\documentclass[11pt]{article}

\usepackage{setspace}
\usepackage[utf8]{inputenc}
\usepackage{amsmath}
\usepackage{amsthm}
\usepackage{amsfonts} 
\usepackage[hidelinks,urlcolor=cyan]{hyperref}
\urlstyle{same}
\usepackage[top=2cm,bottom=2cm,left=2.5cm,right=2.5cm,marginparwidth=1.75cm]{geometry}
\setlength{\parindent}{0cm}

\title{Density of $\mathbb{Q}$ in $\mathbb{R}$}
\author{Fletcher Gornick}
\date{November 1, 2021}

\newtheorem{theorem}{Theorem}
\newtheorem{lemma}[theorem]{Lemma}
\newcommand\e{\varepsilon}

\spacing{1.3}
\begin{document}
\maketitle
\begin{theorem}
  The set of rationals $\mathbb{Q}$ is dense in the set of real numbers $\mathbb{R}$. 
  Meaning that between any two real numbers, there exists a rational number.
\end{theorem}

\begin{proof}
  Let us first take two arbitrary real numbers $x$ and $y$ such that $x<y$. Since $x<y$, 
  we know $y-x>0$.  Then, by the Archimedean Principle, there exists an $n \in \mathbb{N}$ 
  such that $$0 < \frac{1}{n} < y-x \;\;\Rightarrow\;\; x < x + \frac{1}{n} < y$$ Now we 
  have the addition of a real number and a rational number exists between two reals, our 
  next step is to show there exists a rational number between $x$ and $y$. \\
  
  Now we can take the set $S = \{z \in \mathbb{Z} \;:\; \frac{z}{n} \leq x\}$ where $x$ 
  is our real number and $n$ is our natural number previously defined.  And since 
  $x \geq \frac{z}{n}$ for all $z \in \mathbb{Z}$, $xn$ is an upper bound. \\

  Now, since our set $S$ is a nonempty subset of $\mathbb{Z}$, and is bounded from above, 
  by \textit{\hyperref[L]{Lemma 2}}, the set contains a maximum value.  We can call this 
  max value $m \in \mathbb{Z}$.  And since $xn$ is an upper bound of $S$,
  $\frac{m}{n} \leq x$, and since $\frac{m}{n}$ is the maximum value satisfying the 
  inequality, it must also be the case that $\frac{m+1}{n} > x$, giving us 
  $$\frac{m}{n} \;\leq\; x \;<\; \frac{m+1}{n}$$
  We can take the fact that $x + \frac{1}{n} < y$ to finally show that there must exist a 
  rational number between $x$ and $y$. 
  $$x < \frac{m+1}{n} = \frac{m}{n} + \frac{1}{n} \leq x + \frac{1}{n} < y$$

  So take our rational number $q$ to be $\frac{m+1}{n}$, giving us $$x < q < y$$
  Thus, we've proved that for any two given real numbers, there exists a rational number 
  in between, and therefore, the set of rationals $\mathbb{Q}$ is dense in the set of real 
  numbers $\mathbb{R}$.

\end{proof}
\newpage

\begin{lemma}\label{L}
  Take a set $S$, if $S \neq \emptyset$, $S \subseteq \mathbb{Z}$, and $S$ is bounded above, then 
  $S$ has a maximum element.
\end{lemma}

\begin{proof}
  Since $S$ is bounded from above, by the completeness axiom, $S$ contains a supremum 
  $x \in \mathbb{R}$.  We will show that $x \in S$, that is, the supremum is the largest 
  element in $S$. \\

  Suppose, to the contrary, that $x$ is a supremum, but $x \not\in S$.  This means that 
  $x$ is strictly greater than every element in $S$, otherwise it would equal an 
  element in $S$, and therefore be in $S$.  And since the supremum is not in the set, 
  the set has no largest value. \\

  Now, since $x$ is the smallest value greater than every element of $S$, there 
  must exist an $s \in S$ such that $s > x-1$, because $x > x-1$, so $x-1$ can't be an 
  upper bound, otherwise $x$ wouldn't be our supremum.  So we get 
  $$x-1 < s < x$$
  And since our set has no largest element, there must exist an element $t$ that's bigger 
  than $s$ but less than $x$.  This gives us
  $$x-1 < s < t < x$$
  And since $S \subseteq \mathbb{Z}$ the smallest value that $t$ can be that's strictly 
  greater than $s$ is $s+1$, but this would imply that $s+1 < x$.  Since we already know 
  $x-1 < s \;\Rightarrow\; x < s+1$, we have a contradiction. \\

  Therefore, our assumption must be false, meaning that the supremum of our set $S$ is 
  in $S$.  Therefore, given a set $S$, where $S \neq \emptyset$, $S \subseteq \mathbb{Z}$, 
  and $S$ is bounded above, $S$ must contain a largest element.

\end{proof}


\end{document}
