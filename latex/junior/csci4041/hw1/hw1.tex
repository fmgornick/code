
\documentclass[11pt]{article}

\usepackage{setspace}
\usepackage{amsmath}
\usepackage{enumitem}
\usepackage{amsfonts} 
\usepackage{relsize}
\usepackage[top=2cm,bottom=2cm,left=2.5cm,right=2.5cm,marginparwidth=1.75cm]{geometry}
\setlength{\parindent}{0cm}

\def\lc{\left\lceil}   
\def\rc{\right\rceil}
\def\lf{\left\lfloor}   
\def\rf{\right\rfloor}

\title{4041 Homework 1}
\author{Fletcher Gornick}
\date{September 16, 2021}

\spacing{1.5}
\begin{document}
 \maketitle 
 \section*{6.1}
 \subsection*{6.1-1}
  \textbf{What are the minimum and maximum numbers of elements in a heap of height $h$?}

  Starting with the max number of nodes, we first know that a tree of height 0 has 1 node.
  Also, any time height goes up by 1, we add double the nodes of the previous level, so the
  number of nodes is $\sum_{k=0}^{h}2^{k} = 2^{h+1}-1$.  Since there are $2^h$ leaves in a
  full heap of height $h$, and only 1 in the most incomplete heap, then there are $2^h-1$ less
  nodes, so the least number of elements is $(2^{h+1}-1) - (2^h-1) = 2^h$.

 \subsection*{6.1-2}
  \textbf{Show that an $n-$element heap has height $\lf \lg n \rf$.}

  From the last problem, we found that the number of nodes in a heap must be in this interval:
  $2^h \leq n \leq 2^{h+1}-1$. Since $\lg{(2^h)} = h$, and $\lg{(2^{h+1}-1)} < \lg{(2^{h+1})} = h+1$,
  we know that $\lf \lg{(2^h)} \rf = \lf \lg{(2^{h+1}-1)} \rf = h$ as well as for any value $n$
  between them.
 \subsection*{6.1-3}
  Show that in any subtree of a max-heap, the root of the subtree contains the largest 
  value occurring anywhere in that subtree.
 \subsection*{6.1-4}
  Where in a max-heap might the smallest element reside, assuming that all elements are 
  distinct?
 \subsection*{6.1-5}
  Is an array that is in sorted order a min-heap?
 \subsection*{6.1-6}
  Is the array with values $\langle 23, 17, 14, 6, 13, 10, 1, 5, 7, 12 \rangle$ a max heap?
 \subsection*{6.1-7}
  Show that, with the array representation for storing an n-element heap, the leaves are 
  the nodes indexed by $\lf n/2 \rf + 1, \lf n/2 \rf + 2, ..., n$.

 \section*{6.2}
 \subsection*{6.2-3}
  What is the effect of calling MAX-HEAPIFY$(A,i)$ when the element $A[i]$ is larger than its 
  children?
 \subsection*{6.2-4}
  What is the effect of calling MAX-HEAPIFY$(A,i)$ for \textit{i $>$ A.heap-size/2}?
 \subsection*{6.2-5}
  The code for MAX-HEAPIFY is quite efficient in terms of constant factors, except possibly 
  for the recursive call in line 10, which might cause some compilers to produce inefficient 
  code. Write an efficient MAX-HEAPIFY that uses an iterative control construct (a loop) 
  instead of recursion.

 \section*{6.3}
 \subsection*{6.3-3}
 Show that there are at most $\lc n/2^{h+1} \rc$ nodes of height $h$ in any $n$-element heap.
 
\end{document}
