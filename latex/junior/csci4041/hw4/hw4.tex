\documentclass[11pt]{article}

\usepackage{setspace}
\usepackage{amsmath}
\usepackage{enumitem}
\usepackage{amsfonts} 
\usepackage{relsize}
\usepackage{graphicx}
\usepackage[top=2cm,bottom=2cm,left=2.5cm,right=2.5cm,marginparwidth=1.75cm]{geometry}
\setlength{\parindent}{0cm}

\def\lc{\left\lceil}   
\def\rc{\right\rceil}
\def\lf{\left\lfloor}   
\def\rf{\right\rfloor}

\title{4041 Homework 4}
\author{Fletcher Gornick}
\date{October 24, 2021}

\spacing{1.5}
\begin{document}
 \maketitle 
 \section*{21.3}
 \subsection*{21.3-3}
 \textbf{Give a sequence of $m$ \texttt{make-set}, \texttt{union}, and \texttt{find-set} 
 operations, $n$ of which are \texttt{make-set} operations, that takes $\Omega(m\lg n)$ time 
 when we use union by rank only.} \\ 

 First, we will note that there are $n$ elements, otherwise calling \texttt{make-set} $n$ times 
 is pointless, so we'll call these nodes $x_1, x_2, \dots , x_n$.  If we're going to introduce 
 $\lg n$ complexity into our operations, the only way I can think to do this would be to use 
 \texttt{union} to link these elements in a sort of binary tree structure.  Let us also assume 
 $n$ is a power of two, because if it's not, then we would need to call \texttt{union} extra 
 times which would only further worsen our complexity, meaning we would still be bounded by 
 $\Omega (m \lg n)$. \\

 Our sequence will first call \texttt{make-set} $n$ times.  Then we will call \texttt{union} 
 $n/2$ times to link pairs of nodes, then call it another $n/4$ times to link the pairs into 
 groups of $4$.  We will keep calling \texttt{union} until all the nodes are linked in a sort 
 of binary tree structure.  Assuming $n = 2^k$ for some $k$, the number of calls to union is 
 $\frac{n}{2} + \frac{n}{4} + \dots + \frac{n}{n} = n-1$ (we actually know that any number of 
 ways to call \texttt{union} is done in $n-1$ calls through the logic laid out in problem 
 15.2-6).  There are $k$ groups of these calls, each of which increases the tree's height by 
 $1$.  So after we finish calling \texttt{union} $n-1$ times, we have a tree containing all 
 the elements with a height of $k = \lg n$. \\

 Finally we can call \texttt{find-set} $m$ times which traverses the tree in $\lg n$ time for 
 each call, and with $m$ calls, we have a running time of $\Omega (m \lg n)$.  This is obviously 
 assuming that \texttt{find-set} doesn't implement path compression, otherwise the running-time 
 would improve with each call.  

 \newpage

 \subsection*{21.3-4}
 \textbf{Suppose that we wish to add the operation \texttt{print-set(x)}, which is given a node 
 $x$ and prints all the members of $x$’s set, in any order. Show how we can add just a single 
 attribute to each node in a disjoint-set forest so that \texttt{print-set(x)} takes time 
 linear in the number of members of $x$’s set and the asymptotic running times of the other 
 operations are unchanged. Assume that we can print each member of the set in $O(1)$ time.} \\

 Disjoint set forests already have rank variables in each of their nodes, so we can introduce 
 a new variable to make printing the whole set linear time.  We can create a list member 
 variable that contains each element in the set.  The best data structure for this list would 
 probably be a linked list, as that allows us to easily adjust the size. \\

 So every time we call \texttt{make-set(x)}, it'll add a single node linked list containing 
 element $x$, which is done in $O(1)$ time, so \texttt{make-set} keeps it's same run-time.  
 Then every time we call \texttt{union(x,y)}, it'll take the node with the 
 higher rank (or an arbitrary one if the rank is equal), and it'll become the new root, 
 appending the other set's root-node list member variable onto it's own.  So say 
 \texttt{x.rank $\geq$ y.rank}, assuming $x$ and $y$ are the roots, then the list containing 
 all the values in $x$ will get linked with the list containing all the values in $y$, thus 
 the resulting list contains all the elements in the new unioned set. This link is also an 
 $O(1)$ procedure, so \texttt{union} run-time is unchanged. Finally, since \texttt{find-set} 
 doesn't need any changes to help our \texttt{print-set} function, it's run-time is also 
 unchanged. \\

 So our \texttt{print-set(x)} function first calls \texttt{find-set(x)} to get the root node.  
 We know that the worst case for \texttt{find-set} if we're not even using union by rank is 
 $O(n)$, which most likely is not the case, but we will just say it's $O(n)$.  Next, 
 \texttt{print-set} just iterates through the linked-list member variable in the root node which 
 is again $O(n)$.  So we have \texttt{print-set(x)} is $O(n) + O(n) = O(n)$ time, and is thus 
 linear.
 \newpage

 \section*{15.2}
 \subsection*{15.2-2}
 \textbf{Give a recursive algorithm \texttt{matrix-chain-multiply(A,s,i,j)} that actually 
 performs the optimal matrix-chain multiplication, given the sequence of matrices 
 $\langle A_1, A_2, \dots , A_n \rangle$, the $s$ table computed by \texttt{matrix-chain-order}, 
 and the indices $i$ and $j$. (The initial call would be 
 \texttt{matrix-chain-multiply(A,s,1,n)}).} \\
 
 In this example, $s$ represents the auxillary table where $s[i,j] = k$, $k$ representing the 
 optimal position to split the matrices.  In other words, $A_i A_{i+1} \dots A_j$ is optimal 
 when split between $A_k$ and $A_{k+1}$, making the new multiplication 
 $(A_i \dots A_k)(A_{k+1} \dots A_j)$. \\

 Knowing this, we can now get into what our algorithm should look like.  First note that if 
 $i=j$, then $A_i$ and $A_j$ are the same matrix, so we don't need to do a multiplication, and 
 we can just return $A_i$. 
 Also if $j=i+1$ then we only have two matrices, so there's only one way to multiply them, 
 and that's by simply returning $A_i \cdot A_j$. \\

 Now is the actual recursive case.  We want the algorithm to return
 $(A_i \dots A_k)(A_{k+1} \dots A_j)$, where $k$ is denoted by $s[i,j]$, and thus $k+1$ is 
 defined as $s[i,j]+1$.  So our final case will recursively call \texttt{matrix-chain-multiply} 
 from $i$ to $s[i,j]$, and multiply it by the recursive call from $s[i,j]+1$ to $j$, giving us 
 an algorithm that looks like this...

 \begin{verbatim}
   matrix-chain-order(A,s,n)
       return matrix-chain-multiply-helper(A,s,1,n)

   matrix-chain-multiply-helper(A,s,i,j)
       if (i == j) then return A_i
       if (i == j-1) then return A_i * A_j 
       return mcmh(A,s,i,s[i,j]) * mcmh(A,s,s[i,j]+1,j)
 \end{verbatim}
 PS: I abbreviated \texttt{matrix-chain-multiply-helper} to \texttt{mcmh} to make the algorithm 
 look a little prettier.
 \newpage

 \subsection*{15.2-3}
 \textbf{Prove that $P(n) \geq \frac{1}{4}2^n$ by induction (\textit{Hint:} Base cases are 
 $n=1,2,3$).}
 \[
   P(n) = 
   \begin{cases}
     1 & \text{if } n = 1, \\
     \displaystyle\sum_{k=1}^{n-1} P(k)P(n-k) & \text{if } n \geq 2.
   \end{cases}
 \]

 Let's first look at the 3 base cases. 
$$P(1) \;=\; 1 \;\geq\; \frac{1}{4}2^1$$
$$P(2) \;=\; \displaystyle\sum_{k=1}^{1}P(k)P(2-k) \;=\; P(1)P(1) \;=\; 1 \;\geq\; 
\frac{1}{4}2^2$$
$$P(3) \;=\; \displaystyle\sum_{k=1}^{2}P(k)P(3-k) \;=\; P(1)P(2) + P(2)P(1) \;=\; 
1 + 1 \;=\; 2 \;\geq\; \frac{1}{4}2^3$$ \\

Now, since we know $P(n)$ holds for $n=1,2,3$, we can assume $P(k)$ is true for some $k > 1$, and 
show that $P(k+1)$ must be true.  And by assuming $P(k)$ to be true, I mean 
$P(k) \geq \frac{1}{4}2^k$.  We can proceed like so...
\begin{align*}
  && P(k+1) &= \displaystyle\sum_{l=1}^{k}P(l)P(k+1-l) && \\
  && &= P(1)P(k) + P(2)P(k-1) + \dots + P(k-1)P(2) + P(k)P(1) && \\
  && &\geq \frac{1}{4}2^k + P(2)P(k-1) + \dots + P(k-1)P(2) + \frac{1}{4}2^k \quad
  \text{(inductive hypothesis)} && \\
  && &\geq \frac{1}{4}2^k + \frac{1}{4}2^k && \\
  && &= \frac{1}{2}2^k && \\
  && &= \frac{1}{4}2^{k+1} &&
\end{align*}
Therefore, by the principle of mathematical induction, the statement $P(n) \geq \frac{1}{4}2^n$ 
holds for all $n \in \mathbb{N}$.
 \newpage

 \subsection*{15.2-6}
 \textbf{Show that a full parenthesization of an $n$-element expression has exactly $n-1$ pairs 
 of parentheses.} \\

 The easiest way to show that the full parenthesization of an $n$-element expression consists of 
 $n-1$ pairs of parentheses is to use induction.  So let us first look at the base case.  When 
 $n=1$ there's one element in the expression, so no multiplication is needed and therefore, no 
 parenthesization either. $n-1=0$, so the base case holds.  Let's also look at $n=2$ even though 
 it's not all that necessary, but with two matrices, they can just be multiplied together 
 normally, but we must put parentheses around the two in order to make sure they get multiplied 
 together first on an outer case, so $n=2$ leads to $1$ pair of parentheses, meaning the above 
 statement still holds. \\

 For our inductive step, we will use strong mathematical induction, so assuming the 
 parenthesization of a $k$-element expression consists of  $k-1$ pairs of parentheses, AND that 
 the parenthesization of all $l$-element expressions where $1 \leq l \leq k$ consists of $l-1$ 
 pairs of parentheses, we will show that the full parenthesization of an expression with $k+1$ 
 elements consist of $k$ pairs of parentheses. \\ 

 Let's say that the optimal split of our $k+1$-element expression $A_1 \cdot A_2 \cdots A_{k+1}$ 
 is at $A_l$, this gives us $(A_1 \cdots A_l)\cdot(A_{l+1} \cdots A_{k+1})$ As the best spot to 
 split our big expression.  These two sub-expressions also have their own optimal parenthesization 
 (including the parentheses I put on the outside of each of them).  And by our inductive hypothesis, 
 since both these sub-expressions have less than $k+1$ elements, their optimal parenthesization is 
 equal to the number of elements subtracted by $1$.  So the number of pairs of parentheses for 
 $A_1 \cdots A_l$ is $l-1$, and the number of parentheses for $A_{l+1} \cdots A_{k+1}$ is 
 $(k+1)-(l+1) = k-l$.  Now to find our how many pairs of parentheses we need for $k+1$ elements, 
 we must first add them, so we now have $l-1+k-l=k-1$ pairs of parentheses.  Finally we need one 
 more pair to group our entire expression, giving us $k-1+1=k$ pairs of parentheses, therefore, 
 the statement holds for our inductive step as well. \\

 In conclusion, by the principle of strong mathematical induction, the statement that a full 
 parenthesization of an $n$-element expression has exactly $n-1$ pairs of parentheses holds 
 for all $n \in \mathbb{N}$.
 \newpage

 \section*{15.4}
 \subsection*{15.4-3}
 \textbf{Give a memoized version of \texttt{lcs-length} that runs $O(mn)$ time.}
 \newpage

 \subsection*{15.4-5}
 \textbf{Give an $O(n^2)$-time algorithm to find the longest monotonically increasing 
 subsequence of a sequence of $n$ numbers.}
 \newpage

\end{document}
