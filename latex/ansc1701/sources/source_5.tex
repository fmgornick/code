\textbf{Source 5:} Excerpt from Part II of the book \emph{A Farewell to Alms: A Brief Economic History of the World}, by Gregory Clark \cite{clark07}.

\url{http://faculty.econ.ucdavis.edu/faculty/gclark/ecn110b/readings/ecn110b-chapter2-2005.pdf} \\

``In 1824 there were no steam powered railways in Britain. In 1850, 26 years later there were 6,000 miles of track. The modern railway was really an amalgam of an old technology – freight tramlines powered by horses – to a new power source, the steam engine. The horse powered
tramline existed long before the development of the Watt steam engine. Thus the mines in Newcastle in the northeast of England as early as 1676 were employing such tram lines to pull coal carts from the pitheads down to the wharves where the coal was loaded onto ships for
carriage to London and other markets. By 1820 the coalfields in Glamorgen in the south of Wales had 250 miles of such horse powered `rail lines' while the Newcastle district had 400 miles.
 
The Watt steam engine had no immediate effect on the development of modern railways because the Watt engine was too heavy per unit of power delivered to be used to power trains. The steam engine used for the railroad had to be much more powerful at the same size. Thus the
railroad waited on the development of the high pressure steam engine, which could only be produced if boilers could be made stronger and cylinders bored more finely.'' \\