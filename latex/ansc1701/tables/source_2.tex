{\renewcommand{\arraystretch}{2}
\begin{longtable}{ | p{3cm} | p{13cm} | }
\hline

\textbf{Source 2} &
Page 406 and 407 of \emph{The Horse : With a Treatise on Draught and a Copious Index}.
\\\hline

\textbf{Subject} &
List of factors that contribute to the annual expense of both the horse and the high-pressure locomotive steam engine.  Meant to provide further insight on which would be a better investment.
\\\hline

\textbf{Author} &
\textbf{Name:} William Youatt \n

\textbf{Point of View:} Youatt is a British veterinary surgeon who had a first hand view of many aspects of the horse, as well as how they would compare to the (at the time) newly invented steam engine.  So this information comes through the point of view of a credible veterinarian.
\\\hline

\textbf{Production} &
\textbf{When was the source produced:} Published in 1831

\textbf{Where was the source produced:} London, UK \n

\textbf{What was happening within the immediate and broader context at the time the source was produced, as it relates to the source? } \n

This book was written around the same time the steam engine was becoming more and more popular, and people started to question whether investing in a horse was worth it for specific tasks.
\\\hline

\textbf{Audience} &
\textbf{Who was the intended audience?} Other veterinarians or people just interested in learning about horses \n 

\textbf{Was it meant to be persuasive to a particular point of view? } \n

This books primary purpose was to provide information on the general history and zoological classifications of the horse, which is strictly fact based.  The excerpt that discusses annual expenses of horses and steam engines also does a very good job at weeding out the bias that can come from a veterinarian.
\\\hline
\end{longtable}}