{\renewcommand{\arraystretch}{2}
\begin{longtable}{ | p{3cm} | p{13cm} | }
\hline

\textbf{Source 5} &
Excerpt from Part II of the book \emph{A Farewell to Alms: A Brief Economic History of the World}
\\\hline

\textbf{Subject} &
General information about the industrial revolution.  The book also mentions the horses prolonged use even after the development of the steam engine.
\\\hline

\textbf{Author} &
\textbf{Name:} Gregory Clark \n

\textbf{Point of View:} Unbiased historical account of the industrial revolution.  The book was written from the perspective of a historian.
\\\hline

\textbf{Production} &
\textbf{When was the source produced:} Published in 2007

\textbf{Where was the source produced:} Princeton, NJ \n

\textbf{What was happening within the immediate and broader context at the time the source was produced, as it relates to the source? } \n

Again, since this was written in 2007, there's not too much going on in the immediate broader context, but just like I said for Source 1, the industrial revolution is always a relevant topic, especially in places like Princeton University.
\\\hline

\textbf{Audience} &
\textbf{Who was the intended audience?} Scholars and those generally interested in the industrial revolution \n 

\textbf{Was it meant to be persuasive to a particular point of view? } \n

This book is mainly a historical account of the industrial revolution, providing unbiased information to the readers, so there's no real persuasive argument.
\\\hline
\end{longtable}}