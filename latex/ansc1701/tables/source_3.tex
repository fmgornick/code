{\renewcommand{\arraystretch}{2}
\begin{longtable}{ | p{3cm} | p{13cm} | }
\hline

\textbf{Source 3} &
Painting of a carthorse being backed into a dray.  Painted in 1792 by George Garrard.
\\\hline

\textbf{Subject} &
As the steam engine slowly took away the need for mill horses, brewery production skyrocketed and the need for dray-horses to cart around the beer went up.  So this painting showcases the importance of dray-horses, as well as the relationship between them and their owners in these urban cities.
\\\hline

\textbf{Author} &
\textbf{Name:} George Garrard \n

\textbf{Point of View:} Garrard seems to be looking at the inner workings of the London Brewing Industry.  The industry's dependence on dray-horses to distribute alcohol is captured well by Garrard in this painting.
\\\hline

\textbf{Production} &
\textbf{When was the source produced:} Painted in 1792

\textbf{Where was the source produced:} London, UK \n

\textbf{What was happening within the immediate and broader context at the time the source was produced, as it relates to the source? } \n

This painting was made around the time the mill-horse was rendered obsolete by the steam engine.  As a result of the steam engine taking over the process of crushing grains in the brewing industry, the reliance on the dray-horse to distribute beer increased.  So as the steam engine rose in popularity, the number of horses in the brewing industry, surprisingly, also rose.  Please jump to Source 4 if you would like to see a graph depicting this change.
\\\hline

\textbf{Audience} &
\textbf{Who was the intended audience?} London art enthusiasts \n 

\textbf{Was it meant to be persuasive to a particular point of view? } \n

I believe this painting was more meant to be a slice of life depicting the London Brewing Industry.  There doesn't seem to be any notable bias, Garrard most likely was just in the mood to paint some urban horses.
\\\hline
\end{longtable}}