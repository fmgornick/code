{\renewcommand{\arraystretch}{2}
\begin{longtable}{ | p{3cm} | p{13cm} | }
\hline

\textbf{Source 6} &
Image excerpt of early NYC fire department from Augustine Costello's collection of fireman biographies titled \emph{Our Firemen}
\\\hline

\textbf{Subject} &
The book this picture is taken from is meant to serve as a sort of biography of early NYC firemen.
\\\hline

\textbf{Author} &
\textbf{Name:} Augustine Costello \n

\textbf{Point of View:} Perspective of a historical non-fiction writer with an admiration for the police and fire department.
\\\hline

\textbf{Production} &
\textbf{When was the source produced:} Published in 1887

\textbf{Where was the source produced:} New York City, NY \n

\textbf{What was happening within the immediate and broader context at the time the source was produced, as it relates to the source? } \n

This was written around the time NYC Fire Department was first established.  This provided many roles for horses as they were the first means of quick inner-city travel for the department.
\\\hline

\textbf{Audience} &
\textbf{Who was the intended audience?} Those interested in learning about the origins of one of the most important fire departments \n 

\textbf{Was it meant to be persuasive to a particular point of view? } \n

This book serves as a sort of biography of NYC firemen, but it also displays a bit of admiration for these fire fighters.  Therefore, this book could be meant to persuade people into supporting the NYC fire department.
\\\hline
\end{longtable}}