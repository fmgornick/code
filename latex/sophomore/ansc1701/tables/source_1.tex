{\renewcommand{\arraystretch}{2}
\begin{longtable}{ | p{3cm} | p{13cm} | }
\hline

\textbf{Source 1} &
Scholarly article on the importance of the horses in the brewing industry.
\\\hline

\textbf{Subject} &
Article that discusses the impact of horses on people in urban civilizations, and points out their importance to the economy.
\\\hline

\textbf{Author} &
\textbf{Name:} Thomas Almeroth-Williams \n

\textbf{Point of View:} A relatively unbiased view of Hanoverian London around the time of the industrial revolution.  This is a historical account, so there's not too much room for personal opinion.
\\\hline

\textbf{Production} &
\textbf{When was the source produced:} Written in 2013

\textbf{Where was the source produced:} Cambridge, UK \n

\textbf{What was happening within the immediate and broader context at the time the source was produced, as it relates to the source? } \n

Being that this is a secondary source, the broader context around the time this was produced really has no relevance to the topic of Horses in the industrial revolution.  But the industrial revolution has been an important topic ever since it was given a name.
\\\hline

\textbf{Audience} &
\textbf{Who was the intended audience?} Animal science researchers \n 

\textbf{Was it meant to be persuasive to a particular point of view? } \n

Although this article gathers sources in an unbiased manner, it is trying to convey the importance of horses during the industrial revolution as well as today.  So anyone who reads this will most likely have a deeper appreciation for our equestrian friends.
\\\hline
\end{longtable}}