{\renewcommand{\arraystretch}{2}
\begin{longtable}{ | p{2.8cm} | p{1.5cm} | p{1.7cm} | p{10cm} | }
\hline

\RaggedRight \textbf{Sources of Information} & 
Primary & Secondary &
Reason why you classified it as such? \n\n

How does the source relate to your question? \n\n

Location of the source? \newline
\\\hline\hline



\RaggedRight Scholarly article on the importance of the horses in the brewing industry
& & X &
\textbf{Secondary source criteria:} This article was written in 2013, so it's not a firsthand account.  It gathers information about equine power in London around the time period in question. \newline

\textbf{Relevance:} It's an article detailing the importance of the horse in the brewing industry which closely mirrors my topic. \newline

\RaggedRight \textbf{Location:} Research paper written by Thomas Almeroth-Williams: \n
\url{https://www.jstor.org/stable/26398226} \newline
\\\hline



\RaggedRight Page 406 and 407 of \emph{The Horse : With a Treatise on Draught and a Copious Index}
& X & & 
\textbf{Primary source criteria:} Written in 1831, provides plenty of relevant information around the end of the industrial revolution. \newline

\textbf{Relevance:} While not 100\% relevant to the topic of the horse's role in the industrial revolution, the book provide plenty of relevant information on horse's history and relation to us, as well as some important zoological classifications to keep my facts straight. \newline

\textbf{Location:} Published by Baldwin and Cradock, and written in London.  \n

\RaggedRight Found on Biodiversity Heritage Library: \newline
\url{https://archive.org/details/horsewithtreatis00youa/page/n5/mode/2up} \newline
\\\hline



\RaggedRight Painting of carthorse being backed into a dray 
& X & &
\textbf{Primary source criteria:} Painted in 1792 around the start of the industrial revolution. \newline

\textbf{Relevance:} The mill-horse was gradually replaced by steam engine, leading to a higher demand for the dray-horse. \newline

\textbf{Location:} painted engraving of Whitbread Brewery in Chiswell Street London in 1792 by George Garrard. \n

\RaggedRight found on Spartacus Educational: \newline
\url{https://spartacus-educational.com/BUwhitbread.htm} \newline
\\\hline



\RaggedRight Graph of Truman Brewery horse stock showing change of mill-horse and dray-horse use over time
& X & &
\textbf{Primary source criteria:} Graph created in 1790 around the start of the industrial revolution. \newline

\textbf{Relevance:} Same reasoning as previous source, but actually shows statistics of relative decline of mill-horse, and incline of dray-horse. \newline

\textbf{Location:} Created in London and submitted to the London Metropolitan Archives: \n\n

\url{https://horsesrevolution.wordpress.com/} \newline
\\\hline



\RaggedRight Excerpt from Part II of \emph{A Farewell to Alms: A Brief Economic History of the World}
& & X &
\textbf{Secondary source criteria:} Book written in 2007. Gathers information about general history of the Industrial Revolution. \newline

\textbf{Relevance:} Mentions horses' prolonged use in jobs such as transportation even after development of steam engine during industrial revolution. \newline

\RaggedRight \textbf{Location:} Book written by Gregory Clark for the Princeton University Press: \n
\url{http://faculty.econ.ucdavis.edu/faculty/gclark/ecn110b/readings/ecn110b-chapter2-2005.pdf} \newline
\\\hline



\RaggedRight Image of horse-drawn carriage used for early NYC fire department
& & X &
\textbf{Secondary source criteria:} Not during industrial revolution per se, but first hand account of the use of horses almost directly after the industrial revolution. \newline

\textbf{Relevance:} Shows a specific use for horses after the industrial revolution. \newline

\textbf{Location:} Picture taken in New York, in 1887. \n

\RaggedRight Found on archives.org: \newline
\url{https://archive.org/details/ourfiremenhistor00cost/page/n7/mode/1up?view=theater} \newline
\\\hline
\end{longtable}}