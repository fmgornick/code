{\renewcommand{\arraystretch}{2}
\begin{longtable}{ | p{3cm} | p{13cm} | }
\hline

\textbf{Source 4} &
Graph of the Truman Brewery horse stock over time
\\\hline

\textbf{Subject} &
The graph depicts a the population of both the mill-horse and dray horse.  Shows how around the time the steam engine was developed, the relative population of the mill-horse decreased, while the dray-horse population increased.
\\\hline

\textbf{Author} &
\textbf{Name:} Unknown \n

\textbf{Point of View:} Statistician  working in the London metropolitan area gathering information about brewing industries
\\\hline

\textbf{Production} &
\textbf{When was the source produced:} Graphed in 1759

\textbf{Where was the source produced:} London, UK \n

\textbf{What was happening within the immediate and broader context at the time the source was produced, as it relates to the source? } \n

This graph, just like the painting in Source 3 was made around the time the mill-horse was rendered obsolete by the steam engine.  This graph was most likely made to provide a sort of explanation as to why these horse populations were changing.
\\\hline

\textbf{Audience} &
\textbf{Who was the intended audience?} Other statisticians or brewery owners \n 

\textbf{Was it meant to be persuasive to a particular point of view? } \n

This graph is sheerly informational, and meant to graph the correlation between the rise of the steam engine and the relative population of certain horses.  Therefore this is not meant to persuade, it's instead meant to inform
\\\hline
\end{longtable}}