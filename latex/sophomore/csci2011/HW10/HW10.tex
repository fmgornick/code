\documentclass[10pt]{article}

\title{CSCI 2011 HW 10}
\author{Fletcher Gornick}

\usepackage{amsmath,amssymb}
\usepackage{enumitem}
\usepackage{colortbl}
\usepackage{soul}
\usepackage{xcolor}
\usepackage[margin=1.0in]{geometry}

\setlength{\parindent}{0cm}

\begin{document}
\maketitle

\section{Chapter 8.2 Problem 16}
\textbf{Each of the four sets $A_1,A_2,A_3$ and $A_4$ contains four elements.  The intersection of every $i$ of these four
sets $(2 \leq i \leq 4)$ consists of $5 - i$ elements.  What is $|A_1 \cup A_2 \cup A_3 \cup A_4|$?} \\

We know that intersections of two sets contain $5 - 2 = 3$ elements, intersection of three sets contains $5 - 3 = 2$ elements, and
the intersection of four sets has $5 - 4 = 1$ element.  By the principle of Inclusion-Exclusion, we know that the carinality of the 
union of four sets is the total of all their elements minus the sum of all the two set intersections, plus the sum of all the three
set intersections, minus the intersection of all four sets. \\

looking first at the two set intersections, we have... \\
$\Big\{\{A_1 \cap A_2\}, \{A_1 \cap A_3\}, \{A_1 \cap A_4\}, \{A_2 \cap A_3\}, \{A_2 \cap A_4\}, \{A_3 \cap A_4\}\Big\}$, since there 
are six sets, each of which have three elements, the carinality of all these sets put together is $6 * 3 = 18$. \\

Next, looking at the three set intersections... \\
$\Big\{\{A_1 \cap A_2 \cap A_3\}, \{A_1 \cap A_2 \cap A_4\}, \{A_1 \cap A_3 \cap A_4\}, \{A_2 \cap A_3 \cap A_4\}\Big\}$, there are four
sets in this case, so the total cardinality for these is $4 * 2 = 8$. \\

Finally for the intersection of all four sets, there's only one possibility which contains one element.  Therefore, by the principle
of Inclusion-Exclusion, $|A_1 \cup A_2 \cup A_3 \cup A_4| = 4 + 4 + 4 + 4 - 18 + 8 - 1 = 5$.


\section{Chapter 8.3 Problem 18}
\textbf{A man runs a grocery store in a small town. He keeps a number of \$1 bills, \$5 bills, \$10 bills and \$20 bills in 
the cash register to make change. A good friend of the man often comes to the grocery store, asking the man to change a \$100 
bill into smaller bills all of the same denomination. The store owner never knows beforehand which denomination his friend 
will request. How many bills would the man have to keep in the cash register at all times to be certain that he can satisfy 
his friend’s request?} \\

If the friend asks for \$100 all in the same denomination, then there are four choices: all ones, all fives, all tens, and all twentys.
First, if the guy asks for change in ones, then the cashier will need 100 ones.  Second, if the guy asks for change in fives, the 
cashier will need 20 fives.  Third if the guy asks for change in tens, then the cashier will need 10 tens.  Finally, if the guy asks for
change in twentys, the cashier will need 5 twentys.  Assuming we don't know what the guy will ask for, we must come prepared for all
possibilities, therefore, the cashier will need $100 + 20 + 10 + 5 = 135$ bills. \\\\


\section{Chapter 8.4 Problem 32}
\textbf{Show that $\binom{2n}{2} = 2\binom{n}{2} + n^2$.} \\

$\binom{2n}{2} = \frac{(2n)!}{2!(2n - 2)!} = \frac{(2n)(2n-1)(2n-2)!}{2(2n-2)!} = 2n^2 - n = n(n-1) + n^2 = \frac{(n)(n-1)(n-2)!}{(n-2)!} + n^2
= 2\frac{n!}{2!(n-2)!} + n^2 = 2\binom{n}{2} + n^2$


\section{Chapter 8.5 Problem 10}
\textbf{A total of 5 seniors, 3 juniors and 4 sophomores have volunteered to serve on a 4-person committee.  How many committees
are possible if}

\begin{enumerate}[label=(\alph*)]
    \item \textbf{there is no other restriction on membership for the committee?} \\
        $5 + 3 + 4 = 12$, $\binom{12}{4} = \frac{12!}{4!8!} = 495$.

    \item \textbf{at least one senior, one junior and one sophomore must serve on the committee?} \\
        Three possibilities: (1) two seniors, one junior, one sophomore.  (2) one senior, two juniors, one sophomore.  (3) one senior, one junior,
        two sophomores.  Therefore, our total is... \\
        $\binom52\binom31\binom41 + \binom51\binom32\binom41 + \binom51\binom31\binom42 = (10\cdot3\cdot4) + (5\cdot3\cdot4) + (5\cdot3\cdot6) = 270$.

    \item \textbf{at least 3 seniors must serve on the committee?} \\
        Three possibilities: (1) four seniors.  (2) three seniors, one junior.  (3) three seniors, one sophomore.  Therefore, our total is... \\
        $\binom54 + \binom53\binom31 + \binom53\binom41 = 5 + (10\cdot3) + (10\cdot4) = 75$.

    \item \textbf{at least one senior must serve on the committee?} \\
        For this case, we can subtract the inverse from our total possibilities.  So we look at the cases where there are zero seniors.  Four
        possibilities: (1) 3 juniors, 1 sophomore.  (2) 2 juniors, 2 sophomores.  (3) 1 junior, 3 sophomores.  (4) 4 sophomores.  Therefore, our total
        is \\
        $\binom{12}{4} - \binom33\binom41 - \binom32\binom42 - \binom31\binom43 - \binom44 = 495 - 4 - (3\cdot6) - (3\cdot4) - 1 = 460$.

\end{enumerate}


\section{Chapter 8 Problem 26}
\textbf{For a positive integer $n$, let $t_n$ be the number of different ways to divide $2n$ people into pairs to play $n$
tennis matches.}

\begin{enumerate}[label=(\alph*)]
    \item \textbf{Determine $t_1$ and $t_2$.} \\
        $t_1$ is the number of ways to divide two people into pairs for one match, since there are only two people to divvy up for one match, there's 
        only one possibility.  For $t_2$, there are 4 people and two matches.  There are three different ways to pair up one player with another for one
        match, and the other two players can then also be matched up together for the other match, meaning that $t_2 = 3$.

    \item \textbf{Give a recursive definition of $t_n$ for $n \geq 1$.} \\
        Every time $n$ is incremented, there are two new players added to the mix, so with $2n$ people, one person can match up with $2n - 1$ people, then 
        for all the other ways everyone else can be divvied up, that was already solved from the previous term.  Therefore $t_n = (2n-1)t_{n-1}$.

    \item \textbf{Use the recursive definition in (b) to compute $t_3$ and $t_4$.} \\
        $t_3 = (2\cdot3 - 1)t_2 = 5 \cdot 3 = 15$. \\
        $t_4 = (2\cdot4 - 1)t_3 = 7 \cdot 15 = 105$.

\end{enumerate}

\end{document}
