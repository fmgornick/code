\documentclass[10pt]{article}

\title{CSCI 2011 HW 1}
\author{Fletcher Gornick}

\usepackage{amsmath,amssymb}
\usepackage{enumitem}
\usepackage{colortbl}
\usepackage{soul}
\usepackage[margin=1.0in]{geometry}

\begin{document}
\maketitle

\section{1.2 Problem 18}
\textbf{Let \textit{P}, \textit{Q} and \textit{R} be statements. Determine whether
the following is true.} $$P \oplus (Q \oplus R) \equiv (P \oplus Q) \oplus R$$

\begin{table}[ht]
    \centering
    \begin{tabular}{|c|c|c|c|c|c|c|}

        \hline
        $P$ & $Q$ & $R$ & $P \oplus Q$ & $Q \oplus R$ & $P \oplus (Q \oplus R)$ & 
        $(P \oplus Q) \oplus R$ \\
        \hline
        \hline
        T & T & T & F & F & \cellcolor[gray]{0.8} T & \cellcolor[gray]{0.8} T \\
        \hline
        T & T & F & F & T & \cellcolor[gray]{0.8} F & \cellcolor[gray]{0.8} F \\
        \hline
        T & F & T & T & T & \cellcolor[gray]{0.8} F & \cellcolor[gray]{0.8} F \\
        \hline
        T & F & F & T & F & \cellcolor[gray]{0.8} T & \cellcolor[gray]{0.8} T \\
        \hline
        F & T & T & T & F & \cellcolor[gray]{0.8} F & \cellcolor[gray]{0.8} F \\
        \hline
        F & T & F & T & T & \cellcolor[gray]{0.8} T & \cellcolor[gray]{0.8} T \\
        \hline
        F & F & T & F & T & \cellcolor[gray]{0.8} T & \cellcolor[gray]{0.8} T \\
        \hline
        F & F & F & F & F & \cellcolor[gray]{0.8} F & \cellcolor[gray]{0.8} F \\
        \hline

    \end{tabular}
\end{table}

Since both $P \oplus (Q \oplus R)$ and $(P \oplus Q) \oplus R$ have the same truth
values, the two are logically equivalent.



\section{1.3 Problem 18}
\textbf{The \underline{inverse} of the implication of $P \Rightarrow Q$ is the
implication $(\sim P) \Rightarrow (\sim Q)$.}

\begin{enumerate}[label=(\alph*)]

    \item \textbf{Use a truth table to verify that $P \Rightarrow Q \not\equiv 
        (\sim P) \Rightarrow (\sim Q)$.}

    \begin{table}[ht]
        \centering
        \begin{tabular}{|c|c|c|c|c|c|}

            \hline
            $P$ & $Q$ & $\sim P$ & $\sim Q$ & $P \Rightarrow Q$ & 
            $(\sim P) \Rightarrow (\sim Q)$ \\
            \hline
            \hline
            T & T & F & F & \cellcolor[gray]{0.8} T & \cellcolor[gray]{0.8} T \\
            \hline
            T & F & F & T & \cellcolor[rgb]{1,0,0} F & \cellcolor[rgb]{1,0,0} T \\
            \hline
            F & T & T & F & \cellcolor[rgb]{1,0,0} T & \cellcolor[rgb]{1,0,0} F \\
            \hline
            F & F & T & T & \cellcolor[gray]{0.8} T & \cellcolor[gray]{0.8} T \\
            \hline

        \end{tabular}
    \end{table}

    $P \Rightarrow Q$ and $(\sim P) \Rightarrow (\sim Q)$ don't contain the same
    truth values in row two and three, so they are not logically equivalent.

    \item \textbf{Find another implication that is logically equivalent to
        $(\sim P) \Rightarrow (\sim Q)$ and verify your answer.}

    \begin{align*}
        && (\sim P) \Rightarrow (\sim Q) &\equiv \sim(\sim P) \vee (\sim Q) && 
        \text{(Theorem 1.48 from textbook and 3a of HW)} \\
        && &\equiv P \vee (\sim Q) && \text{(Double negation)} \\
        && &\equiv (\sim Q) \vee P && \text{(Commutative Law)} \\
        && &\equiv Q \Rightarrow P && \text{(Theorem 1.48 from textbook again)}
    \end{align*}

    \hl{$(\sim P) \Rightarrow (\sim Q) \equiv Q \Rightarrow P$} as proven by the 
    above logical equivalencies.

\end{enumerate}



\section{1.3 Problem 25}
\textbf{For two statements \textit{P} and \textit{Q}, use logical equivalencies
to verify the following.}

\begin{enumerate}[label=(\alph*)]

    \item \textbf{$P \vee Q \equiv (\sim P) \Rightarrow Q$.}

        \begin{table}[ht]
            \centering
            \begin{tabular}{|c|c|c|c|}

                \hline
                $P$ & $Q$ & $P \Rightarrow Q$ & $(\sim P) \vee Q$ \\
                \hline
                \hline
                T & T & \cellcolor[gray]{0.8} T & \cellcolor[gray]{0.8} T \\
                \hline
                T & F & \cellcolor[gray]{0.8} F & \cellcolor[gray]{0.8} F \\
                \hline
                F & T & \cellcolor[gray]{0.8} T & \cellcolor[gray]{0.8} T \\
                \hline
                F & F & \cellcolor[gray]{0.8} T & \cellcolor[gray]{0.8} T \\
                \hline

            \end{tabular}
        \end{table}

    This shows that $P \Rightarrow Q \equiv (\sim P) \vee Q$, It's also theorem
    1.48 in the textbook. Now we can use logical equivalencies to prove that 
    $P \vee Q \equiv (\sim P) \Rightarrow Q$.

    \begin{align*}
        && (\sim P) \Rightarrow Q &\equiv \sim (\sim P) \vee Q && \text{(Theorem 1.48)} \\
        && &\equiv P \vee Q && \text{(Double negation)} 
    \end{align*}

    \item \textbf{$P \wedge Q \equiv \sim (P \Rightarrow (\sim Q))$.}

    \begin{align*}
        && P \wedge Q &\equiv \sim (\sim (P \wedge Q)) && \text{(Double negation)} \\
        && &\equiv \sim ((\sim P) \vee (\sim Q)) && \text{(De Morgan's Law)} \\
        && &\equiv \sim (P \Rightarrow (\sim Q)) && \text{(Theorem 1.48)} \\
    \end{align*}
        
    \item \textbf{$\sim (P \Rightarrow Q) \equiv P \wedge (\sim Q)$.}

    \begin{align*}
        && \sim (P \Rightarrow Q) &\equiv \sim ((\sim P) \vee Q) && \text{(Theorem 1.48)} \\
        && &\equiv \sim (\sim P) \wedge (\sim Q) && \text{(De Morgan's Law)} \\
        && &\equiv P \wedge (\sim Q) && \text{(Double negation)}
    \end{align*}

\end{enumerate}



\section{1.4 Problem 12}
\textbf{For every two statements \textit{P} and \textit{Q}, use logical equivalencies
to verify the following.}

\begin{enumerate}[label=(\alph*)]

    \item \textbf{$P \Leftrightarrow Q \equiv (\sim P) \Leftrightarrow (\sim Q)$.}

    \begin{align*}
        && P \Leftrightarrow Q &\equiv (P \Rightarrow Q) \wedge (Q \Rightarrow P) && 
        \text{(Definition of a biconditional)} \\
        && &\equiv ((\sim P) \vee Q) \wedge ((\sim Q) \vee P) && \text{(Theorem 1.48)} \\
        && &\equiv (Q \vee (\sim P)) \wedge (P \vee (\sim Q)) && \text{(Commutative property)} \\
        && &\equiv ((\sim Q) \Rightarrow (\sim P)) \wedge ((\sim P) \Rightarrow (\sim Q)) &&
        \text{(Theorem 1.48)} \\
        && &\equiv (\sim P) \Leftrightarrow (\sim Q) && \text{(Definition of a biconditional)}
    \end{align*}

    \item \textbf{$P \Leftrightarrow Q \equiv (P \wedge Q) \vee ((\sim P) \wedge
        (\sim Q))$.}

    \begin{align*}
        && P \Leftrightarrow Q &\equiv (P \Rightarrow Q) \wedge (Q \Rightarrow P) &&
        \text{(Definition of a biconditional)} \\
        && &\equiv ((\sim P) \vee Q) \wedge ((\sim Q) \vee P) && \text{(Theorem 1.48)} \\
        && &\equiv ((\sim P) \wedge ((\sim Q) \vee P)) \vee (Q \wedge ((\sim Q) \vee P)) &&
        \text{(Distributive property)} \\
        && &\equiv (((\sim P) \wedge (\sim Q)) \vee ((\sim P) \wedge P)) \vee 
        ((Q \wedge (\sim Q)) \vee (Q \wedge P)) && \text{(Distributive property)} \\
        && &\equiv ((\sim P) \wedge (\sim Q)) \vee (Q \wedge P) && \text{($(\sim X) \wedge X = F$)} \\
        && &\equiv (P \wedge Q) \vee ((\sim P) \wedge (\sim Q)) && \text{(Commutative property)}
    \end{align*}

    \item \textbf{$\sim (P \Leftrightarrow Q) \equiv P \Leftrightarrow (\sim Q)$.}
        
    \begin{align*}
        && \sim (P \Leftrightarrow Q) &\equiv \sim ((P \Rightarrow Q) \wedge (Q \Rightarrow P)) &&
        \text{(Definition of a biconditional)} \\
        && &\equiv \sim (((\sim P) \vee Q) \wedge ((\sim Q) \vee P)) && \text{(Theorem 1.48)} \\
        && &\equiv (\sim ((\sim P) \vee Q)) \vee (\sim ((\sim Q) \vee P)) &&
        \text{(De Morgan's Law)} \\
        && &\equiv (P \wedge (\sim Q)) \vee (Q \wedge (\sim P)) && \text{(De Morgan's Law)} \\
        && &\equiv (P \vee (Q \wedge (\sim P))) \wedge ((\sim Q) \vee (Q \wedge (\sim P))) &&
        \text{(Distributive property)} \\
        && &\equiv ((P \vee Q) \wedge (P \vee (\sim P))) \wedge (((\sim Q) 
        \vee Q) \wedge ((\sim Q) \vee (\sim P))) && \text{(Distributive property)} \\
        && &\equiv (P \vee Q) \wedge ((\sim Q) \vee (\sim P)) && \text{($\sim X \vee X = T$)} \\
        && &\equiv (Q \vee P) \wedge ((\sim P) \vee (\sim Q)) && \text{(Commutative property)} \\
        && &\equiv ((\sim Q) \Rightarrow P) \wedge (P \Rightarrow (\sim Q)) &&
        \text{(Theorem 1.48)} \\
        && &\equiv P \Leftrightarrow (\sim Q) && \text{(Definition of a biconditional)}
    \end{align*}

\end{enumerate}



\section{1.5 Problem 10}
\textbf{Let \textit{S} and \textit{R} be two compound statements with the same
component statements. If \textit{S} is a tautology and \textit{R} is a contradiction,
then what is the truth value of the following?}

\begin{enumerate}[label=(\alph*)]
    
    \item \textbf{$S \vee R$} \\
        \text{\hl{True} (Always True $\vee$ Always False $\equiv$ Always True)}

    \item \textbf{$S \wedge R$} \\
        \text{\hl{False} (Always True $\wedge$ Always False $\equiv$ Always False)}

    \item \textbf{$S \Rightarrow R$} \\
        \text{\hl{False} (Always True $\Rightarrow$ Always False $\equiv$ Always False)}

    \item \textbf{$R \Rightarrow S$} \\
        \text{\hl{True} (Always False $\Rightarrow$ Always True $\equiv$ Always True)}

    \item \textbf{$S \Leftrightarrow R$} \\
        \text{$S \Leftrightarrow R \equiv (S \Rightarrow R) \wedge (R \Rightarrow S)$} \\
        \text{(Always False) $\wedge$ (Always True) $\equiv$ Always False} \\
        \text{\hl{False}}

\end{enumerate}


\end{document}
