\documentclass[10pt]{article}

\title{CSCI 2011 HW 8}
\author{Fletcher Gornick}

\usepackage{amsmath,amssymb}
\usepackage{enumitem}
\usepackage{colortbl}
\usepackage{soul}
\usepackage{xcolor}
\usepackage[margin=1.0in]{geometry}

\setlength{\parindent}{0cm}

\begin{document}
\maketitle


\section{6.2 Problem 8}
\textbf{For the function $f$ defined by $f(n) = \frac{n^2+1}{n+1}$ for each $n \in \mathbb{N}$, show that $f(n) = O(n)$.} \\

for $n \geq 1$,
$f(n) = \frac{n^2+1}{n+1} \leq \frac{n^2+n}{n+1} < \frac{n^2+n}{n} = n + 1 \leq n + n = 2n$.  Therefore $f(n) < 2n$ for $n \geq 1$, and so $f(n) = O(n)$.



\section{Chapter 6 Problem 12}
\textbf{Let $f: \mathbb{N} \to \mathbb{R}^+$ and $g: \mathbb{N} \to \mathbb{R}^+$ be defined by $f(n) = 2n^3 + n + 10$ and $g(n) = n^3 + 4n^2 + 1$
for $n \in \mathbb{N}$.  Show that $f = \Theta(g)$}. \\

First we show that $f = O(g)$. \\
for $n \geq 1$, $f(n) = 2n^3 + n + 10 \leq 2n^3 + n^2 + 10 < 10n^3 + 40n^2 + 10 = 10(n^3 + 4n^2 + 1)$.  Therefore $f(n) < 10 \cdot g(n)$ for $n \geq 1$, and so
$f = O(g)$. \\

Now we show that $f = \Omega(g)$ or, put more simply, $g = O(f)$. \\
for $n \geq 1$, $g(n) = n^3 + 4n^2 + 1 \leq n^3 + 4n^3 + 1 = 5n^3 + 1 < 10n^3 + 50 < 10n^3 + 5n + 50 = 5(2n^3 + n + 10)$.  Therefore $g(n) < 5 \cdot f(n)$ for
$n \geq 1$, and so $g = O(f)$. \\

Since $f = O(g)$ and $g = O(f)$, it must be the case that $f = \Theta(g)$



\section{6.2 Problem 14}
\textbf{Let $f: \mathbb{N} \to \mathbb{R}^+$, $g: \mathbb{N} \to \mathbb{R}^+$ and $h: \mathbb{N} \to \mathbb{R}^+$ be three functions.  Prove that
if $f = \Theta(g)$ and $g = \Theta(h)$, then $f = \Theta(h)$.} \\

For $n,k \in \mathbb{Z}$, $n \geq k$, $f = \Theta(g)$, so there must exist $c_1, c_2 \in \mathbb{N}$, such that $c_1g(n) \leq f(n) \leq c_2g(n)$. \\

For $n,k \in \mathbb{Z}$, $n \geq k$, $g = \Theta(h)$, so there must exist $d_1, d_2 \in \mathbb{N}$, such that $d_1h(n) \leq g(n) \leq d_2h(n)$. \\

Since $d_1h(n) \leq g(n)$, it follows that $c_1d_1h(n) \leq c_1g(n)$.  We can let $e_1 = c_1d_1$. \\
Also, since $d_2h(n) \geq g(n)$, it follows that $c_2d_2h(n) \geq c_2g(n)$.  We can let $e_2 = c_2d_2$. \\

$e_1h(n) \leq c_1g(n) \leq f(n) \leq c_2g(n) \leq e_2h(n)$ implies that $e_1h(n) \leq f(n) \leq e_2h(n)$.  Since $e_1, e_2$ are constants, it follows that
$f = \Theta(h)$. \\


\section{7.1 Problem 16}
\textbf{Prove that $4 \mid (3^{2n-1} + 1)$ for every positive integer $n$.} \\
We will proceed using an Inductive proof. \\

BASE CASE: First we show that $4 \mid (3^{2n-1} + 1)$ for the base case, where $n = 1$.  So $3^{2n-1} + 1 = 4q$ for some $q \in \mathbb{Z}$.  
$3^{2n-1} + 1 = 3^{2-1} + 1 = 3 + 1 = 4 = 4(1)$, therefore the claim holds for the base case. \\

INDUCTIVE STEP: Suppose $4 \mid (3^{2k-1} + 1)$ is true for some $k \geq 2$, we show that $4 \mid (3^{2k+1} + 1)$.  Also keep in mind that $4 \mid (3^{2k-1} + 1)$
implies that $3^{2k-1} + 1 = 4q$ for some $q \in \mathbb{Z}$.
$$3^{2k+1} + 1 = 9 \cdot 3^{2k-1} + 1 = 9 \cdot 3^{2k-1} + 9 - 8 = 9(3^{2k-1} + 1) - 8 = 9(4q) - 8 = 4(9q - 2)$$ \\
Keep in mind that the second to last step was from our inductive hypothesis.  Since $9q - 2$ is an integer, the claim holds for our inductive step as well. \\

By the principle of mathematical induction, the claim $4 \mid (3^{2n-1} + 1)$ holds for all positive integers $n$.


\section{7.2 Problem 8}
\textbf{Prove that every prime except one has the form $a^2 - b^2$ for some positive integers $a$ and $b$.} \\

First, we can express $a^2 - b^2$ as $(a + b)(a - b)$, so in order to yield a prime, we can assume that $|a| > |b|$ (otherwise $a - b \leq 0$, which cannot yield
a prime). \\

The two smallest positive integers that fit our rule is $a = 2$ and $b = 1$ because they must both be positive, and cannot be equal.  $(2 + 1)(2 - 1) = 3 > 2$.  
Since the smallest number that can be represented by this equation is $3$, it must follow that $2$ is the one prime that cannot be expressed by $a^2 - b^2$. \\

Now since we are looking for a prime outcome, and primes must be a multiple of themself and $1$, and there are no two positive integers that 
add up to $1$, it must be the case that $a - b = 1$ and $a + b$ is equal to the prime number.  Since $a - b = 1$, it follows that $a = b + 1$, so prime number 
$p = 2b + 1$, so $b = \frac{p-1}{2}$. \\

Since $p$ is prime and $p \not= 2$, $p$ must be odd, so $p - 1$ is even and $\frac{p-1}{2}$ must be an integer.  Also, $b = a - 1$, so $2a - 1 = p \Rightarrow
a = \frac{p + 1}{2}$.  Again since all primes are odd except two, which is the exception, $\frac{p+1}{2}$ must be an integer. \\

Since $a$ and $b$ are integers if $p$ is a prime (except 2), and $p = a + b$, it follows that $p$ can be expressed as $a^2 - b^2$.

\end{document}
