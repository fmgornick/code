\documentclass[fleqn]{article}
\usepackage{amsmath}
\usepackage{csci2041}

\title{Homework 6}
\author{Fletcher Gornick}\date{March 8, 2021}
\begin{document}
\maketitle

\section*{Problem 1}

\begin{code}
let rec append l1 l2 =
  match l1 with
  | [] -> l2
  | h :: t -> h :: append t l2
\end{code}

\begin{theorem}
  For any type \InlineCode{a} and any value \InlineCode{l} of type \InlineCode{a list}, we have
  \[
    \InlineCode{append l []} = \InlineCode{l}
  \]
\end{theorem}

\begin{description}
    \item[Domain] All values of type \InlineCode{a list}.
    \item[Property] \emph{P}(\InlineCode{l}): \InlineCode{append l []} = \InlineCode{l}.
    \item[Inductive order] \emph{R}(\InlineCode{l1},\InlineCode{l2}): \InlineCode{l1} is a 
        proper suffix of \InlineCode{l2}.
\end{description}

\begin{enumerate}
    \item \InlineCode{l} = \InlineCode{[]}. \\
        Left-hand side: \InlineCode{append [] []} = \InlineCode{[]}. \\
        Right-hand side: \InlineCode{[]}. \\
        Both sides agree.

    \item \InlineCode{l} = \InlineCode{h :: t}. \\
        Left-hand side:
        \[
            \InlineCode{append (h :: t) []} = \InlineCode{h :: (append t [])}
        \]

        Right-hand side:
        \[
            \InlineCode{l} = \InlineCode{h :: t}
        \]

        It is thus sufficient to show that \InlineCode{append t []} = \InlineCode{t},
        which is exactly \emph{P}(\InlineCode{t}).  \emph{P}(\InlineCode{t}) is implied
        by the inductive hypothesis because \InlineCode{t} is a proper suffix of \InlineCode{l}.
        That is, \emph{R}(\InlineCode{t},\InlineCode{l}).
\end{enumerate}
        





\section*{Problem 2 (30 points)}

\begin{code}
let rec sum l =
  match l with
  | [] -> 0
  | h :: t -> h + sum t

let rec append l1 l2 =
  match l1 with
  | [] -> l2
  | h :: t -> h :: append t l2
\end{code}
\begin{theorem}
  For any two values \InlineCode{l1} and \InlineCode{l2} of type \InlineCode{int list}, we have
  \[
    \InlineCode{sum (append l1 l2)} = \InlineCode{sum l1 + sum l2}
  \]
\end{theorem}

\begin{description}
    \item[Domain] All values of type \InlineCode{a list}.
    \item[Property] \emph{P}(\InlineCode{l1}): for any value \InlineCode{l2} of type 
        \InlineCode{a list}, \InlineCode{sum (append l1 l2)} = \InlineCode{sum l1 + sum l2}.
    \item[Inductive order] \emph{R}(\InlineCode{l1},\InlineCode{l2}): \InlineCode{l1} is a 
        proper suffix of \InlineCode{l2}.
\end{description}

\begin{enumerate}
    \item \InlineCode{l1} = \InlineCode{[]}. \\
        Left-hand side: \InlineCode{sum (append [] l2)} = \InlineCode{sum l2}. \\
        Right-hand side: \InlineCode{sum [] + sum l2} = \InlineCode{0 + sum l2} 
        = \InlineCode{sum l2}. \\
        Both sides agree.

    \item \InlineCode{l1} = \InlineCode{h :: t}. \\
        Left-hand side:
        \begin{align*}
            &  \InlineCode{sum (append (h :: t) l2)} && \\
            &= \InlineCode{sum (h :: append t l2)} && \\
            &= \InlineCode{h + sum (append t l2)}
        \end{align*}

        Right-hand side:
        \begin{align*}
            &  \InlineCode{sum (h :: t) + sum l2} && \\
            &= \InlineCode{(h + sum t) + sum l2} && \\
            &= \InlineCode{h + (sum t + sum l2)}
        \end{align*}

        It is thus sufficient to show that \InlineCode{sum (append t l2)} = 
        \InlineCode{sum t + sum l2}, which is exactly \emph{P}(\InlineCode{t}).  
        \emph{P}(\InlineCode{t}) is implied by the inductive hypothesis because 
        \InlineCode{t} is a proper suffix of \InlineCode{l1}.  That is, 
        \emph{R}(\InlineCode{t},\InlineCode{l1}).
\end{enumerate}






\section*{Bonus Problem (10 points)}

\begin{code}
let rec filter p l =
  match l with
  | [] -> []
  | h :: t -> if p h then h :: filter p t else filter p t
\end{code}

\begin{theorem}
  For any type \InlineCode{a} and any value \InlineCode{p} of type \InlineCode{a -> bool}, any value \InlineCode{l} of type \InlineCode{a list}, we have
  \[
    \InlineCode{filter p (filter p l)} = \InlineCode{filter p l}
  \]
\end{theorem}

\begin{description}
    \item[Domain] All values of type \InlineCode{a list}.
    \item[Property] \emph{P}(\InlineCode{l}): \InlineCode{filter p (filter p l)}
        = \InlineCode{filter p l}.
    \item[Inductive order] \emph{R}(\InlineCode{l1},\InlineCode{l2}): \InlineCode{l1} is a 
        proper suffix of \InlineCode{l2}.
\end{description}

\begin{enumerate}
    \item \InlineCode{l} = \InlineCode{[]}. \\
        \InlineCode{filter p (filter p [])} = \InlineCode{filter p []}.

    \item \InlineCode{l} = \InlineCode{h :: t}. \\
        \begin{align*}
            &  \InlineCode{filter p (filter p (h :: t))} && \\
            &= \InlineCode{filter p (if p h then h :: filter p t else filter p t)}
        \end{align*}

        The result critically depends on the value of \InlineCode{p h}, which is of type
        \InlineCode{bool}.  We can thus proceed by case analysis. 

        \begin{description}
            \item[Domain] All values of type \InlineCode{bool}.  That is, \InlineCode{true}
                and \InlineCode{false}.
            \item[Property] \emph{Q(b)}: If \InlineCode{p h} = \emph{b}, then
                \InlineCode{filter p (filter p (h :: t))} = \InlineCode{filter p (h :: t)}
        \end{description}

        \begin{enumerate}[(a)]
            \item \InlineCode{p h} = \emph{b} = \InlineCode{true}.
                \begin{align*}
                    &  \InlineCode{filter p (filter p (h :: t))} && \\
                    &= \InlineCode{filter p (if p h then h :: filter p t else filter p t)} && \\
                    &= \InlineCode{filter p (if true then h :: filter p t else filter p t)} && \\
                    &= \InlineCode{filter p (h :: filter p t)} && \\
                    &= \InlineCode{if p h then h :: filter p (filter p t) else filter p (filter p t)} && \\
                    &= \InlineCode{if true then h :: filter p (filter p t) else filter p (filter p t)} && \\
                    &= \InlineCode{h :: filter p (filter p t)}
                \end{align*}

            \item \InlineCode{p h} = \emph{b} = \InlineCode{false}.
                \begin{align*}
                    &  \InlineCode{filter p (filter p (h :: t))} && \\
                    &= \InlineCode{filter p (if p h then h :: filter p t else filter p t)} && \\
                    &= \InlineCode{filter p (if false then h :: filter p t else filter p t)} && \\
                    &= \InlineCode{filter p (filter p t)}
                \end{align*}
        \end{enumerate}

        In both cases, it is sufficient to show that \InlineCode{filter p (filter p t)} = 
        \InlineCode{filter p t}, which is exactly \emph{P}(\InlineCode{t}).  \emph{P}(\InlineCode{t})
        is implied by the inductive hypothesis because \InlineCode{t} is a proper suffix of
        \InlineCode{l}.  That is, \emph{R}(\InlineCode{t},\InlineCode{l}).
\end{enumerate}

\end{document}
