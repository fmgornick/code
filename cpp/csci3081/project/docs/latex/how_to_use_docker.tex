\hypertarget{how_to_use_docker_why_we_use_docker}{}\section{Why We Use Docker}\label{how_to_use_docker_why_we_use_docker}
Docker is used in order for all group members to have access to the same libraries and same development environment. Because of docker we all have the same access to libraries like open\+CV and all the libaries concerning the web application. Without docker, we would all have to try and make the same environment on our local machines which could cause major issues when pushing code to develop that may work on one of our local environments but not the others. Docker is a great way for group work to stay standardized and allows for the development of the project to go quickly and be able to be realeased quicker.\hypertarget{how_to_use_docker_how_we_use_docker}{}\section{How We Use Docker}\label{how_to_use_docker_how_we_use_docker}
In our project, we have a docker file that allows us to complie all the libraries that we need in the same way on all of our local machines. We run the command bin/build-\/env.\+sh. This builds our docker image based off of what our docker page says. After doing that, we all have the ability to run a container that is the exact same on all of our local machines. Once the environment is built we don\textquotesingle{}t have to build it every time. Therefore if we want to run the container we run the command bin/run-\/env.\+sh and then you can open the container on your text editor and run the environment. This allows us to make changes that will work on all of our local machines.\hypertarget{how_to_use_docker_how_to_run_our_project_using_docker}{}\section{How To Run Our Project Using Docker}\label{how_to_use_docker_how_to_run_our_project_using_docker}
In order to run our project, navigate to our docker submission on docker hub. The U\+Rl is \href{https://hub.docker.com/r/gorni025/sim}{\tt https\+://hub.\+docker.\+com/r/gorni025/sim}. Run the command \char`\"{}docker pull gorni025/sim\char`\"{} To then run the simulation run the command \char`\"{}docker run -\/-\/name=sim -\/p 127.\+0.\+0.\+1\+:8080\+:8081 -\/d gorni025/sim\char`\"{}, and to see the simulation navigate to \char`\"{} http\+://127.\+0.\+0.\+1\+:8080/\char`\"{} in your browser, and to see our documentation navigate to \char`\"{}http\+://127.\+0.\+0.\+1\+:8080/docs/\char`\"{}. To stop the simulation run the command \char`\"{}docker kill sim\char`\"{} then \char`\"{}docker rm sim\char`\"{}. You can run the following line to enter the docker image and see it\textquotesingle{}s contents, \char`\"{}docker run -\/it -\/-\/entrypoint /bin/bash gorni025/sim\char`\"{} Then simply type exit to get out of the image. If you want to try running the simulation on a C\+SE labs machine, you can first run this command to ensure the simulation will run on your local machine... \char`\"{}ssh -\/\+L 8081\+:127.\+0.\+0.\+1\+:8081 $<$x500$>$@$<$machine$>$.\+cselabs.\+umn.\+edu\char`\"{} Then you can pull and run the docker image with these commands \char`\"{}singularity pull docker\+://gorni025/sim\char`\"{}, \char`\"{}singularity run docker\+://gorni025/sim\char`\"{} Then navigate to \href{http://127.0.0.1:8081/}{\tt http\+://127.\+0.\+0.\+1\+:8081/} to see the simulation, and press C\+T\+R\+L-\/C to kill it. If instead, you\textquotesingle{}d like to run our whole project using our docker environment, yo can first clone our repository \char`\"{}git clone https\+://github.\+umn.\+edu/umn-\/csci-\/3081-\/f21/repo-\/team-\/0.\+git\char`\"{} Then, in the repo-\/team-\/0 directory, you can build and run the docker environment with these commands \char`\"{}bin/build-\/env.\+sh\char`\"{}, \char`\"{}bin/run-\/env.\+sh\char`\"{}. Now that you\textquotesingle{}re in the environment with all the dependencies, you can visit This Page To see how to make and run everything. 