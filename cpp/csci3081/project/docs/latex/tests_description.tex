\hypertarget{tests_description_Testing}{}\section{Overview}\label{tests_description_Testing}
The tests are what allow us to check the accuracy of our code, without tests we would not be able to find different issues in our code-\/ simply relying on a successful compile is both ignorant and silly. We use a combination of unit tests, component tests, integration tests, and regression tests to test the accuracy of our code.

Unit tests test one unit at a time, allowing us to easily determine when something specific is going wrong in a specific piece of code. Unit tests are super useful in the beginning stage of your testing process, when you are writing a new \textquotesingle{}unit\textquotesingle{} of code you use various unit tests to test whether your code is acting as you would expect it to.

Component tests are virtually identical in nature to unit tests but they use real data rather than made up or \char`\"{}dummy\char`\"{} data to test code. Component tests are used in the second stage of your testing process, using real data more accurately shows how a piece of code will interact with your program and gives you a better image of how your code is really working.

Integration tests check to see if multiple units work together in tandem-\/ they basically make sure nothing breaks when you combine certain units together. Integration tests are the third step in the testing progess, you must test whether your new addition into your program doesn\textquotesingle{}t break your existing pieces of code.

Regression tests check if certain changes to existing code have caused issues with anything else anywhere else within the code by testing it against known scenarios. Regression tests basically test if the units have been integrated correctly. Regression tests are the last step in the testing progress, these tests basically mimic how your code will run all together as a whole.

Together, these tests provide an exhaustive look into how our code looks both as single parts on their own and each of these individual parts working together to form a whole. Without these tests, you could find bugs when using the code later on and without tests it would be far more difficult to find where exactly an issue is happening. Especially in the case where the issue lies not in a certain piece of code but rather in where pieces of code are working together. Thoughtful testing also helps in the extendability of this project, these tests can provide a helpful template for writing new tests in the future and can also be useful for people looking at the code and trying to figure out how to show what it is doing or run commands.

Ultimately, while I hope what I have written above has shown this I want to clearly state-\/ T\+E\+S\+T\+I\+NG IS S\+U\+P\+ER I\+M\+P\+O\+R\+T\+A\+NT. Having good and all-\/encompassing tests is an integral part of the coding process and helped us greatly in this project to see how each of our individual pieces (units) of code worked alongside each others. 