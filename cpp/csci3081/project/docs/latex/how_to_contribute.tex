\hypertarget{how_to_contribute_adding_a_feature}{}\section{Adding a Feature}\label{how_to_contribute_adding_a_feature}
Once you\textquotesingle{}ve decided upon a feature you\textquotesingle{}d like to add, navigate to the team Git\+Hub page. From there, click on the \char`\"{}\+Issues\char`\"{} tab. In the \char`\"{}\+Issues\char`\"{} page, click on the \char`\"{}\+New Issue\char`\"{} Button and fill the title with the relevant name and bracketed Git\+Hub key with point value. Then, fill in the description with any additional necessary information, assign the issue to a group member, and update the issue with any \char`\"{}\+Project\char`\"{} or \char`\"{}\+Milestone\char`\"{} tags if applicable. Finally, click the \char`\"{}\+Submit New Issue\char`\"{} button to create the issue.\hypertarget{how_to_contribute_creating_a_local_branch}{}\section{Creating a Local Branch}\label{how_to_contribute_creating_a_local_branch}
Once the issue has been created, you can now create the branch that you will develop the feature on. It\textquotesingle{}s important to maintain good source source control and ensure that working versions of code are not overwritten by dysfunctional code that\textquotesingle{}s still under development. The easiest way to create the branch is through the terminal. In your terminal, navigate to your cloned version of the repository. Run the command \char`\"{}git status\char`\"{} to verify which branch you\textquotesingle{}re currently on. If you\textquotesingle{}re not currently on \textquotesingle{}develop\textquotesingle{}, stash or commit any changes you\textquotesingle{}ve made to your branch, and then use \char`\"{}git checkout develop\char`\"{} to switch to the \textquotesingle{}develop\textquotesingle{} branch. Then, enter \char`\"{}git pull\char`\"{} to ensure you\textquotesingle{}re working on the most up-\/to-\/date version of the \textquotesingle{}develop\textquotesingle{} code. Finally, enter \char`\"{}git checkout -\/b \textquotesingle{}your issue name here\textquotesingle{}\char`\"{} to create the new branch.\hypertarget{how_to_contribute_coding_principles}{}\section{Coding Principles}\label{how_to_contribute_coding_principles}
Now that you\textquotesingle{}ve created a new branch, it\textquotesingle{}s imporant to follow principled coding standards. Be sure to be consistent with the naming conventions established across the code base. Make sure to leave clear comments in the class and header files that you write. It is important that memory is being properly managed, so check that your programs are free of memory leaks with Valgrind. Be sure to use the \textquotesingle{}Big Three\textquotesingle{} of the copy constructor, destructor, and assignment operator. Object oriented programming is important for this project, so be sure to follow procedure with encapsulation, as well as inheritance and polymorphism, which feature prominently in the code. It is expected that you follow S\+O\+L\+ID design principles, i.\+e. Single Responsibility for classes, Open/\+Closed Principle, etc. Also, follow safe coding practices by using const references and methods when possible, and be sure to keep interfaces separated.\hypertarget{how_to_contribute_pull_requests}{}\section{Pull Requests and Code Review}\label{how_to_contribute_pull_requests}
Finally, when you have the added functionality running correctly and without memory leaks on your local machine, use \textquotesingle{}git commit\textquotesingle{} and \textquotesingle{}git push\textquotesingle{} to push your code to the upstream branch. Then, create pull request, requesting to merge your branch into develop. When creating the pull request, make the name of the pull request the same as the issue that you\textquotesingle{}re working on. Write a short description describing the changes and what\textquotesingle{}s been added, and include the \#tag that corresponds to the issue. Request review from one or more other developers, and then submit the request. The review process of the pull request is carried out by a teammate, who goes through the changed files, verifying that the they\textquotesingle{}ve viewed the changes and that said changes appear to follow the coding principles in the above section, as well as that the changes appear to accomplish what the created issue states. Comments will be left in any areas of significance or concern, and depending on whether or not the code is ready for deployment, the request will either be approved and merged, or closed. Should the request be closed, the developer will address comments left in the review, and make changes and do further testing on the local branch before starting the pull request and review process over again. 