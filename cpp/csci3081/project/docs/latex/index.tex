\hypertarget{index_intro_sec}{}\section{The Droner\textquotesingle{}s Drone Project}\label{index_intro_sec}
This is a project developed in C++ by Gabe Fendrich, Fletcher Gornick, Peyton Johnson, Luke Wiskus and Jason Woitalla. The purpose of this project was to solve a problem presented in our computer science class. We have been given a simulation where a drone, robot, hospital and recharge station are present. Our program runs the backend of this simulation to move the drone with the end goal of locating the robot. We do this using two major subsystems. Fist we have the \href{image_processing_description.html}{\tt image processing subsystem} and next we have the \href{simulation_description.html}{\tt simulation backend subsystem}.

To actually run the project, you can view our \href{getting_started.html}{\tt getting started with the libraries page} which details how to build, the code. It also links to our github \href{https://github.umn.edu/umn-csci-3081-f21/repo-team-0}{\tt https\+://github.\+umn.\+edu/umn-\/csci-\/3081-\/f21/repo-\/team-\/0} which has more information on cloneing and contributing to the project. This project is open source, so if you would like to contribute to it please read our \href{how_to_contribute.html}{\tt How to Contribute page}.

Another tool we used for this project was deploying it to docker. More information can be seen on our, \href{how_to_use_docker.html}{\tt how we use docker page}. Docker is a container based development environment that allows us to build this project with little overhead because it just contains the dependencies we need. The project can then be built or contributed to using docker.

Once our docker container is running, you can view the simulation in action by going to \href{http://127.0.0.1:8081/}{\tt http\+://127.\+0.\+0.\+1\+:8081/} . Or go to \href{http://127.0.0.1:8080/}{\tt http\+://127.\+0.\+0.\+1\+:8080/} if you\textquotesingle{}re running our actual docker image on \href{https://hub.docker.com/r/gorni025/sim}{\tt https\+://hub.\+docker.\+com/r/gorni025/sim}. To see more about what our project can do go to the \href{overview.html}{\tt overview page}. It details that this simulation contains a drone that is patrolling the scene for a robot. The drone knows it found the robot when the image processing library reports a hit back to the drone. From there, the drone deploys a rescue drone to go save the robot, and return it to the hospital.

To read more about the technologies that power our image processing library, \href{image_processing_description.html}{\tt go here}.

To read more about the technologies that power our drone simulation, \href{simulation_description.html}{\tt go here}. 