\hypertarget{simulation_description_Simulation}{}\section{Overview}\label{simulation_description_Simulation}
The simulation is what allows you to visualize the project.

To see visualization using docker-\/
\begin{DoxyEnumerate}
\item Build docker image\+: bin/build-\/env.\+sh
\item Run docker image\+: $\ast$bin/run-\/env.sh //\+Usage bin/run-\/env.\+sh $<$port -\/ optional(default 8081)$>$
\item Build project web server (inside docker image). (You won\textquotesingle{}t be able to cd here yet because the project directory does not exist. If you were able to launch the above commands you should now be inside the docker image). You can exit it with C\+T\+R\+L+D now\+: $\ast$cd /home/user/repo/project $\ast$make -\/j
\item Run web server (within project/directory inside docker image)\+: $\ast$cd /home/user/repo/project $\ast$./build/web-\/app 8081 web
\item Open up Firefox and browse to \href{http://127.0.0.1:8081/}{\tt http\+://127.\+0.\+0.\+1\+:8081/}
\end{DoxyEnumerate}

To see visualization using S\+SH into C\+SE Labs machines-\/
\begin{DoxyEnumerate}
\item S\+SH into a C\+SE Lab Machine using port forwarding for the UI. (If port 8081 is not available, choose a different port (8082, 8083, etc))\+: $\ast$ssh -\/L 8081\+:127.\+0.\+0.\+1\+:8081 \href{mailto:x500@csel-xxxx.cselabs.umn.edu}{\tt x500@csel-\/xxxx.\+cselabs.\+umn.\+edu} //ssh -\/L 8081\+:127.\+0.\+0.\+1\+:8081 \href{mailto:x500@csel-kh1250-05.cselabs.umn.edu}{\tt x500@csel-\/kh1250-\/05.\+cselabs.\+umn.\+edu}
\item Compile the project (within ssh session)\+: $\ast$cd /path/to/cloned/repository $\ast$cd project $\ast$make -\/j
\item Run project (within project/directory inside ssh session) $\ast$./build/web-\/app 8081 web
\item Open up Firefox and browse to \href{http://127.0.0.1:8081/}{\tt http\+://127.\+0.\+0.\+1\+:8081/}
\end{DoxyEnumerate}

To see visualization using C\+SE Labs machines/\+V\+O\+L\+E-\/ V\+O\+LE\+: \href{https://cse.umn.edu/cseit/self-help-guides/virtual-online-linux-environment-vole}{\tt https\+://cse.\+umn.\+edu/cseit/self-\/help-\/guides/virtual-\/online-\/linux-\/environment-\/vole}
\begin{DoxyEnumerate}
\item Build project\+: $\ast$cd /path/to/cloned/repository $\ast$cd project $\ast$make -\/j
\item Run project (within project/directory)\+: $\ast$./build/web-\/app 8081 web
\item Open up Firefox and browse to \href{http://127.0.0.1:8081/}{\tt http\+://127.\+0.\+0.\+1\+:8081/}
\end{DoxyEnumerate}

The above is taken from the beta code read me page-\/ (\href{https://github.umn.edu/umn-csci-3081-f21/shared-upstream/tree/support-code/project}{\tt https\+://github.\+umn.\+edu/umn-\/csci-\/3081-\/f21/shared-\/upstream/tree/support-\/code/project})

The simulation implements all of our files in tandem with the web app and simulation facade. The simulation facade is the main class for the simulation, it starts by creating a simulation facade object. From there, the most pertinant fuctions are the ones to create entities from the J\+S\+ON scene from the web app and populate the entities arrays with the J\+S\+ON object, add entities to their respective arrays, and update the simulation after any of these functions are implemented. For more information on these there is commenting in the simulation\+\_\+facade page.

There are a couple main drawbacks to our simulation\+:

-\/\+The simulation is not accessable with certain S\+SH environments. When you S\+SH into a labs computer on your personal computer if the environment you are using doesn\textquotesingle{}t support opening a browser within the session (VS Code for example) you cannot reach the webpage.

-\/\+There is a camera delay such that the camera takes a photo at a frequency of 2.\+5 seconds which then causes a delay in the simulation, meaning the simulation does not update instantaneously.

-\/\+The patrol pattern is less comprehensive than we may have hoped as if the robot is behind or in/on a building it will not recognize that.

The most problematic issues with our simulation are the camera delay and patrol pattern. Their effects would be far more obvious and detrimental if we were to actually use this drone for \textquotesingle{}real life\textquotesingle{} search and rescue rather than just finding a robot in the simulation. The camera delay partnered with the speeds of various other functions could cause issues since given that the camera isn\textquotesingle{}t taking photos in constant time there is a possibility the program as a whole could miss capturing the individual that is meant to be rescued. As an example if the speed of the drone was to increase to 100mph (which I read is the speed limit for drones) it would be travelling over 100 feet per second (rounded down significantly to account for various variables the drone might run into in the air that may slow the drone down) and if you are taking a picture every 2.\+5 seconds that would mean you would need to have the drone high enough to have the image cover an area of at least 250 feet. In that example if the speed or height of the drone was off you could miss capturing a photo of the target completely and the most reasonable way to try to fix that would be to find a speed and a way in which the drone could take photos in constant time. Similary, the patrol pattern given the issues with the buildings could also prevent us from \char`\"{}seeing\char`\"{} and thus finding our target.

With these two examples it is easy to see the importance of all of the different parts of this project working together, especially when trying to create a smooth simulation. To see the complete structure of how our simulation uses each of the files you can access the U\+ML diagrams generated on our docker page, however as a brief overview if we start at the simulation facade, then when creating objects we make use of the web app and entity factory pages, the web app gives us the information from the J\+S\+ON scene and the entity factory pages are used for making new entities. Implementation for these is actually carried out by the specific entities factories. As a sort of hierarchy it goes from the web app, simulation facade, entity factories, then to the find robot sequence which uses camera, the various filters, movement techniques, and search strategies to-\/ you probably guessed it already, find the robot!

The inheritance diagrams are very useful for visualizing how the simulation works, I suggest looking at the diagrams generated for the web app or simulation facade while you\textquotesingle{}re reading to actually see what I\textquotesingle{}m talking about here. 